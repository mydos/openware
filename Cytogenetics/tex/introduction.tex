% Copyright © 2012 Edward O'Callaghan. All Rights Reserved.

\section{Introduction}
\label{sec:introduction}

Cytogenetics is the subbranch of genetic research that is
primarily concerned with the structural functionality of
a cells constituent components. In particular, the cells
chromosomes are of particular focus. This is because the
chromosomes are the constituent of the cell that contains
its genetic code and the structural functionality plays
a critical role in epigenetic expression of the encoded
genes.

One particular routine analytic method used commonly
is the study of \emph{G-banded chromosomes}. Giemsa banding
or \emph{G-banding} is technique to produce visible 
\emph{karyotypes} by staining condensed chromosomes. The
term \emph{karyotypes} is the Greek word for kernel type or
nucleus type, we shall make this language clear later.
The stain involved is \emph{Giemsa} and hence the name
G-bandeding. The stain Giemsa is named after a German
scientist who first used it however the compound is actually
a mixture of methylene blue, eosin (named after the Greek
god Eos of the morning dawn and so eosin is red) and azure B. 
