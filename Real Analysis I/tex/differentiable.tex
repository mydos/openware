% Copyright © 2012 Edward O'Callaghan. All Rights Reserved.

\section{Differentiable} % (fold)
\label{sec:differentiable}

\begin{defn}[Differentiable]
	Let $f: \Omega \to \R^m$ be a function defined
	on an open set $\Omega \in \R^n$ and point
	$\vec{a} \in \Omega$. The function $f$ is said
	to be $\emph{differentiable}$ at $\vec{a}$ if
	there is a linear transformation $T: \R^n \to \R^m$
	such that,
	\[
		\lim_{\mathbf{x} \to \vec{a}}
		\frac{\|f(\mathbf{x}) - f(\vec{a}) - T(\mathbf{x} - \vec{a})\|_{m}}
		{\|\mathbf{x} - \vec{a}\|_{n}} = 0.
	\]
	If $f$ is differentiable at every point $\vec{a} \in \Omega$
	then $f$ is differentiable $\emph{on}$ $\Omega$.
\end{defn}

\begin{thm}[Linear Approximation]
	The function $f: \Omega \subseteq \R^n \to \R^m$ is
	differentiable at point $\vec{a}$ if and only if
	there is a function $\epsilon(\mathbf{x})$ so that
	for $\mathbf{x} \in \Omega$ we have,
	\[
		f(\mathbf{x}) = f(\vec{a}) + T(\mathbf{x} - \vec{a})
		+ \epsilon(\mathbf{x}) \| \mathbf{x} - \vec{a} \|
	\]
	with $\epsilon(\mathbf{x}) \to 0$ as $\mathbf{x} \to \vec{a}$.
\end{thm}

\begin{proof}
	\begin{align*}
		\intertext{Set}
		\epsilon(\mathbf{x}) &=
		\frac{f(\mathbf{x}) - f(\vec{a}) - T(\mathbf{x} - \vec{a})}
		{\| \mathbf{x} - \vec{a} \|} : \mathbf{x} \neq \vec{a}.
		\intertext{Now, if $f$ is differentiable at $\vec{a}$, then
		$\lim_{\mathbf{x} \to \vec{a}} \epsilon(\mathbf{x}) = 0$.}
		\intertext{Conversely, suppose}
		f(\mathbf{x}) &= f(\vec{a}) + T(\mathbf{x} - \vec{a})
		+ \epsilon(\mathbf{x}) \| \mathbf{x} - \vec{a} \|
		\intertext{holds, and since $\mathbf{x} \neq \vec{a}$, we have}
		& \frac{f(\mathbf{x}) - f(\vec{a}) - T(\mathbf{x} - \vec{a})}
		{\| \mathbf{x} - \vec{a} \|} = \epsilon(\mathbf{x}) \to 0
		\intertext{as $\mathbf{x} \to \vec{a}$ and so $f$ is differentiable
		at the point $\vec{a}$.}
	\end{align*}
\end{proof}

\begin{thm}[Chain Rule]
	Let $\Omega$ be a open set in $\R^n$ and $f: \Omega \to \R^m$
	and $g: U \to \R^p$, where $U$ is a open set in $\R^m$ with
	$f(\Omega) \subseteq U$. If $f$ is differentiable at
	$\vec{a} \in \Omega$ and $g$ is differentiable at $f(\vec{a})$,
	then $g \circ f$ is differentiable at $\vec{a}$ and
	\[
		D_{(g \circ f)}(\vec{a}) = D_{g}(f(\vec{a})) D_{f}(\vec{a}).
	\]
\end{thm}

\begin{proof}
	TODO
\end{proof}

\subsection{Partial derivatives}

\begin{defn}[Directional Derivative]
	The directional derivative of $f$ at $\vec{a}$ in the direction
	of a non-zero vector $\vec{u} \in \R^n$, denoted by
	$D_{\vec{u}} f(\vec{a})$ is defined by,
	\[
		D_{\vec{u}} f(\vec{a}) = \lim_{t \to 0}
		\frac{f(\vec{a} + t \vec{u}) - f(\vec{a})}{t},
	\]
	whenever the limit exists.
\end{defn}

\begin{thm}
	If $f: \Omega \subseteq \R^n \to \R$ is differentiable at
	$\vec{a} \in \Omega$, then for any direction non-zero
	$\vec{u} \neq 0$, $\vec{u} \in \R^n$,
	$D_{\vec{u}} f(\vec{a})$ exists and
	\[
		D_{\vec{u}} f(\vec{a}) =
		\langle \nabla f(\vec{a}), \vec{u} \rangle .
	\]
\end{thm}

\begin{proof}
	TODO
\end{proof}
