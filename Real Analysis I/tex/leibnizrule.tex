% Copyright © 2013 Edward O'Callaghan. All Rights Reserved.
% !Tex root = Real Analysis I.tex

\section{Leibniz Rule} % (fold)
\label{sec:leibnizrule}

We begin by generalising the product rule in the following ways
and building a general result about integration by parts for
vector valued functions called the \emph{Leibniz Rule}.

\begin{prop}
  Suppose that $f,g \in \mathcal{C}^n(\R)$ are n-times differentiable.
  Then the $\nth$ derivative of the product $f \circ g$ is given by,
  \[
    (f \circ g)^n = \sum_{k=0}^{\infty} \binom{n}{k} f^{k} g^(n-k).
  \]
\end{prop}

\begin{proof}
  TODO.. by mathematical induction..
\end{proof}

Even more generally we can show that the product rule in this way
holds for vector valued functions.

\begin{prop}
  Suppose that the vector valued functions $\vec{f},\vec{g} \in \mathcal{C}^k(\R^n)$.
  Then the $k^{th}$ partial derivative of the product $\vec{f} \circ \vec{g}$ is given by,
  \[
  \delta^{\alpha} (\vec{f} \circ \vec{g}) =
  \sum_{\beta : \beta \leq \alpha} \binom{\alpha}{\beta} (\delta^{(\alpha - \beta)} \vec{f}) (\delta^{\beta} \vec{g})
 \]
 where we make use of the \emph{multi-index} notation that,
 $\alpha = (\alpha_1, \alpha_2, \dots, \alpha_n)$ and $\beta = (\beta_1, \beta_2, \dots, \beta_n)$
 are both n-tuples.
\end{prop}

\begin{proof}
  TODO..
\end{proof}
