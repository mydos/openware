\section{Prelude} % (fold)
\label{sec:purelude}
First a word on how to study category theory. Category theory is extremely heavy on new terminology at first sight.
It is typically misconceived as abstractly and arbitrarily renaming things to no real gain and a generally poorly understood
topic in undergraduate mathematics. Consider taking the time to highlight these notes in assortment of coloured pens,
with each new term given a corresponding colour. To get over the initial shock of, perhaps, new terms and to consolidate concepts.
Also note that everything may be represented as a diagram and that we make the notion of a diagram rigous later. However,
a categorial diagram is simply a directed graph, where the nodes are objects from a category while morphism are the edges.
I highly recommend you copy each new term with a corresponding diagram to anecdote the idea at hand.
If all else fails, be sure to remember to break the term down into prefix and suffix and reason it.
Typically terms are fairly simple to reason out (e.g., mono = ``one'', auto = ``self'', morphism = ``to shape'').
Category theory is actually a generalisation of set theory and so, we `stress' that a class of objects is not
nessarily a set and is considered more abstract. Keep this in mind when we look at `small' and `large' categories
we shall see later.

\begin{note}[Class]
 A $\emph{class}$ is a collection of mathematical objects (e.g., sets)
 which can be unambiguously defined by a property that all its members share.
\end{note}

\begin{note}[Morphism]
 A $\emph{morphism}$, from Greek, is an abstraction derived from
 structure-preserving mappings between two mathematical structures.
 The notion of morphism recurs in much of contemporary mathematics.
\end{note}