\section{Natural Transformations} % (fold)
\label{sec:natural transformations}
A natural transformation provides a way of transforming one functor into another
while respecting the internal structure (i.e. the composition of morphisms) of the categories involved.
Hence, a natural transformation can be considered to be a ``morphism of functors''.
Indeed this intuition can be formalized to define so-called functor categories.
Natural transformations are, after categories and functors, one of the most basic notions of category theory
and consequently appear in the majority of its applications.

\begin{defn}[Natural Transformation]
 Let $F$ and $G$ be functors both from the category $\mathcal{C}$ to $\mathcal{K}$.
 Then a $\emph{natural transformation}$ $\eta : F \to G$ associates to every object
 $X \in \mathcal{C}$ a morphism $\eta_{X} : F(X) \to G(X) \in \mathcal{K}$, called the $\emph{component}$
 of $\eta$ at $X$. The component morphism $\eta_{X}$ is subject to the condition of $\emph{naturality}$,
 that is, the diagram commutes for every morphism $\varphi: X \to Y \in \mathcal{C}$; or simply,
 $\eta_{Y} \circ F(\varphi) = G(\varphi) \circ \eta_{X}$.
 \\
 We draw the commutative diagram in the category $\mathcal{K}$ know as the $\emph{naturality square}$:
 \begin{tikzcd}
 F(X) \rar[][color=red]{\eta_{X}} \dar[]{F(\varphi)}
  & G(X) \dar[][color=red]{G(\varphi)} \\
 F(Y) \rar{\eta_{Y}}
  & G(Y)
 \end{tikzcd}
\end{defn}
\begin{rem}
 $\textbf{N.B.}$ The naturality square is the key concept to understanding natural transformations and
 there is one of them for every morphism in $\mathcal{C}$ as stated above. Note this well.
\end{rem}

Constructions are often "naturally related" - a vague notion, at first sight.
This leads to the clarifying concept of natural transformation, a way to map one functor to another.
\begin{rem}
 Various important constructions in mathematics can be studied in this context.
 Naturality is a principle, like that of general covariance in physics, that cuts deeper than is initially apparent.
\end{rem}