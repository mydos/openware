\section{Morphisms} % (fold)
\label{sec:morphisms}
A morphism is a map between two objects in an abstract category from $\emph{dom}$ain to $\emph{codom}$ain, more on this later.
A general morphism is called a homomorphism. A homomorphism is a term typically used in category theory.
The term itself derives from the Greek $o \mu o$ (omo), ``alike'' and $\mu o \rho\varphi\omega\sigma\iota\varsigma$
(morphosis), ``to form'' or ``to shape''.
\begin{rem}
 The similarity in meaning and form of the words ``homomorphism'' and ``homeomorphism'' is unfortunate and typically
 a source of common confusion.
\end{rem}

Morphisms can have any of the following properties.

\subsection{Monomorphism} % (fold)
\label{subsec:monomorphism}
\begin{defn}[Monomorphism]
 A morphism $\varphi: X \to Y$ is a $\emph{monomorphism}$ if $\varphi \circ \psi = \varphi \circ \phi \implies \psi = \phi$ for all morphisms $\psi,\phi: Z \to X$.
 \\
 \begin{tikzcd}
  Z \rar[transform canvas={yshift=.5ex}]{\psi} \rar[transform canvas={yshift=-.5ex}, swap]{\phi} & X \rar{\varphi} & Y
 \end{tikzcd}
 \end{defn}

\subsection{Epimorphism} % (fold)
\label{subsec:epimorphism}
\begin{defn}[Epimorphism]
 A morphism $\varphi: X \to Y$ is a $\emph{epimorphism}$ if $\psi \circ \varphi = \phi \circ \varphi \implies \psi = \phi$ for all morphisms $\psi,\phi: Y \to Z$.
 \\
 \begin{tikzcd}
  X \rar{\varphi} & Y \rar[transform canvas={yshift=.5ex}]{\psi} \rar[transform canvas={yshift=-.5ex}, swap]{\phi} & Z
 \end{tikzcd}
\end{defn}
 
\subsection{Bimorphism} % (fold)
\label{subsec:bimorphism}
\begin{defn}[Bimorphism]
 A morphism $\varphi: X \to Y$ is a $\emph{bimorphism}$ if $\varphi$ is both a monomorphism and epimorphism.
\end{defn}

\subsection{Isomorphism} % (fold)
\label{subsec:isomorphism}
\begin{defn}[Isomorphism]
 A morphism $\varphi: X \to Y$ is a $\emph{isomorphism}$ if there exists a morphism $\psi: Y \to X : \varphi \circ \psi = \Id{Y}$ and $\psi \circ \varphi = \Id{X}$.
\end{defn}

\subsection{Endomorphism} % (fold)
\label{subsec:endomorphism}
\begin{defn}[Endomorphism]
 A morphism $\varphi: X \to X$ is a $\emph{endomorphism}$.
 The class of endomorphisms of $\varphi$ is denoted by $end(\varphi)$.
\end{defn}

\subsection{Automorphism} % (fold)
\label{subsec:automorphism}
\begin{defn}[Automorphism]
 A morphism $\varphi: X \to Y$ is a $\emph{automorphism}$ if $\varphi$ is both an endomorphism and an isomorphism.
 The class of automorphisms of $\varphi$ is denoted by $aut(\varphi)$.
\end{defn}

\subsection{Retractions and Sections} % (fold)
\label{subsec:retractionsandsections}
\begin{defn}[Retraction]
 A morphism $\varphi: X \to Y$ has a $\emph{retraction}$ if there exists a morphism $\psi: Y \to X$ with $\varphi \circ \psi = \Id{Y}$ - ``if a right inverse of $\varphi$ exists''.
\end{defn}

\begin{defn}[Section]
 A morphism $\varphi: X \to Y$ has a $\emph{section}$ if there exists a morphism $\psi: Y \to X$ with $\psi \circ \varphi = \Id{X}$ - ``if a left inverse of $\varphi$ exists''.
\end{defn}

\begin{lem}[Retractions and Sections]
Every retraction is an epimorphism and every section is a monomorphism.
We have the following three equivalent statements:
\begin{itemize}
\item $\varphi$ is a monomorphism and a retraction;
\item $\varphi$ is an epimorphism and a section;
\item $\varphi$ is an isomorphism.
\end{itemize}
\end{lem}