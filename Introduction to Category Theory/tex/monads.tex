\section{Monads} % (fold)
\label{sec:monads}
$\emph{Monads}$ derived from Greek, $\mu o \nu\alpha\varsigma$ (monas), ``unit''.
The computer science interpretation is that of a unit of computation.
Monads are like algebraic theories and that an algebra for a monad is a model of that theory. A monad is simply
a functor with some extra 'stuff' that tells us what is going on in the associated theory.
The notion of $\emph{algebras for a monad}$ generalises classical notions from universal algebra,
and in this sense, monads can be thought of as $\emph{theories}$.
More formally, a monad or $\emph{(Kleisli triple)}$, is an endofunctor coupled with two natural transformations,
$\eta$ and $\mu$ that satisfy so-called $\emph{coherence conditions}$.

\begin{defn}[Monad]
 Given a category $\mathcal{K}$, a $\emph{monad}$ on $\mathcal{K}$ is the endofunctor $T: \mathcal{K} \to \mathcal{K}$
 coupled with two natural transformations, $\eta: \Id{\mathcal{C}} \rightrightarrows T$ and $\mu: T^2 \rightrightarrows T$
 that satisfy the coherence conditions:
 \begin{itemize}
  \item $\mu \circ T \mu = \mu \circ \mu T$, and
  \item $\mu \circ T \eta = \mu \circ \eta T = \Id{T}$.
 \end{itemize}
\end{defn}
\begin{rem}
 N.B. $\eta$ called $\emph{the unit}$ since it is a natural mapping from idenity and
 $\mu$ called $\emph{the multiplication}$ also noting the natural mapping from $T^2 = T \circ T$.
\end{rem}

Recall the definition of a natural transformation of $\eta$ and $\mu$, that for every $X \in \text{Ob}(\mathcal{K})$
there is a $\emph{component}$ morphism of $\eta$ and $\mu$ at $X$, denoted by, $\eta_{X}$ and $\mu_{X}$ respectively.
\\
So, we have the following:
\begin{itemize}
 \item $\eta_{X}: X \to T(X) \in \mathcal{K}$,
 read, $\emph{``the component of $\eta$ at $X$ is a morphism in $\mathcal{K}$ from $X$ to $T$ of $X$''}$, and,
 \item $\mu_{X}: T^2(X) \to T(X) \in \mathcal{K}$,
 read, $\emph{``the component of $\mu$ at $X$ is a morphism in $\mathcal{K}$ from $T^2$ of $X$ to $T$ of $X$''}$.
\end{itemize}

Hence, the coherence conditions are shown the by commutative diagrams for every component $X \in \text{Ob}(\mathcal{K})$
\\
For the existence idenity by $\emph{the left and right unit triangles}$:
\begin{tikzcd}
T(X) \drar[equals] \rar[][color=red]{\eta_{T(X)}} \dar[swap]{T(\eta_{X})}
 & T^2(X) \dar[][color=red]{\mu_{X}} \\
T^2(X) \rar[swap]{\mu_{X}}
 & T(X)
\end{tikzcd}
\\
and for that of $\emph{the associativity square}$ by:
\begin{tikzcd}
T^3(X) \rar[][color=red]{T(\mu_{X})} \dar[swap]{\mu_{T(X)}}
 & T^2(X) \dar[][color=red]{\mu_{X}} \\
T^2(X) \rar[swap]{\mu_{X}}
 & T(X)
\end{tikzcd}
Since these are natural transformations,
we can write these without $X$ and by writing $\eta_{T(X)}$ as $\eta T$ and $T(\eta_{X})$ as $T \eta$, for example.
Since the commutative diagrams in the category $\mathcal{K}$ are component wise, that is, in terms of $X$.

\begin{rem}[Comonad]
 A $\emph{Co}$monad, or just $\emph{cotriple}$, is the categorical dual of a Monad.
 That is, A comonad for a category $\mathcal{K}$ is a monad for the opposite category $\mathcal{K}^{op}$.
\end{rem}


\begin{exmp}[Monad for monoids]
 From the definition of a monoid,
 take the category $\mathcal{K} = \textbf{Set}$ and a functor $T: X \to X^{*}$ where $X$ is the ``set of words'' in $X$.
 \begin{rem}
  A word is just a list of objects in $X$ e.g., $(x_{1}, x_{2}, x_{3})$ or simply the empty list $()$.
 \end{rem}
 Therefore, we define the natural transformations component wise, as: $\eta_{X}: X \to X^{*}$ and $\mu_{X}: X^{**} \to X^{*}$.
 \begin{rem}
  To understand $\mu_{X}: X^{**} \to X^{*}$ consider the example: $(a, (b, c)) \to (a, b, c)$
 \end{rem}
 Finish me...... TODO.
\end{exmp}

\begin{note}
Haskell Note:
\begin{itemize}
 \item $\eta$ is the Monadic equivalent of the "return" function and,
 \item $\mu$ is the Monadic equivalent of the "join" operator.
\end{itemize}
\end{note}


\subsection{Algebras for Monads} % (fold)
\label{sec:monadalgebra}
Given the monad $(T, \eta, \mu)$ on a category $\mathcal{K}$. A $T$-algebra denoted by $(x,h)$,
where $x \in \text{Ob}(\mathcal{K})$ coupled with $h \in \text{Hom}(\mathcal{K})$ with
$h: T(x) \to x \in \mathcal{K}$. The morphism $h$ is called the structure map of the algebra $T$ provided
the usual diagrams commute (i.e., the coherence conditions are satisfied).

The $\emph{the associativity square}$ by:
\begin{tikzcd}
T^2(x) \rar[][color=red]{T(x)} \dar[swap]{\mu_{x}}
 & T(x) \dar[][color=red]{h} \\
T(x) \rar[swap]{h}
 & x
\end{tikzcd}
\\
and the $\emph{right unit triangle}$ (idenity) by:
\begin{tikzcd}
 x \drar[swap]{\Id{x}} \rar{\eta_{x}} & T(x)
 \dar[]{h} \\ & x
\end{tikzcd}

A morphism $f: (x,h) \to (x',h')$ of $T$-$\emph{algebras}$ is a morphism $f: x \to x' \in \mathcal{K}$
given that the following diagram commutes:
\begin{tikzcd}
T(x) \rar[][color=red]{T(f)} \dar[swap]{h}
 & T(x') \dar[][color=red]{h'} \\
x \rar[swap]{f}
 & x'
\end{tikzcd}