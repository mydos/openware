\section{Categories} % (fold)
\label{sec:categories}
A category consists of three things:
a collection of objects, for each pair of objects a collection of morphisms from one to another, and a binary
operation defined on compatible pairs of morphisms called composition denoted by $\circ$. The category must satisfy an identity axiom
and an associative axiom which is analogous to the monoid axioms, seen later. We require that morphisms preserve the
mathematical structure of the objects (i.e., for vector spaces as objects one would choose linear maps as morphisms, and so on).
\begin{defn}[Category]
	A $\emph{category}$ $\Cat$ consists of the following three mathematical entities:
	\begin{enumerate}
		\item A $\emph{class}$ $\text{Ob}(\Cat)$ of objects
		
		\item A class $\text{Hom}(A,B)$ of $\emph{morphisms}$, from $A \longrightarrow B$ such that $A, B \in \text{Ob}(\Cat)$.
		\\
		e.g. $\varphi : A \to B$ to mean $\varphi \in \text{Hom}(A,B)$.
		\begin{rem}
		The class of \emph{all} morphisms of $\Cat$ is denoted $\text{Hom}(\Cat)$.
		\end{rem}

		\item Given $A, B, C \in \text{Ob}(\Cat)$, a binary operation $\circ : Hom(A,B) \times Hom(B,C) \to Hom(A,C)$ called
		\emph{composition}, satisfying:
		\begin{enumerate}
			\item \emph{(closed associativity)} Given $\varphi : A \to B$, $\psi: B \to C$ and $\phi: A \to C$
			\\
			we have $\phi \circ (\psi \circ \varphi) = (\phi \circ \psi) \circ \varphi$.
			\\
			\begin{tikzcd}
			A \drar{\phi} \rar{\varphi}
			& B \dar{\psi} \\
			& C
			\end{tikzcd}
			
			\item \emph{(identity)} For any object X there is an identity morphism $\Id{X} : X \to X$ such that for any $\varphi: A \to B$ we have $\Id{B} \circ \varphi = \varphi = \varphi \circ \Id{A}$.
			\\
			\begin{tikzcd}
			X \arrow[in=30, out=60, loop]{}[name=idX]{\Id{X}}
			\end{tikzcd}
		\end{enumerate}
	\end{enumerate}
\end{defn}

It is also worth noting about what we mean by `small' and `large' categories.

\begin{defn}[Small Category]
	A category $\Cat$ is called small if both $\text{Ob}(\Cat)$ and $\text{Hom}(\Cat)$ are sets.
	If $\Cat$ is not small, then it is called large.
	$\Cat$ is called locally small if $\text{Hom}(A,B)$ is a set for all $A, B \in \text{Ob}(\Cat)$.
\end{defn}

\begin{rem}
	Most important categories in mathematics are not small however, are locally small.
\end{rem}

Most basic categories have as objects certain mathematical structures, and the structure-preserving
functions as morphisms. A common list is given;
\begin{itemize}
\item $\emph{Pos}$ is the category of partially ordered sets and monotonic functions.
\item $\emph{Top}$ is the category of topological spaces and continuous functions.
\item $\emph{Rng}$ is the category of rings and ring homomorphisms.
\item $\emph{Grp}$ is the category of groups and group homomorphisms.
\item $\emph{Grph}$ is the category of graphs and graph homomorphisms.
\end{itemize}

\subsection{Product} % (fold)
\label{subsec:product}
Let $A,B \in \text{Ob}(\mathcal{K})$ for some category $\mathcal{K}$.
A $\emph{product}$ of $A$ and $B$ is an object $C$, together with morphisms from
$\alpha: C \to A$ and $\beta: C \to B$, such that the following property holds:
if $C^{'}$ is any object, and any morphisms $\alpha^{'}: C^{'} \to A$ and $\beta^{'}: C^{'} \to B$,
there is a unique morphism $\gamma: C^{'} \to C$ such that the following diagram commutes:

\begin{tikzcd}
& C^{'}
 \ar[swap]{ddl}{\alpha^{'}} \ar{ddr}{\beta^{'}} \dar{\gamma} & \\
 & C
 \dlar{\alpha} \drar[swap]{\beta} & \\
 A & & B
\end{tikzcd}

The morphisms $\alpha$ and $\beta$ are called the $\emph{canonical projections}$.
We denote the product of $A$ and $B$ as: $C = A \prod B$ or sometimes written $C = A \bigotimes B$.