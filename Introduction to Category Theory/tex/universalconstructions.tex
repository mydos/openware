\section{Universal Constructions} % (fold)
\label{sec:universalconstructions}
Categorial objects are both abstract and have atomic type.
Thus we are face with the issue of defining objects without referring to the internal structure.
To characterise these objects, we look at them from the prospective of their relations to other objects, as given by the morphisms 
of the respective categories. Therefore we must find universal properties that uniquely determine the objects of interest.
Indeed, it turns out that numerous important constructions can be described purely in this way.
The central concept which is needed for this purpose is called categorical limit, and can be dualized to yield the notion of a colimit.

\subsection{Diagram} % (fold)
\label{subsec:diagram}
A diagram is the categorial analogue of an indexed family in set theory. In particular, one also must consider that
morphism as well as objects must be indexed. Hence, a diagram is a collection of objects and morphisms, that is,
a category that is indexed by another fixed category. Since we are mapping from one category to another, a diagram is
naturally then a functor as you may expect.
\begin{defn}[Diagram]
 A $\emph{diagram}$ $D$ is given by the functor $D: \mathcal{J} \to \mathcal{K}$
 where $\mathcal{J}$ is some $\emph{index category}$ in the category $\mathcal{K}$.
\end{defn}

The actual detail of the internals of the index category $\mathcal{J}$ is of no concern, merely the internal relationships
are of importance. Hence, diagrams and functors are technically the same thing, however, a different matter of prospective.

\begin{cor}
 A diagram is said to be $\emph{small}$, or finite, whenever the index category $\mathcal{J}$ is.
\end{cor}

Again, recall that a digram is a essentially a functor.
Now, suppose two diagrams $D_{1}$ and $D_{2}$ by some fixed indexing category $\mathcal{J}$ in the category $\mathcal{K}$.
Then each morphism in $D_{1}$ can be found in $D_{2}$ by the natural transformation from $D_{1}$ to $D_{2}$.
Hence we have motivation to form a definition of the $\emph{category of diagrams}$, or $\emph{functor category}$ when
$\mathcal{J}$ is small, given by the following:

\begin{defn}[Category of Diagrams]
 The category of diagrams by the index category, or type, $\mathcal{J}$ in the category $\mathcal{K}$ as the $\emph{functor category}$,
 denoted by $\mathcal{K}^{\mathcal{J}}$ with diagrams as the defined object class.
\end{defn}

\begin{exmp}
 Take some small type $\mathcal{J}$ in the category $\mathcal{K}$,
 then $\mathcal{J}$ merely indexes the objects in the category $\mathcal{K}$.
 \begin{rem}
  Also note that the diagram is again be finite since $\mathcal{J}$ is.
 \end{rem}
\end{exmp}

One of the wonderful results of diagrams in category theory is that we have a new simple, yet effective, proving technique.
This technique of proving with categorial diagrams is know as $\emph{diagram chasing}$ where we look for equivalent map compositions in commutative diagrams.
Typically exploiting the properties of monomorphic, epimorphic and bimorphic type homomorphisms and their $\emph{exact sequences}$.
\begin{note}
 As a side note;
 Another area of mathematics where diagram chasing is oftern used is that of homological algebra, where Modules are of primary concern.
\end{note}

\subsection{Cone} % (fold)
\label{subsec:cone}
The abstract notion of a $\emph{cone of a functor}$ shall be the essence of what it is to define $\emph{the limit of a functor}$.

\begin{defn}
 Given some diagram $F: \mathcal{J} \to \mathcal{K}$ with type $\mathcal{J}$ in the category $\mathcal{K}$.
 A $\emph{cone}$ to the diagram $F$ is an object $N$ of the category $\mathcal{K}$ endowed with a family $\Psi_{X}: N \to F(X)$ of morphisms
 indexed by $\text{Ob}(\mathcal{J})$ such that every morphism $\varphi: X \to Y \in \mathcal{J}$, we have $F(\varphi) \circ \Psi_{X} = \Psi_{Y}$.
 The cone is denoted by the pair $(N, \Psi)$.
\end{defn}

Hence, we have the following commutative diagram:
\begin{tikzcd}
 N \dar[bend right, swap]{\Psi_{X}} \drar[bend left]{\Psi_{Y}} \\
 F(X) \rar[swap]{F(\varphi)} & F(Y)
\end{tikzcd}

\begin{rem}
 The cone $\Psi$ can be said to have vertex $N$ with base $F$.
\end{rem}

\begin{rem}[Cocone]
 The categorial dual of a cone is called a $\emph{co}$cone.
\end{rem}

\subsection{Limit} % (fold)
\label{subsec:limit}
Let $F: J \to \mathcal{J}$ be a diagram of type $J$ in the category $\mathcal{K}$.
A cone to $F$ is a some $N \in \text{Ob}(\mathcal{K})$ together with a family
$\Psi: N \to F(X)$ of morphisms indexed by the objects of $J$, such that
for each $\varphi: X \to Y \in J$, we have $F(\varphi) \circ \Psi_{X} = \Psi_{Y}$.

A $\emph{limit}$ of the diagram $F$ is a cone $(L,\psi)$ to $F$ such that for any other
cone $(N,\Psi)$ to $F$ there exists a $\emph{unique}$ morphism $\gamma: N \to L$ such that
$\psi_{X} \circ \gamma = \Psi_{X} \, \forall X \in J$.

\begin{tikzcd}
& N
 \ar[swap]{ddl}{\Psi_{X}} \ar{ddr}{\Psi_{Y}} \dar{\gamma} & \\
 & L
 \dlar{\psi_{X}} \drar[swap]{\psi_{Y}} & \\
 F(X) & \rar[swap]{F(\varphi)} & F(Y)
\end{tikzcd} % FIXME: F(f) arrow too short?

This definition should look familiar to that of a $\emph{product}$.
Here, the cone $(N, \Psi)$ factors through the cone $(L, \psi)$ with the unique factorization $\gamma$.
We call the morphism, $\gamma$, the mediating morphism.

\begin{rem}[Colimit]
 The categorial dual of a limit is called a $\emph{co}$limit.
\end{rem}

% \begin{exmp}
%  Suppose that $U: \mathcal{K} \to \mathcal{C}$ is a functor from a category $\mathcal{K}$ to a category $\mathcal{C}$,
% and let $X \in \text{Ob}(\mathcal{C})$. Consider the following dual notions:
% \begin{defn}[Initial morphism]
%  An $\emph{initial morphism}$ from $X \to U$ is an initial object in the category
%  $(X \downarrow U)$ of morphisms from $X \to U$. In particular, it consists
%  of a pair $(A, \varphi)$ where $A \in \text{Ob}(\mathcal{K})$ and $\varphi: X \to U(A) \in \text{Hom}(\mathcal{C})$,
%  such that the following initial property is satisfied:
%  \\
%  Whenever $Y \in \text{Ob}(\mathcal{K})$ and $f: X \to U(Y) \in \text{Hom}(\mathcal{C})$, then there exists a unique morphism
%  $g: A \to Y$ such that the following diagram commutes:
% \begin{tikzcd}
%  X \drar[swap]{f} \rar{\varphi} & U(A)
%  \dar[dashed]{U(g)} \\ & U(Y)
% \end{tikzcd}
% \end{defn}
% 
% \begin{defn}[Terminal morphism]
%  A $\emph{terminal morphism}$ from $U \to X$ is a terminal object in the $\emph{comma category (i.e., morphisms become objects)}$
%  $(U \downarrow X)$ of morphisms from $U \to X$. In particular, it consists of a pair $(A, \varphi)$
%  where $A \in \text{Ob}(\mathcal{K})$ and $\varphi: U(A) \to X \in \text{Hom}(\mathcal{C})$,
%  such that the following terminal property is satisfied:
%  \\
%  Whenever $Y \in \text{Ob}(\mathcal{K})$ and $f: U(Y) \to X \in \text{Hom}(\mathcal{C})$, then there exists a unique morphism
%  $g: Y \to A$ such that the following diagram commutes:
% \begin{tikzcd}
%  U(Y) \dar[dashed,swap]{U(g)} \drar{f} \\
%  U(A) \rar[swap]{\varphi} & X
% \end{tikzcd}
% \end{defn}
% \end{exmp}