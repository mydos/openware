% Copyright © 2012 Edward O'Callaghan. All Rights Reserved.

%\RequirePackage[l2tabu, orthodox]{nag}

\documentclass[10pt, oneside, reqno]{amsart}
\usepackage{geometry, setspace, graphicx, enumerate, amssymb}
%\usepackage{dcpic,pictex}
\usepackage{pgf}
\usepackage{tikz}
\usepackage{tikz-cd}
%\usepackage{bbm} % /mathbbm{1}
\usepackage{microtype}
\onehalfspacing                 

%\usepackage[ruled,section]{algorithm}
%\usepackage{algpseudocode}


% AMS Theorems
\theoremstyle{plain}% default 
\newtheorem{thm}{Theorem}[section] 
\newtheorem{prob}[thm]{Problem}
\newtheorem{question}[thm]{Question}

\newtheorem{lem}[thm]{Lemma}
\newtheorem{prop}[thm]{Proposition}
\newtheorem*{cor}{Corollary}

\theoremstyle{definition}
\newtheorem{defn}[thm]{Definition}
\newtheorem{conj}[thm]{Conjecture}
\newtheorem{exmp}[thm]{Example}

\theoremstyle{remark} 
\newtheorem*{rem}{Remark} 
\newtheorem*{note}{Note} 
\newtheorem{case}{Case} 

\newcommand{\expc}[1]{\mathbb{E}\left[#1\right]}

\newcommand{\Q}{\mathbb{Q}}
\newcommand{\R}{\mathbb{R}}
\newcommand{\C}{\mathbb{C}}
\newcommand{\Z}{\mathbb{Z}}
\newcommand{\F}{\mathbb{F}}
\newcommand{\Ga}{\mathbb{G}}

\newcommand{\Cat}{\mathcal{K}}
\newcommand{\id}{\mbox{id}_}

\newcommand{\nth}{n\textsuperscript{th}}
\newcommand{\bigo}[1]{\mathcal{O}(#1)}

% \setlength{\topmargin}{0.2cm}
% \setlength{\footskip}{0.2cm}
% \setlength{\hoffset}{-1cm}
% \setlength{\voffset}{-2cm}

	
\usepackage{hyperref}
	
\title{Category Theory}                               % Document Title
\author{Edward O'Callaghan}
%\date{}                                           % Activate to display a given date or no date



\begin{document}
\maketitle \tableofcontents \clearpage

\section{Prelude} % (fold)
\label{sec:purelude}
First a word on how to study category theory.
Category theory is extremely heavy on new terminology at first sight.
It is typically misconceived as abstractly and arbitrarily renaming things to no real gain and a generally poorly understood
topic in ungrad mathematics. To get over the initial shock of the many perhaps new terms and to consolidate simple concepts
and hence sometimes hard to remember due to their intrinsic simplicity. One should consider taking the time to highlight these
notes in assortment of coloured pens, with each new term given a corresponding colour. Also note that almost everything, at least
at this level in category theory may be represented as a diagram. I highly recommend you copy each new term with a corresponding
diagram to anecdote the idea at hand. If all else fails, be sure to remember to break the term down into prefix and suffix and reason
it. Typically terms are fairly simple to reason out (e.g., mono = ``one'', auto = ``self'', morphism = ``change shape'').

Here we just fix some notation and typically assumed terminology.
\begin{note}[Class]
 A $\emph{class}$ is a collection of mathematical objects (e.g., sets)
 which can be unambiguously defined by a property that all its members share.
\end{note}

\begin{note}[Morphism]
 A $\emph{morphism}$, from Greek, is an abstraction derived from
 structure-preserving mappings between two mathematical structures.
 The notion of morphism recurs in much of contemporary mathematics.
\end{note}


\section{Introduction} % (fold)
\label{sec:introduction}
In each area of mathematics (e.g., sets, groups, topological spaces) there are available many definitions
and constructions. It turns out, however, that there are a number of notions (e.g. that of a product)
that occur naturally in various areas of mathematics, with only slight changes from one area to another.
It is convenient to take advantage of this observation. Category theory can be described as that branch of mathematics
in which one studies certain definitions in a broader context - without reference to the particular area to which
the definition might be applied.


\section{Categories} % (fold)
\label{sec:categories}
We begin by defining what we mean by a category:
\begin{defn}[Category]
	A $\emph{category}$ $\Cat$ consists of the following three mathematical entities:
	\begin{enumerate}
		\item A $\emph{class}$ $\text{Ob}(\Cat)$ of objects
		
		\item A class $\text{Hom}(A,B)$ of $\emph{morphisms}$, from $A \longrightarrow B$ such that $A, B \in \text{Ob}(\Cat)$.
		\\
		e.g. $f : A \to B$ to mean $f \in \text{Hom}(A,B)$.
		\begin{rem}
		The class of \emph{all} morphisms of $\Cat$ is denoted $\text{Hom}(\Cat)$.
		\end{rem}

		\item Given $A, B, C \in \text{Ob}(\Cat)$, a binary operation $\circ : Hom(B,C) \times Hom(A,B) \to Hom(A,C)$ called \emph{composition},
		satisfying:
		\begin{enumerate}
			\item \emph{(closed associativity)} Given $f : A \to B$, $g : B \to C$ and $h : C \to D$
			\\
			we have $h \circ (g \circ f) = (h \circ g) \circ f$.
			\\
			\begin{tikzcd}[row sep=tiny]
			& B \arrow{dd}{g} \\
			A \arrow{ur}{f} \arrow{dr}{h} & \\
			& C
			\end{tikzcd}
			
			\item \emph{(identity)} For any object X there is an identity morphism $\id{X} : X \to X$ such that for any $f : A \to B$ we have $\id{B} \circ f = f = f \circ \id{A}$.
			\\
			\begin{tikzcd}
			X \arrow[in=30, out=60, loop]{}[name=idX]{\id{X}}
			\end{tikzcd}
		\end{enumerate}
	\end{enumerate}
\end{defn}

It is also worth noting about what we mean by `small' and `large' categories.

\begin{defn}[Small Category]
	A category $\Cat$ is called small if both $\text{Ob}(\Cat)$ and $\text{Hom}(\Cat)$ are sets.
	If $\Cat$ is not small, then it is called large.
	$\Cat$ is called locally small if $\text{Hom}(A,B)$ is a set for all $A, B \in \text{Ob}(\Cat)$.
\end{defn}

\begin{rem}
	Most important categories in mathematics are not small however, are locally small.
\end{rem}

Most basic categories have as objects certain mathematical structures, and the structure-preserving
functions as morphisms. A common list is given;
\begin{itemize}
\item $\emph{Top}$ is the category of topological spaces and continuous functions.
\item $\emph{Grp}$ is the category of groups and group homomorphisms.
\item $\emph{Rng}$ is the category of rings and ring homomorphisms.
\item $\emph{Grph}$ is the category of graphs and graph homomorphisms.
\item $\emph{Pos}$ is the category of partially ordered sets and monotonic functions.
\end{itemize}

\begin{prop}[Properties of morphisms]
Morphisms can have any of the following properties. A morphism $f: X \to Y$ is a:
\begin{itemize}
\item $\emph{monomorphism}$ if $f \circ g = f \circ h \implies g = h$ for all morphisms $g,h: X \to Y$.
\item $\emph{epimorphism}$ if $g \circ f = h \circ f \implies g = h$ for all morphisms $g,h: Y \to X$.
\item $\emph{bimorphism}$ if $f$ is both a monomorphism and epimorphism.
\item $\emph{isomorphism}$ if there exists a morphism $g: Y \to X : f \circ g = \id{Y}$
 and $g \circ f = \id{X}$.
\item $\emph{endomorphism}$ if $X = Y$.
The class of endomorphisms of a is denoted by $end(a)$.
\item $\emph{automorphism}$ if $f$ is both an endomorphism and an isomorphism.
The class of automorphisms of a is denoted by $aut(a)$.
\item $\emph{retraction}$ if a right inverse of $f$ exists,
 i.e. if there exists a morphism $g: Y \to X$ with $f \circ g = \id{Y}$.
\item $\emph{section}$ if a left inverse of f exists,
 i.e. if there exists a morphism $g: Y \to X$ with $g \circ f = \id{X}$.
\end{itemize}
\end{prop}

\begin{rem}[Retractions and Sections]
Every retraction is an epimorphism and every section is a monomorphism.
We have the following three equivalent statements:
\begin{itemize}
\item $f$ is a monomorphism and a retraction;
\item $f$ is an epimorphism and a section;
\item $f$ is an isomorphism.
\end{itemize}
\end{rem}


\section{Functors} % (fold)
\label{sec:functors}
A category is itself a type of mathematical structure, so we can look for "processes" which preserve this structure in some sense; such a process is called a functor.
A functor associates to every object of one category an object of another category, and to every morphism in the first category a morphism in the second.
Hence, Functors are structure-preserving maps between categories and can be thought of as morphisms in the category of all (small) categories.

In particular, what we have done is define a category of categories and functors – the objects are categories, and the morphisms (between categories) are functors.
By studying categories and functors, we are not just studying a class of mathematical structures and the morphisms between them;
we are studying the relationships between various classes of mathematical structures.

\begin{defn}[Functor]
	Let $\mathcal{C}$ and $\mathcal{K}$ be categories. A $\emph{functor}$ $F$ from $\mathcal{C}$ to $\mathcal{K}$ is a mapping that:
	\begin{enumerate}
		\item associates to each object $X \in \mathcal{C}$ an object $F(X) \in  \mathcal{K}$
		
		\item associates to each morphism $f : X \to Y \in \mathcal{C}$ a morphism $F(f) : F(X) \to F(Y) \in \mathcal{K}$
		satisfying:
		\begin{enumerate}
			\item $F(id_{X}) = id_{F(X)}$ for every object $X \in \mathcal{C}$
			
			\item $F(g \circ f) = F(g) \circ F(f)$ for all morphisms $f : X \to Y$ and $g : Y \to Z$
			\begin{rem}
				That is, functors must preserve identity morphisms and composition of morphisms.
			\end{rem}
		\end{enumerate}
	\end{enumerate}
\end{defn}

\subsection{Types of Functors} % (fold)
\label{subsec:functorstypes}
Like many things in category theory, functors come in a kind of ``dual'' type in concepts;
that of the $\emph{contravariant}$- and $\emph{covariant}$- functors, defined as follows:

\begin{defn}[Covariant Functor]
 A $\emph{covariant}$ functor $F$ from a category $\mathcal{C}$ to a category $\mathcal{K}$, $F: \mathcal{C} \to \mathcal{K}$, consists of:
 \begin{itemize}
  \item for each object $X \in \mathcal{C}$, an object $F(X) \in \mathcal{K}$
  \item for each morphism $f: X \to Y \in \mathcal{C}$, a morphism $F(f): F(X) \to F(Y)$
 \end{itemize}
 provided the following two properties hold:
 \begin{itemize}
  \item For every object $X \in \mathcal{C}$, $F(\id{X}) = \id{F(X)}$
  \item For all morphisms $f: X \to Y$ and $g: Y \to Z$, $F(g \circ f) = F(g) \circ F(f)$
 \end{itemize}
\end{defn}

\begin{defn}[Contravariant Functor]
 A $\emph{contravariant}$ functor $F: \mathcal{C} \to \mathcal{K}$, is a `reversed' covariant functor.
 In particular, for every morphism $f: X \to Y \in \mathcal{C}$ must be assigned to a morphism $F(f) : F(Y) \to F(X) \in \mathcal{D}$.
 Alternatively, contravariant functors acts as covariant functor from the opposite category $\mathcal{C}^{op} \to \mathcal{K}$.
\end{defn}

In particular, functors have particular properties, essentially the same as morphisms do.
This may seem obvious, since functors are a generalisation of the notion of a morphism.
The following definitions should mostly be obvious if you reflect back to the corresponding morphism
definitions above or just break the word down grammatically.

\begin{defn}[Endofunctor]
 A $\emph{endofunctor}$ is a functor that maps a category to its self, sometimes called a $\emph{identityfunctor}$.
\end{defn}

\begin{defn}[Bifunctor]
 A $\emph{bifunctor}$ is a functor in $\emph{two}$ arguments. More formally, a bifunctor is a functor whose
 domain is a $\emph{product category}$.
 The $Hom$ functor is a natural example; it is contravariant in one argument while covariant in the other.
 Hence, the $Hom$ functor is of the type $\mathcal{C}^{op} \times \mathcal{C} \to \textbf{Set}$.
\end{defn}

\begin{defn}[Multifunctor]
 A $\emph{multifunctor}$ is a generalisation of the functor concept to n variables, (e.g., A $\emph{bifunctor}$ is when $n=2$)
\end{defn}

\subsection{Properties of Functors} % (fold)
\label{subsec:functorsproperties}
However, some properties are more particular to functors and we review them here to families ourselves.
A functor $F : \mathcal{A} \to \mathcal{B}$ is
\begin{enumerate}
 \item $\emph{faithful}$ if for every parallel pair of morphisms $f,g : A \rightrightarrows A' \in \mathcal{A}$, one has $f = g$ whenever $F(f) = F(g)$
 \item $\emph{full}$ if for every morphism $b: F(A) \to F(A') \in \mathcal{B}$, there exists a morphism $a: A \to A' \in \mathcal{A}$ such that $F(a) = b$
 \item essentially $\emph{surjective}$ if for every object $B \in \mathcal{B}$, there exists an object $A \in \mathcal{A}$ with $B$ isomorphic to $F(A)$
 \item an $\emph{equivalence}$ if there exists a functor $F': \mathcal{B} \to \mathcal{A}$ such that both $F \circ F'$ and $F' \circ F$ are naturally
 isomorphic to the identity functors; such a functor $F'$ is called a $\emph{quasi-inverse}$ of $F$
 \item an $\emph{isomorhpism}$ if there exists a functor $F': \mathcal{B} \to \mathcal{A}$ such that both $F \circ F'$ and $F' \circ F$
 are equal to the identity functors
 \item $\emph{conserative}$ if it reflects isomorphisms; that is, $a: A \to A'$ is an isomorphism whenever $F(a): F(A) \to F(A')$ is
\end{enumerate}


\section{Monoids} % (fold)
\label{sec:monoids}

\begin{defn}[Monoid]
A category with exactly one object is called a $\emph{monoid}$.
\end{defn}

Thus, expanding out the definition of a category we have;
\begin{cor}[Monoid]
A $\emph{monoid}$ is a set $X$ together with a law of composition $\circ$
which associates and has identity $e \in X$.
\end{cor}


\section{Natural Transformations} % (fold)
\label{sec:natural transformations}
In category theory, a branch of mathematics, a natural transformation provides a way of transforming one functor into another
while respecting the internal structure (i.e. the composition of morphisms) of the categories involved.
Hence, a natural transformation can be considered to be a ``morphism of functors''.
Indeed this intuition can be formalized to define so-called functor categories.
Natural transformations are, after categories and functors, one of the most basic notions of category theory and consequently
appear in the majority of its applications.

\begin{defn}[Natural Transformation]
	Let $F$ and $G$ be functors between the categories $\mathcal{C}$ and $\mathcal{K}$, then a $\emph{natural transformation}$ $\eta : F \to G$
	associates to every object $X \in \mathcal{C}$ a morphism $\eta_{X} : F(X) \to G(X)$ between objects of $\mathcal{K}$, called the $\emph{component}$
	of $\eta$ at $X$, such that for every morphism $f : X \to Y \in \mathcal{C}$ we have:
	\\
	$\eta_{Y} \circ F(f) = G(g) \circ \eta_{X}$
\end{defn}
Constructions are often "naturally related" - a vague notion, at first sight.
This leads to the clarifying concept of natural transformation, a way to "map" one functor to another.
Many important constructions in mathematics can be studied in this context.
"Naturality" is a principle, like general covariance in physics, that cuts deeper than is initially apparent.


\end{document}% End of document.
