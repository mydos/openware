% Copyright © 2012 Edward O'Callaghan. All Rights Reserved.

%\RequirePackage[l2tabu, orthodox]{nag}

\documentclass[10pt, oneside, reqno]{amsart}
\usepackage{geometry, setspace, graphicx, enumerate, amssymb}
%\usepackage{dcpic,pictex}
\usepackage{pgf}
\usepackage{tikz}
\usepackage{tikz-cd}
%\usepackage{bbm} % /mathbbm{1}
\usepackage{microtype}
\onehalfspacing                 

%\usepackage[ruled,section]{algorithm}
%\usepackage{algpseudocode}


% AMS Theorems
\theoremstyle{plain}% default 
\newtheorem{thm}{Theorem}[section] 
\newtheorem{prob}[thm]{Problem}
\newtheorem{question}[thm]{Question}

\newtheorem{lem}[thm]{Lemma}
\newtheorem{prop}[thm]{Proposition}
\newtheorem*{cor}{Corollary}

\theoremstyle{definition}
\newtheorem{defn}[thm]{Definition}
\newtheorem{conj}[thm]{Conjecture}
\newtheorem{exmp}[thm]{Example}

\theoremstyle{remark} 
\newtheorem*{rem}{Remark} 
\newtheorem*{note}{Note} 
\newtheorem{case}{Case} 

\newcommand{\expc}[1]{\mathbb{E}\left[#1\right]}

\newcommand{\Q}{\mathbb{Q}}
\newcommand{\R}{\mathbb{R}}
\newcommand{\C}{\mathbb{C}}
\newcommand{\Z}{\mathbb{Z}}
\newcommand{\F}{\mathbb{F}}
\newcommand{\Ga}{\mathbb{G}}

\newcommand{\Cat}{\mathcal{K}}
\newcommand{\Id}{\mathit{id}_}

\usepackage{xparse}
\NewDocumentCommand{\monoid}{m}
 {
  \makeargument{#1}
  \begin{tikzcd}[->,font=\normalsize,>=angle 90,ampersand replacement=\&]
  \argument{0} \arrow{d}{\argument{5}} \arrow[<->]{rr}{\alpha}\& \& \argument{1} \arrow{d}{\argument{6}}\\
  \argument{2} \arrow{dr}{\argument{7}} \& \& \argument{3}\arrow{dl}{\argument{8}}\\
   \& \argument{4} \&
  \end{tikzcd}
 }
\ExplSyntaxOn
\seq_new:N \l_sk_monoid_seq
\cs_new:Npn \makeargument #1
 { \seq_set_split:Nnn \l_sk_monoid_seq { & } { #1 } }
\cs_new:Npn \argument #1
 { \seq_item:Nn \l_sk_monoid_seq {#1} }
\ExplSyntaxOff



\newcommand{\nth}{n\textsuperscript{th}}
\newcommand{\bigo}[1]{\mathcal{O}(#1)}

% \setlength{\topmargin}{0.2cm}
% \setlength{\footskip}{0.2cm}
% \setlength{\hoffset}{-1cm}
% \setlength{\voffset}{-2cm}

	
\usepackage{hyperref}
	
\title{Category Theory}                               % Document Title
\author{Edward O'Callaghan}
%\date{}                                           % Activate to display a given date or no date



\begin{document}
\maketitle \tableofcontents \clearpage

\section{Prelude} % (fold)
\label{sec:purelude}
First a word on how to study category theory. Category theory is extremely heavy on new terminology at first sight.
It is typically misconceived as abstractly and arbitrarily renaming things to no real gain and a generally poorly understood
topic in undergraduate mathematics. Consider taking the time to highlight these notes in assortment of coloured pens,
with each new term given a corresponding colour. To get over the initial shock of, perhaps, new terms and to consolidate concepts.
Also note that everything may be represented as a diagram and that we make the notion of a diagram rigous later. However,
a categorial diagram is simply a directed graph, where the nodes are objects from a category while morphism are the edges.
I highly recommend you copy each new term with a corresponding diagram to anecdote the idea at hand.
If all else fails, be sure to remember to break the term down into prefix and suffix and reason it.
Typically terms are fairly simple to reason out (e.g., mono = ``one'', auto = ``self'', morphism = ``to shape'').

\begin{note}[Class]
 A $\emph{class}$ is a collection of mathematical objects (e.g., sets)
 which can be unambiguously defined by a property that all its members share.
\end{note}

\begin{note}[Morphism]
 A $\emph{morphism}$, from Greek, is an abstraction derived from
 structure-preserving mappings between two mathematical structures.
 The notion of morphism recurs in much of contemporary mathematics.
\end{note}

\section{Introduction} % (fold)
\label{sec:introduction}
In each area of mathematics (e.g., sets, groups, topological spaces) there are available many definitions
and constructions. It turns out, however, that there are a number of notions (e.g. that of a product)
that occur naturally in various areas of mathematics, with only slight changes from one area to another.
It is convenient to take advantage of this observation. Category theory can be described as that branch of mathematics
in which one studies certain definitions in a broader context - without reference to the particular area to which
the definition might be applied.


\section{Morphisms} % (fold)
\label{sec:morphisms}
A morphism is a map between two objects in an abstract category from $\emph{dom}$ain to $\emph{codom}$ain, more on this later.
A general morphism is called a homomorphism. A homomorphism is a term typically used in category theory.
The term itself derives from the Greek $o \mu o$ (omo), ``alike'' and $\mu o \rho\varphi\omega\sigma\iota\varsigma$
(morphosis), ``to form'' or ``to shape''.
\begin{rem}
 The similarity in meaning and form of the words ``homomorphism'' and ``homeomorphism'' is unfortunate and typically
 a source of common confusion.
\end{rem}

Morphisms can have any of the following properties.

\subsection{Monomorphism} % (fold)
\label{subsec:monomorphism}
\begin{defn}[Monomorphism]
 A morphism $\varphi: X \to Y$ is a $\emph{monomorphism}$ if $\varphi \circ \psi = \varphi \circ \phi \implies \psi = \phi$ for all morphisms $\psi,\phi: Z \to X$.
 \\
 \begin{tikzcd}
  Z \rar[transform canvas={yshift=.5ex}]{\psi} \rar[transform canvas={yshift=-.5ex}, swap]{\phi} & X \rar{\varphi} & Y
 \end{tikzcd}
 \end{defn}

\subsection{Epimorphism} % (fold)
\label{subsec:epimorphism}
\begin{defn}[Epimorphism]
 A morphism $\varphi: X \to Y$ is a $\emph{epimorphism}$ if $\psi \circ \varphi = \phi \circ \varphi \implies \psi = \phi$ for all morphisms $\psi,\phi: Y \to Z$.
 \\
 \begin{tikzcd}
  X \rar{\varphi} & Y \rar[transform canvas={yshift=.5ex}]{\psi} \rar[transform canvas={yshift=-.5ex}, swap]{\phi} & Z
 \end{tikzcd}
\end{defn}
 
\subsection{Bimorphism} % (fold)
\label{subsec:bimorphism}
\begin{defn}[Bimorphism]
 A morphism $\varphi: X \to Y$ is a $\emph{bimorphism}$ if $\varphi$ is both a monomorphism and epimorphism.
\end{defn}

\subsection{Isomorphism} % (fold)
\label{subsec:isomorphism}
\begin{defn}[Isomorphism]
 A morphism $\varphi: X \to Y$ is a $\emph{isomorphism}$ if there exists a morphism $\psi: Y \to X : \varphi \circ \psi = \Id{Y}$ and $\psi \circ \varphi = \Id{X}$.
\end{defn}

\subsection{Endomorphism} % (fold)
\label{subsec:endomorphism}
\begin{defn}[Endomorphism]
 A morphism $\varphi: X \to X$ is a $\emph{endomorphism}$.
 The class of endomorphisms of $\varphi$ is denoted by $end(\varphi)$.
\end{defn}

\subsection{Automorphism} % (fold)
\label{subsec:automorphism}
\begin{defn}[Automorphism]
 A morphism $\varphi: X \to Y$ is a $\emph{automorphism}$ if $\varphi$ is both an endomorphism and an isomorphism.
 The class of automorphisms of $\varphi$ is denoted by $aut(\varphi)$.
\end{defn}

\subsection{Retractions and Sections} % (fold)
\label{subsec:retractionsandsections}
\begin{defn}[Retraction]
 A morphism $\varphi: X \to Y$ has a $\emph{retraction}$ if there exists a morphism $\psi: Y \to X$ with $\varphi \circ \psi = \Id{Y}$ - ``if a right inverse of $\varphi$ exists''.
\end{defn}

\begin{defn}[Section]
 A morphism $\varphi: X \to Y$ has a $\emph{section}$ if there exists a morphism $\psi: Y \to X$ with $\psi \circ \varphi = \Id{X}$ - ``if a left inverse of $\varphi$ exists''.
\end{defn}

\begin{lem}[Retractions and Sections]
Every retraction is an epimorphism and every section is a monomorphism.
We have the following three equivalent statements:
\begin{itemize}
\item $\varphi$ is a monomorphism and a retraction;
\item $\varphi$ is an epimorphism and a section;
\item $\varphi$ is an isomorphism.
\end{itemize}
\end{lem}


\section{Categories} % (fold)
\label{sec:categories}
A category consists of three things:
a collection of objects, for each pair of objects a collection of morphisms from one to another, and a binary
operation defined on compatible pairs of morphisms called composition denoted by $\circ$. The category must satisfy an identity axiom
and an associative axiom which is analogous to the monoid axioms, seen later. We require that morphisms preserve the
mathematical structure of the objects (i.e., for vector spaces as objects one would choose linear maps as morphisms, and so on).
\begin{defn}[Category]
	A $\emph{category}$ $\Cat$ consists of the following three mathematical entities:
	\begin{enumerate}
		\item A $\emph{class}$ $\text{Ob}(\Cat)$ of objects
		
		\item A class $\text{Hom}(A,B)$ of $\emph{morphisms}$, from $A \longrightarrow B$ such that $A, B \in \text{Ob}(\Cat)$.
		\\
		e.g. $\varphi : A \to B$ to mean $\varphi \in \text{Hom}(A,B)$.
		\begin{rem}
		The class of \emph{all} morphisms of $\Cat$ is denoted $\text{Hom}(\Cat)$.
		\end{rem}

		\item Given $A, B, C \in \text{Ob}(\Cat)$, a binary operation $\circ : Hom(A,B) \times Hom(B,C) \to Hom(A,C)$ called
		\emph{composition}, satisfying:
		\begin{enumerate}
			\item \emph{(closed associativity)} Given $\varphi : A \to B$, $\psi: B \to C$ and $\phi: A \to C$
			\\
			we have $\phi \circ (\psi \circ \varphi) = (\phi \circ \psi) \circ \varphi$.
			\\
			\begin{tikzcd}
			A \drar{\phi} \rar{\varphi}
			& B \dar{\psi} \\
			& C
			\end{tikzcd}
			
			\item \emph{(identity)} For any object X there is an identity morphism $\Id{X} : X \to X$ such that for any $\varphi: A \to B$ we have $\Id{B} \circ \varphi = \varphi = \varphi \circ \Id{A}$.
			\\
			\begin{tikzcd}
			X \arrow[in=30, out=60, loop]{}[name=idX]{\Id{X}}
			\end{tikzcd}
		\end{enumerate}
	\end{enumerate}
\end{defn}

It is also worth noting about what we mean by `small' and `large' categories.

\begin{defn}[Small Category]
	A category $\Cat$ is called small if both $\text{Ob}(\Cat)$ and $\text{Hom}(\Cat)$ are sets.
	If $\Cat$ is not small, then it is called large.
	$\Cat$ is called locally small if $\text{Hom}(A,B)$ is a set for all $A, B \in \text{Ob}(\Cat)$.
\end{defn}

\begin{rem}
	Most important categories in mathematics are not small however, are locally small.
\end{rem}

Most basic categories have as objects certain mathematical structures, and the structure-preserving
functions as morphisms. A common list is given;
\begin{itemize}
\item $\emph{Pos}$ is the category of partially ordered sets and monotonic functions.
\item $\emph{Top}$ is the category of topological spaces and continuous functions.
\item $\emph{Rng}$ is the category of rings and ring homomorphisms.
\item $\emph{Grp}$ is the category of groups and group homomorphisms.
\item $\emph{Grph}$ is the category of graphs and graph homomorphisms.
\end{itemize}

\section{Dual Category} % (fold)
\label{sec:dualcategory}
$\emph{Duality}$ is a correpsondence between properties of a category $\mathcal{K}$ and the notion
of $\emph{dual properties}$ of the ``$\emph{op}$posite category'' $\mathcal{K}^{op}$. Suppose some proposition
regarding a category $\mathcal{K}$, by interchanging the domain and codomain of each morphism as well as
the order of composition, a corresponding dual proposition is obtained of $\mathcal{K}^{op}$.

Duality, as such, is the assertion that truth is invariant under this operation on propositions.
\begin{lem}
 $(\mathcal{K}^{op})^{op} = \mathcal{K}$.
\end{lem}

\begin{exmp}
 Suppose a monomorphism in a category $\mathcal{K}$ then an epimorphism is the categorial dual
 in the dual category $\mathcal{K}^{op}$. Prove this by diagram.
\end{exmp}

\begin{rem}
 We make the notion of diagram and hence the idea of commutative diagrams rigous later.
\end{rem}


\section{Functors} % (fold)
\label{sec:functors}
A category is itself a type of mathematical structure and so, one can generalise the notion of a morphism thus preserve this structure by the notion of a
functor.
A functor associates to every object of one category an object of another category, and to every morphism in the first category a morphism in the second.
Hence, functors are structure-preserving maps between categories and can be thought of as morphisms in the category of all (small) categories.

In particular, what we have done is define a category of categories and functors - the objects are categories, and the morphisms (between categories) 
are functors.
By studying categories and functors, we are not merely studying a class of mathematical structures and the morphisms between them,
we are studying the relationships between various classes of mathematical structures.

\begin{defn}[Functor]
	Let $\mathcal{C}$ and $\mathcal{K}$ be categories. A $\emph{functor}$ $F$ from $\mathcal{C}$ to $\mathcal{K}$ is a mapping that:
	\begin{enumerate}
		\item associates to each object $X \in \mathcal{C}$ an object $F(X) \in  \mathcal{K}$
		
		\item associates to each morphism $\varphi: X \to Y \in \mathcal{C}$ a morphism $F(\varphi) : F(X) \to F(Y) \in \mathcal{K}$
		satisfying:
		\begin{enumerate}
			\item $F(id_{X}) = id_{F(X)}$ for every object $X \in \mathcal{C}$
			
			\item $F(\psi \circ \varphi) = F(\psi) \circ F(\varphi)$ for all morphisms $\varphi: X \to Y$ and $\psi: Y \to Z$
			\begin{rem}
				That is, functors must preserve identity morphisms and composition of morphisms.
			\end{rem}
		\end{enumerate}
	\end{enumerate}
\end{defn}

\subsection{Types of Functors} % (fold)
\label{subsec:functorstypes}
Like many things in category theory, functors come in a kind of ``dual'' type in concepts;
that of the $\emph{contravariant}$ and $\emph{covariant}$ functors, defined as follows:

\begin{defn}[Covariant Functor]
 A $\emph{covariant}$ functor $F$ from a category $\mathcal{C}$ to a category $\mathcal{K}$, $F: \mathcal{C} \to \mathcal{K}$, consists of:
 \begin{itemize}
  \item for each object $X \in \mathcal{C}$, an object $F(X) \in \mathcal{K}$
  \item for each morphism $\varphi: X \to Y \in \mathcal{C}$, a morphism $F(\varphi): F(X) \to F(Y)$
 \end{itemize}
 provided the following two properties hold:
 \begin{itemize}
  \item For every object $X \in \mathcal{C}$, $F(\Id{X}) = \Id{F(X)}$
  \item For all morphisms $\varphi: X \to Y$ and $\psi: Y \to Z$, $F(\psi \circ \varphi) = F(\psi) \circ F(\varphi)$
 \end{itemize}
\end{defn}

\begin{defn}[Contravariant Functor]
 A $\emph{contravariant}$ functor $F: \mathcal{C} \to \mathcal{K}$, is a `reversed' covariant functor.
 In particular, for every morphism $\varphi: X \to Y \in \mathcal{C}$ must be assigned to a morphism $F(\varphi): F(Y) \to F(X) \in \mathcal{K}$.
 Alternatively, contravariant functors act as covariant functors from the opposite category $\mathcal{C}^{op} \to \mathcal{K}$.
\end{defn}

Functors have particular properties that essentially are the same as morphisms.
This may seem obvious given that functors are a generalisation of the notion of a morphism.
So, the following definitions should mostly be obvious if you reflect back to the corresponding morphism
definitions above or simply break the word down grammatically.

\begin{defn}[Endofunctor]
 A $\emph{endofunctor}$ is a functor that maps a category to itself, sometimes called a $\emph{identity functor}$.
\end{defn}

\begin{defn}[Bifunctor]
 A $\emph{bifunctor}$ is a functor in $\emph{two}$ arguments. More formally, a bifunctor is a functor whose
 domain is a $\emph{product category}$.
 The $Hom$ functor is a natural example; it is contravariant in one argument while covariant in the other.
 Hence, the $Hom$ functor is of the type $\mathcal{C}^{op} \times \mathcal{C} \to \textbf{Set}$.
\end{defn}

\begin{defn}[Multifunctor]
 A $\emph{multifunctor}$ is a generalisation of the functor concept to n variables, (e.g., A $\emph{bifunctor}$ is when $n=2$)
\end{defn}

\subsection{Properties of Functors} % (fold)
\label{subsec:functorsproperties}
However, some properties are more particular to functors and we review them here to families ourselves.
A functor $F : \mathcal{A} \to \mathcal{B}$ is
\begin{enumerate}
 \item $\emph{faithful}$ if for every parallel pair of morphisms $f,g : A \rightrightarrows A' \in \mathcal{A}$, one has $f = g$ whenever $F(f) = F(g)$
 \item $\emph{full}$ if for every morphism $b: F(A) \to F(A') \in \mathcal{B}$, there exists a morphism $a: A \to A' \in \mathcal{A}$ such that $F(a) = b$
 \item essentially $\emph{surjective}$ if for every object $B \in \mathcal{B}$, there exists an object $A \in \mathcal{A}$ with $B$ isomorphic to $F(A)$
 \item an $\emph{equivalence}$ if there exists a functor $F': \mathcal{B} \to \mathcal{A}$ such that both $F \circ F'$ and $F' \circ F$ are naturally
 isomorphic to the identity functors; such a functor $F'$ is called a $\emph{quasi-inverse}$ of $F$
 \item an $\emph{isomorhpism}$ if there exists a functor $F': \mathcal{B} \to \mathcal{A}$ such that both $F \circ F'$ and $F' \circ F$
 are equal to the identity functors
 \item $\emph{conserative}$ if it reflects isomorphisms; that is, $a: A \to A'$ is an isomorphism whenever $F(a): F(A) \to F(A')$ is
\end{enumerate}

\begin{defn}[Functor category and the Yoneda embedding]
The Yoneda lemma suggests that instead of studying the $\emph{locally small}$ category $\mathcal{K}$,
one should study the category of all functors of $\mathcal{K}$ into $\textbf{Set}$.
Since the category $\textbf{Set}$ is well understood, a functor of $\mathcal{K}$ into $\textbf{Set}$
maybe seen as a $\emph{representation}$ of $\mathcal{K}$ in terms of known structures.
See the $\emph{category of diagrams}$ later.
 \begin{enumerate}
  \item Given a category $\mathcal{K}$ and a small category $\mathcal{J}$, we denote by
  $\mathcal{K}^\mathcal{J}$ the category of functors from $\mathcal{J} \to \mathcal{K}$ and natural transformations
  of each functor from $\mathcal{J} \to \mathcal{K}$.
  \item In case $\mathcal{K} = \textbf{Set}$, we have the $\emph{Yoneda embedding}$:\\
  $Y_{\mathcal{J}}: \mathcal{J}^{op} \to \textbf{Set}^{\mathcal{J}}
  \, \,
  Y_{\mathcal{J}}(X) = \mathcal{J}(X, -)$, which is full and faithful.
 \end{enumerate}
\end{defn}


\section{Monoids} % (fold)
\label{sec:monoids}
A $\emph{monoid}$ is an abstract mathematical structure normally associated with that of a semigroup with identity.
A semigroup with identity, or just monoid, is some set $M$ together with a law of associative composition,
$\forall x,y,z \in M : (x \circ y) \circ z = x \circ (y \circ z)$, that is both closed, $\forall x,y \in M : x \circ y \in M$,
and has idenity not in $M$ such that $\Id{} \circ x = x = x \circ \Id{} \forall x \in M$. More compactly,

\begin{defn}[Monoid]
 A category with exactly one object is called a $\emph{monoid}$.
\end{defn}

Thus, expanding out the definition of a category we have;
\begin{cor}[Monoid]
 A $\emph{monoid}$ is a set $X$ together with a law of composition $\circ$
 which associates, and has identity $\Id{} \in X$.
\end{cor}

\begin{exmp}
 Let $M = \Z$ and the law of composition $+$ as arithmetic addition with $0$ as identity.
\end{exmp}

\begin{exmp}
 A list is an example of a monoid with the empty list $()$ as identity and the law of composition $++$ as appending to the list.
\end{exmp}


% \begin{exmp}
%  The following commutative diagram illustrates the monoidal structure of a group:
%  \monoid{
%   (A\otimes A)\otimes A\otimes A &
%   A\otimes (A\otimes A) &
%   A\otimes A &
%   A\otimes A &
%   A &
%   \mu\otimes \Id{} &
%   \Id{} \otimes\mu &
%   \mu &
%   \mu\otimes \Id{}
%  }
% \end{exmp}


\section{Natural Transformations} % (fold)
\label{sec:natural transformations}
A natural transformation provides a way of transforming one functor into another
while respecting the internal structure (i.e. the composition of morphisms) of the categories involved.
Hence, a natural transformation can be considered to be a ``morphism of functors''.
Indeed this intuition can be formalized to define so-called functor categories.
Natural transformations are, after categories and functors, one of the most basic notions of category theory
and consequently appear in the majority of its applications.

\begin{defn}[Natural Transformation]
 Let $F$ and $G$ be functors both from the category $\mathcal{C}$ to $\mathcal{K}$.
 Then a $\emph{natural transformation}$ $\eta : F \to G$ associates to every object
 $X \in \mathcal{C}$ a morphism $\eta_{X} : F(X) \to G(X) \in \mathcal{K}$, called the $\emph{component}$
 of $\eta$ at $X$. The component morphism $\eta_{X}$ is subject to the condition of $\emph{naturality}$,
 that is, the diagram commutes for every morphism $\varphi: X \to Y \in \mathcal{C}$; or simply,
 $\eta_{Y} \circ F(\varphi) = G(\varphi) \circ \eta_{X}$.
 \\
 We draw the commutative diagram in the category $\mathcal{K}$ know as the $\emph{naturality square}$:
 \begin{tikzcd}
 F(X) \rar[][color=red]{\eta_{X}} \dar[]{F(\varphi)}
  & G(X) \dar[][color=red]{G(\varphi)} \\
 F(Y) \rar{\eta_{Y}}
  & G(Y)
 \end{tikzcd}
\end{defn}
\begin{rem}
 N.B. The naturality square is the key concept to understanding natural transformations and
 there is one of them for every morphism in $\mathcal{C}$ as stated above. Note this well.
\end{rem}

Constructions are often "naturally related" - a vague notion, at first sight.
This leads to the clarifying concept of natural transformation, a way to map one functor to another.
\begin{rem}
 Various important constructions in mathematics can be studied in this context.
 Naturality is a principle, like that of general covariance in physics, that cuts deeper than is initially apparent.
\end{rem}


\section{Universal constructions} % (fold)
\label{sec:universalconstructions}
Categorial objects are both abstract and have atomic type.
Thus we are face with the issue of defining objects without referring to the internal structure.
To characterise these objects, we look at them from the prospective of their relations to other objects, as given by the morphisms 
of the respective categories. Therefore we must find universal properties that uniquely determine the objects of interest.
Indeed, it turns out that numerous important constructions can be described purely in this way.
The central concept which is needed for this purpose is called categorical limit, and can be dualized to yield the notion of a colimit.

\subsection{Diagram} % (fold)
\label{subsec:diagram}
A diagram is the categorial analogue of an indexed family in set theory. In particular, one also must consider that
morphism as well as objects must be indexed. Hence, a diagram is a collection of objects and morphisms, that is,
a category that is indexed by another fixed category. Since we are mapping from one category to another, a diagram is
naturally then a functor as you may expect.
\begin{defn}[Diagram]
 A $\emph{diagram}$ $D$ is given by the functor $D: \mathcal{J} \to \mathcal{K}$
 where $\mathcal{J}$ is some $\emph{index category}$ in the category $\mathcal{K}$.
\end{defn}

The acutal detail of the internals of the index category $\mathcal{J}$ is of no concern, merely the internal relationships
are of importance. Hence, diagrams and functors are technically the same thing, however, a different matter of prospective.

\begin{cor}
 A diagram is said to be $\emph{small}$, or finite, whenever the index category $\mathcal{J}$ is.
\end{cor}

Again, recall that a digram is a essentially a functor.
Now, suppose two diagrams $D_{1}$ and $D_{2}$ by some fixed indexing category $\mathcal{J}$ in the category $\mathcal{K}$.
Then each morphism in $D_{1}$ can be found in $D_{2}$ by the natural transformation from $D_{1}$ to $D_{2}$.
Hence we have motivation to form a definition of the $\emph{category of diagrams}$, or $\emph{functor category}$ when
$\mathcal{J}$ is small, given by the following:

\begin{defn}[Category of Diagrams]
 The category of diagrams by the index category, or type, $\mathcal{J}$ in the category $\mathcal{K}$ as the $\emph{functor category}$,
 denoted by $\mathcal{K}^{\mathcal{J}}$ with diagrams as the defined object class.
\end{defn}

\begin{exmp}
 Take some small type $\mathcal{J}$ in the category $\mathcal{K}$,
 then $\mathcal{J}$ merely indexes the objects in the category $\mathcal{K}$.
 \begin{rem}
  Also note that the diagram is again be finite since $\mathcal{J}$ is.
 \end{rem}
\end{exmp}

One of the wonderful results of diagrams in category theory is that we have a new simple, yet effective, proving technique.
This technique of proving with categorial diagrams is know as $\emph{diagram chasing}$ where we look for equivalent map compositions in commutative diagrams.
Typically exploiting the properties of monomorphic, epicmorphic and bimorphic type homomorphisms and their $\emph{exact sequences}$.
\begin{note}
 As a side note;
 Another area of mathematics where diagram chasing is oftern used is that of homological algebra, where Modules are of primary concern.
\end{note}

\subsection{Cone} % (fold)
\label{subsec:cone}
The abstract notion of a $\emph{cone of a functor}$ shall be the essence of what it is to define $\emph{the limit of a functor}$.

\begin{defn}
 Given some diagram $F: \mathcal{J} \to \mathcal{K}$ with type $\mathcal{J}$ in the category $\mathcal{K}$.
 A $\emph{cone}$ to the diagram $F$ is an object $N$ of the category $\mathcal{K}$ endowed with a family $\Psi_{X}: N \to F(X)$ of morphisms
 indexed by $\text{Ob}(\mathcal{J})$ such that every morphism $\varphi: X \to Y \in \mathcal{J}$, we have $F(\varphi) \circ \Psi_{X} = \Psi_{Y}$.
 The cone is denoted by the pair $(N, \Psi)$.
\end{defn}

Hence, we have the following commutative diagram:
\begin{tikzcd}
 N \dar[bend right, swap]{\Psi_{X}} \drar[bend left]{\Psi_{Y}} \\
 F(X) \rar[swap]{F(\varphi)} & F(Y)
\end{tikzcd}

\begin{rem}
 The cone $\Psi$ can be said to have vertex $N$ with base $F$.
\end{rem}

\begin{rem}[Cocone]
 TODO..
\end{rem}

\subsection{Limit} % (fold)
\label{subsec:limit}
TODO..

\begin{rem}[Colimit]
 TODO..
\end{rem}

% \begin{exmp}
%  Suppose that $U: \mathcal{K} \to \mathcal{C}$ is a functor from a category $\mathcal{K}$ to a category $\mathcal{C}$,
% and let $X \in \text{Ob}(\mathcal{C})$. Consider the following dual notions:
% \begin{defn}[Initial morphism]
%  An $\emph{initial morphism}$ from $X \to U$ is an initial object in the category
%  $(X \downarrow U)$ of morphisms from $X \to U$. In particular, it consists
%  of a pair $(A, \varphi)$ where $A \in \text{Ob}(\mathcal{K})$ and $\varphi: X \to U(A) \in \text{Hom}(\mathcal{C})$,
%  such that the following initial property is satisfied:
%  \\
%  Whenever $Y \in \text{Ob}(\mathcal{K})$ and $f: X \to U(Y) \in \text{Hom}(\mathcal{C})$, then there exists a unique morphism
%  $g: A \to Y$ such that the following diagram commutes:
% \begin{tikzcd}
%  X \drar[swap]{f} \rar{\varphi} & U(A)
%  \dar[dashed]{U(g)} \\ & U(Y)
% \end{tikzcd}
% \end{defn}
% 
% \begin{defn}[Terminal morphism]
%  A $\emph{terminal morphism}$ from $U \to X$ is a terminal object in the $\emph{comma category (i.e., morphisms become objects)}$
%  $(U \downarrow X)$ of morphisms from $U \to X$. In particular, it consists of a pair $(A, \varphi)$
%  where $A \in \text{Ob}(\mathcal{K})$ and $\varphi: U(A) \to X \in \text{Hom}(\mathcal{C})$,
%  such that the following terminal property is satisfied:
%  \\
%  Whenever $Y \in \text{Ob}(\mathcal{K})$ and $f: U(Y) \to X \in \text{Hom}(\mathcal{C})$, then there exists a unique morphism
%  $g: Y \to A$ such that the following diagram commutes:
% \begin{tikzcd}
%  U(Y) \dar[dashed,swap]{U(g)} \drar{f} \\
%  U(A) \rar[swap]{\varphi} & X
% \end{tikzcd}
% \end{defn}
% \end{exmp}


\section{Monads} % (fold)
\label{sec:monads}
$\emph{Monads}$ derived from Greek, $\mu o \nu\alpha\varsigma$ (monas), ``unit''.
The computer science interpretation is that of a unit of computation.
Monads are like algebraic theories and that a algebra for a monad is a model of that theory. A monad is simply a functor
with some extra 'stuff' that tells us what is going on in the associated theory.
The notion of $\emph{algebras for a monad}$ generalises classical notions from universal algebra,
and in this sense, monads can be thought of as $\emph{theories}$.
More formally, a monad or $\emph{(Kleisli triple)}$, is an endofunctor coupled with two natural transformations,
$\eta$ and $\mu$ that satisfy so-called $\emph{coherence conditions}$.

\begin{defn}[Monad]
 Given a category $\mathcal{K}$, a $\emph{monad}$ on $\mathcal{K}$ is the endofunctor $T: \mathcal{K} \to \mathcal{K}$
 coupled with two natural transformations, $\eta: \Id{\mathcal{C}} \rightrightarrows T$ and $\mu: T^2 \rightrightarrows T$
 that satisfy the coherence conditions:
 \begin{itemize}
  \item $\mu \circ T \mu = \mu \circ \mu T$, and
  \item $\mu \circ T \eta = \mu \circ \eta T = \Id{T}$.
 \end{itemize}
\end{defn}
\begin{rem}
 N.B. $\eta$ called $\emph{the unit}$ since it is a natural mapping from idenity and
 $\mu$ called $\emph{the multiplication}$ also noting the natural mapping from $T^2 = T \circ T$.
\end{rem}

Recall the definition of a natural transformation of $\eta$ and $\mu$, that for every $X \in \text{Ob}(\mathcal{K})$
there is a $\emph{component}$ morphism of $\eta$ and $\mu$ at $X$, denoted by, $\eta_{X}$ and $\mu_{X}$ respectively.
\\
So, we have the following:
\begin{itemize}
 \item $\eta_{X}: X \to T(X) \in \mathcal{K}$,
 read, $\emph{``the component of $\eta$ at $X$ is a morphism in $\mathcal{K}$ from $X$ to $T$ of $X$''}$, and,
 \item $\mu_{X}: T^2(X) \to T(X) \in \mathcal{K}$,
 read, $\emph{``the component of $\mu$ at $X$ is a morphism in $\mathcal{K}$ from $T^2$ of $X$ to $T$ of $X$''}$.
\end{itemize}

Hence, the coherence conditions are shown the by commutative diagrams for every component $X \in \text{Ob}(\mathcal{K})$
\\
For the existence idenity by $\emph{the left and right unit triangles}$:
\begin{tikzcd}
T(X) \drar[equals] \rar[][color=red]{\eta_{T(X)}} \dar[swap]{T(\eta_{X})}
 & T^2(X) \dar[][color=red]{\mu_{X}} \\
T^2(X) \rar[swap]{\mu_{X}}
 & T(X)
\end{tikzcd}
\\
and for that of $\emph{the associativity square}$ by:
\begin{tikzcd}
T^3(X) \rar[][color=red]{T(\mu_{X})} \dar[swap]{\mu_{T(X)}}
 & T^2(X) \dar[][color=red]{\mu_{X}} \\
T^2(X) \rar[swap]{\mu_{X}}
 & T(X)
\end{tikzcd}
Since these are natural transformations,
we can write these without $X$ and by writing $\eta_{T(X)}$ as $\eta T$ and $T(\eta_{X})$ as $T \eta$, for example.
Since the commutative diagrams in the category $\mathcal{K}$ are component wise, that is, in terms of $X$.

\begin{rem}[Comonad]
 A $\emph{Co}$monad, or just $\emph{cotriple}$, is the categorical dual of a Monad.
 That is, A comonad for a category $\mathcal{K}$ is a monad for the opposite category $\mathcal{K}^{op}$.
\end{rem}


\begin{exmp}[Monad for monoids]
 From the definition of a monoid,
 take the category $\mathcal{K} = \textbf{Set}$ and a functor $T: X \to X^{*}$ where $X$ is the ``set of words'' in $X$.
 \begin{rem}
  A word is just a list of objects in $X$ e.g., $(x_{1}, x_{2}, x_{3})$ or simply the empty list $()$.
 \end{rem}
 Therefore, we define the natural transformations component wise, as: $\eta_{X}: X \to X^{*}$ and $\mu_{X}: X^{**} \to X^{*}$.
 \begin{rem}
  To understand $\mu_{X}: X^{**} \to X^{*}$ consider the example: $(a, (b, c)) \to (a, b, c)$
 \end{rem}
 Finish me...... TODO.
\end{exmp}

\begin{note}
Haskell Note:
\begin{itemize}
 \item $\eta$ is the Monadic equivalent of the "return" function and,
 \item $\mu$ is the Monadic equivalent of the "join" operator.
\end{itemize}
\end{note}


\subsection{Algebras for Monads} % (fold)
\label{sec:monadalgebra}
Given the monad $(T, \eta, \mu)$ on a category $\mathcal{K}$. A $T$-algebra denoted by $(x,h)$,
where $x \in \text{Ob}(\mathcal{K})$ coupled with $h \in \text{Hom}(\mathcal{K})$ with
$h: T(x) \to x \in \mathcal{K}$. The morphism $h$ is called the structure map of the algebra $T$ provided
the usual diagrams commute (i.e., the coherence conditions are satisfied).

The $\emph{the associativity square}$ by:
\begin{tikzcd}
T^2(x) \rar[][color=red]{T(x)} \dar[swap]{\mu_{x}}
 & T(x) \dar[][color=red]{h} \\
T(x) \rar[swap]{h}
 & x
\end{tikzcd}
\\
and the $\emph{right unit triangle}$ (idenity) by:
\begin{tikzcd}
 x \drar[swap]{\Id{x}} \rar{\eta_{x}} & T(x)
 \dar[]{h} \\ & x
\end{tikzcd}

A morphism $f: (x,h) \to (x',h')$ of $T$-$\emph{algebras}$ is a morphism $f: x \to x' \in \mathcal{K}$
given that the following diagram commutes:
\begin{tikzcd}
T(x) \rar[][color=red]{T(f)} \dar[swap]{h}
 & T(x') \dar[][color=red]{h'} \\
x \rar[swap]{f}
 & x'
\end{tikzcd}


\end{document}% End of document.
