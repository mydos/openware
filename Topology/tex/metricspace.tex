% Copyright © 2012 Edward O'Callaghan. All Rights Reserved.

\section{Metric space} % (fold)
\label{sec:metricspace}

\subsection{Metrics}

\begin{defn}[Metric space]
	A $\emph{metric space}$ is a order pair $(X,d)$
	where $X$ is a set and $d$ is some function
	$d: X \times X \to X$ that satisfies, for all
	$x,y,z \in X$,
	\begin{itemize}
		\item $d(x,y) \geq 0$ and $d(x,y)=0$ iff $x=y$;
		\item $d(x,y)=d(y,x)$ (symmetric);
		\item $d(x,z) \leq d(x,y) + d(y,z)$ (triangle inequality).
	\end{itemize}
	We call $d$ a metric on $X$.
\end{defn}

\begin{prob}[Discrete metric]
	Suppose
	\[
		d(x,y) =
		\begin{cases}
			1 & \text{if } x \neq y,\\
			0 & \text{if } x = y.
		\end{cases}
	\]
	Prove $d(x,y)$ defines a metric.
\end{prob}

\begin{exmp}[Eucliean metric]
	Consider the set of real n-tuples $M=\R^n$.
	
	For points $\mathbf{x}=\{x_1,\dots,x_n\}$ and
	$\mathbf{y}=\{y_1,\dots,y_n\}$ in $\R^n$ we set
	\[
		d(x,y) = \left( \sum_{i=1}^{n} (x_i - y_i)^2 \right)^{\frac{1}{2}}
	\]
\end{exmp}

\begin{defn}[Continuity]
	Let $(X,d_{X})$ and $(Y,d_{Y})$ be metric spaces.
	We say that the mapping $f: X \to Y$ is
	$\emph{continuous at a point}$ $x_0 \in X$, if
	\[
		\forall \epsilon >0 \, \exists \delta >0 \,
		\forall x \in X : d_{X}(x,x_0) < \delta \implies
		d_{Y}(y,y_0) < \epsilon.
	\]
	The mapping $f: X \to Y$ is said to be $\emph{continuous}$
	if $f$ is continuous $\emph{at every point}$ $x_0 \in X$.
\end{defn}

\subsection{Topology of a metric space}

A metric space provides sufficient structure to study the notions of
convergence and thus continuity. A closer study of continuity of mappings
in the setting of metric spaces revels that a metric need not be of a
specific type. Rather, a class of subsets defined by the metric lead to
the concept of the underlying $\emph{topology}$ in a metric space that is
decisive for continuity to make sense.

\begin{defn}[Open set]
	A subset $U$ of a metric space $M=(X,d)$ is said to be $\emph{open}$ if;
	\[
		\forall x \in U \, \exists \epsilon > 0 : d(x,y)
		< \epsilon \, \forall y \in X \implies y \in U.
	\]
\end{defn}

Alternatively, we may consider defining the notion of a
$\emph{open ball}$ $B_{\epsilon}(x)$ and using this
equivalently to redefine a $\emph{open set}$.

\begin{defn}[Open ball]
	Let $M=(X,d)$ be an arbitrary metric space and let some point
	$x_0 \in X$ with $\epsilon \in \R^{+}$. Then an open ball
	with center $x_0$ and radius $\epsilon$ is defined as:
	\[
		B_{\epsilon}(x_0) = \{ x \in X : d(x_0,x) < \epsilon \}
	\]
\end{defn}

\begin{rem}
	A closed ball may be defined in a similar way, that is,
	\[
		\bar{B_{\epsilon}}(x_0) = \{ x \in X : d(x_0,x) \leq \epsilon \}
	\]
\end{rem}

Hence we have the alternative definition in the following way.

\begin{defn}[Open set - alternative]
	For some arbitrary metric space $M=(X,d)$ and open ball $B_{\epsilon}(x)$
	where $\epsilon>0$. A subset $U \subset X$ is said to be a $\emph{open set}$ if,
	\[
		\forall x \in U \, \exists \epsilon >0 \, : B_{\epsilon}(x) \subseteq U.
	\]
\end{defn}

\begin{thm}
	The family of open sets $\mathcal{T}$ in an arbitrary metric space
	$(X,d)$ has the following properties:
	\begin{itemize}
		\item $\emptyset , X \in \mathcal{T}$;
		\item Given some arbitrary finite set $\{U_i\}_{i=1}^{n} \subseteq \mathcal{T}$
			where each $U_i \subset X$, then
			$\displaystyle \bigcap_{i=1}^{n} U_i \in \mathcal{T}$;
		\item Given some arbitrary subsets
			$U_{\lambda} \in \mathcal{T} : \lambda \in \Lambda$ in $X$, then
			the union $\displaystyle \bigcup_{\lambda \in \Lambda} U_{\lambda} \in \mathcal{T}$.
	\end{itemize}
\end{thm}

Hence, a arbitrary metric space $M=(X,d)$ with some family of open sets
$\mathcal{T}$ in $M$ is a $\textbf{topology}$ $\mathcal{T}$ of $M$. The
pair $(M,\mathcal{T})$ is called a $\textbf{topological space}$.

\begin{proof}
	Since there are no points $x_0 \in \emptyset$, then the open ball
	$B_{\epsilon}(x)$ exists for all $x \in \emptyset$ for any $\epsilon > 0$.

	Conversely, since every point $x \in X$ exists then the open ball
	$B_{\epsilon}(x)$ exists for all $x_0 \in X$. Hence both $\emptyset$ and $X$ are open.
	
	If $\displaystyle x_0 \in \bigcap_{i=1}^{n} U_i$ then $x_0 \in U_i$ for each $U_i$.
	Now, since every $U_i$ is open, we have $B_{\epsilon_i}(x_0) \subseteq U_i$. By setting
	$\displaystyle \epsilon = \min_{1 \leq i \leq n} \epsilon_i$ we have $B_{\epsilon}(x) \subseteq U_i$
	for all $i$. Thus $B_{\epsilon}(x) \subseteq \displaystyle \bigcap_{i=1}^{n} U_i$ and so
	$\displaystyle \bigcap_{i=1}^{n} U_i$ is open.

	If $x_0 \in \displaystyle \bigcup_{\lambda \in \Lambda} U_{\lambda}$, then we may find
	some $\lambda_0 \in \Lambda$ so that $x_0 \in U_{\lambda_0}$. Since $U_{\lambda_0}$ is
	open we have $B_{\epsilon}(x) \subseteq U_{\lambda_0}$ for some $\epsilon > 0$. Now since
	\[
		U_{\lambda_0} \subseteq \bigcup_{\lambda \in \Lambda} U_{\lambda}
	\]
	then
	\[
		B_{\epsilon}(x) \subseteq \bigcup_{\lambda \in \Lambda} U_{\lambda}
	\]
	and so $\displaystyle \bigcup_{\lambda \in \Lambda} U_{\lambda}$ is open.
\end{proof}

\begin{lem}
	Any arbitrary metric space $M=(X,d)$ induces a topology $\mathcal{T}$ in $M$.
\end{lem}

Another piece of terminology that is often seen in topology is
that of a neighbourhood which we define here for completeness.

\begin{defn}[Neighbourhood]
	Suppose a arbitrary metric space $M=(X,d)$. A $\emph{neighborhood}$
	of some point $x \in X$ is a subset $V \subset X$ such that
	$B_{\epsilon}(x) \subseteq V$.

	In this case, we call the open set $V$ the $\epsilon$-neighborhood
	of the point $x$ in the set $X$.
\end{defn}

\subsection{Continuous maps and homeomorphisms}

By the abstraction of open sets we may describe continuity of a mapping by
way that is independent of the metric.

\begin{thm}
	Let $f:X \to Y$ be a mapping between metric spaces $(X,d_{X})$ and
	$(Y,d_{Y})$. Then $f$ is continuous if and only if for every open
	set $V$ in $Y$, the set $f^{-1}(V)$ is an open set in $X$.
\end{thm}

\begin{proof}
	Consider the mapping $f:X \to Y$ between metric spaces $(X,d_{X})$
	and $(Y,d_{Y})$. Now, suppose that $V \subset Y$ so that
	$\forall y \in V \, \exists \delta >0 : B_{\delta}(y) \subseteq V$
	is an open set $V$ in $Y$ and if $x \in f^{-1}(V)$, then there exists
	some $y \in V$ such that $y=f(x)$.
	
	Now, if $f$ is continuous and $U \subset X$ with $U=f^{-1}(V)$, we have
	\begin{align*}
		\forall x \in f^{-1}(V) \, \exists \epsilon & >0 : B_{\epsilon}(x) \subseteq f^{-1}(V)
		\\ \implies
		\forall x \in U \, \exists \epsilon & >0 : B_{\epsilon}(x) \subseteq U
	\end{align*}
	and so $U$ is a open set in $X$. We now show the converse.

	Suppose that $U=f^{-1}(V)$ is open in $X$ whenever $V=f(U)$ is open in $Y$. Then,
	\begin{align*}
		\forall x \in U \, \exists \epsilon & >0 : B_{\epsilon}(x) \subseteq U
		\\
		\forall x \in f^{-1}(V) \, \exists \epsilon & >0 : B_{\epsilon}(x) \subseteq f^{-1}(V)
		\\
		\forall y \in f \cdot f^{-1}(V) \, \exists \delta & >0 :
		B_{\delta}(y) \subseteq f \cdot f^{-1}(V)
		\\ \implies
		\forall y \in V \, \exists \delta & >0 : B_{\delta}(y) \subseteq V.
	\end{align*}
	Thus $f$ is continuous.
\end{proof}

\begin{defn}[Homeomorphism]
	A map $f: X \to Y$ is a $\emph{homeomorphism}$ if $f$ is a continuous bijection.
\end{defn}

\begin{rem}
	A homeomorphism is the structure preserving mapping between topological spaces.
	As we shall see later, a isomorphism is insufficient to preserve the structure
	of a topology since a isomorphism need not be continuous.
\end{rem}

\subsection{Closed sets}

\begin{defn}
	Consider a $\emph{sequence}$ $x_n$ in a metric space $(X,d)$. If $x \in X$
	and given some fixed $\epsilon > 0$ we may find a corresponding integer
	$N \geq 1$ such that
	\[
		d(x_n,x) < \epsilon \, \forall n \geq N.
	\]
	Alternatively, in terms of an $\epsilon$-neighbourhood we have,
	\[
		\lim_{n \to \infty} B_{\epsilon}(x_n) \subseteq B_{\epsilon}(x)
		\text{ for some } \epsilon > 0.
	\]
	In particular, we say that $x_n \to x$ as $n \to \infty$ and that $x$ is
	the $\emph{limit}$ of the sequence $x_n$.
\end{defn}

\begin{exmp}
	Consider two metric spaces $(X,d_{X})$ and $(Y,d_{Y})$.
	Show that a function $f: X \to Y$ is continuous if and only if,
	whenever $x_n \in X$ and $x_n \to x$ as $n \to \infty$, we have
	$f(x_n) \to f(x)$.

	We need only show that, for every open set $V$ in $Y$, the
	set $f^{-1}(V)$ is open in $X$. Hence, given some $U \subset X$
	with $U = f^{-1}(V)$, we have
	\begin{align*}
		\forall x_n \in U \, \exists \epsilon & >0 : B_{\epsilon}(x_n) \subseteq U
		\intertext{since,}
		\lim_{n \to \infty} B_{\epsilon}(x_n) & \subseteq B_{\epsilon}(x)
		\text{ for some } \epsilon >0
		\\ \implies
		\forall x \in U \, \exists \epsilon & >0 : B_{\epsilon}(x) \subseteq U
		\intertext{and so $U$ is a open set in $X$. Therefore}
		\forall y_n \in V \, \exists \delta & >0 : B_{\delta}(y_n) \subseteq V
		\\ \implies
		\forall y \in V \, \exists \delta & >0 : B_{\delta}(y) \subseteq V
	\end{align*}
	and so $V$ is open in $Y$ whenever $U$ is open in $X$.
\end{exmp}
