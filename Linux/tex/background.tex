% Copyright © 2012 Edward O'Callaghan. All Rights Reserved.

\section{Background} % (fold)
\label{sec:background}

The Linux kernel is the piece of software that mediates the interaction
of the needs of client applications to the underling physical hardware.
A typical scenario is that of a text editor which asks the operating system
to print some displayed text, the operating system then works out how to
talk to the printer hardware correctly to get this done. Another such
example is reading a file off a USB storage device in a file manager.
In this case the file manager asks the operating system what files are
stored on the media and the operating system returns a list of names
and locations.

In this course we study the GNU/Linux operating system. This is the
Linux kernel with the GNU \emph{userland} built on top. The GNU userland
provides various simple client applications to get common tasks done
such as moving, copying and renaming files. The userland also provides
system \emph{libraries} that provide common bits of code that these
userland applications share in their functionality. The central
system library on GNU/Linux is called \textbf{glibc}. This library
is probably the most important since it is the glue that fuses the
Linux kernel to userland programs for the fundamental operations such
as file manipulation. Any other client program you may be running is
also part of the userland however need not be part of the GNU core
system tools and libraries.

GNU itself stands for Gnu Not Unix. GNU is arguably the ordinal ideology
behind the \emph{open source} movement. Although other ideologies now
exist, such as the BSD's, the \textbf{GNU/GPL}, or General Public License,
has dominated the scene with its far left ideals. The GPL essentially
asserts that any changes to the code of the system must be published with
the original code and a copy of the GPL license with corresponding references
to who wrote the original works. The central idea is to \emph{force} the
workings of the system to be \emph{always open} and free to use! In this
way, the system is owned by no one and made by everyone who is feels the
need to contribute. Companies may contribute to GNU/Linux and may own the
rights to the original work. However, they do not own the right to demand
anything from you or force upon you questionable practices typically found
in the world of proprietary software.

Open source empowers you with the control and ownership of your own system.
Proprietary technologies are typically licensed in a way that gives you the
rights to \emph{use} the software \emph{subject} to be removed at \emph{any}
time they see fit and not \emph{own} the software you may of just payed for!
In actual fact, if you do not use the software as they specify in their
license agreement you are typically legally liable of braking sometimes
multiple laws if you ever got caught.
