% Copyright © 2013 Edward O'Callaghan. All Rights Reserved.

\section{Introduction} % (fold)
\label{sec:introduction}

The $\lcal$ formalism was introduced in the 1930s by Akonzo Church
to provide a framework for the concept of effective computability.
In this course we build up the rudiments of $\lcal$. We study the
$\lcal$ mathematical formalism of computable functions. The
properties of $\lcal$ have provided both the theoretical and
rigorous foundations to functional programming.

In actual fact the whole of classical computation can be understood
though the $\lcal$ formalism of the classical Turning machine. This
is known as the Church-Turing thesis that sates the follow:

\begin{conj}[Church-Turing thesis]
A function is computable if and only if it is computable by a
Turing machine.
\end{conj}

Hence, if some algorithm exists to carry out some calculation
then so too can a Turing machine compute the result and hence
a recursively definable function and by a $\lambda$-function.

In this way $\lcal$ can be thought of as the
\emph{smallest universal programming language}. To justify this
claim we remark that $\lcal$ only consists of of \emph{single}
transformation rule called \textbf{variable substitution} and
a single function definition scheme. We also make note that
the transformation rule does not refer to the underlaying
hardware implementation and so is independent of the actual
machine implementation and so is a pure software formalisation.

The key concept that motivates this is central to mathematics,
and more generally to language in general, is that of
\emph{representation}. Mathematical computation is essentially
the act of symbolic re-representation based on self consistent
pattern matching. The $\lcal$ is a natural formalisation to this
concept and so this provides useful motivating terms to think in
to familiarise yourself with the $\lcal$.
