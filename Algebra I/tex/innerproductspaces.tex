% Copyright © 2012 Edward O'Callaghan. All Rights Reserved.

\section{Inner Product Spaces} % (fold)
\label{sec:innerproductspaces}

Let $V$ be a complex vector space.

\begin{defn}[Inner product]
	An $\emph{inner product}$ in $V$ is a mapping $\langle \cdot , \cdot \rangle : V \times V \to \C$
	satisfying
	\begin{itemize}
		\item Hermitian symmetry: $\langle x,y \rangle = \overline{\langle y,x \rangle} \, \forall x,y \in V$;
		\item Linearity:
			$\langle \alpha x_1 + \beta x_2,y \rangle = \alpha \langle x_1,y \rangle + \beta \langle x_2,y \rangle$
			$\forall x_1,x_2,y \in V$ and $\alpha , \beta \in \C$;
		\item Positive definite: $\langle x,x \rangle \geq 0$ and $\langle x,x \rangle =0$ iff $x=0, \, \forall x \in V$;
	\end{itemize}
\end{defn}

\begin{rem}
	Note that condition (i) implies that $\langle x,x \rangle \in \R$ and so condition (ii) is consistent.
\end{rem}

\begin{rem}
	Note that condition (ii) defines that the inner product is linear in the first argument. However,
	the inner product is $\emph{conjugate linear}$ in the second argument as shown by:
	\begin{proof}
		\begin{align*}
			\langle x, \alpha y_1 + \beta y_2 \rangle
			&= \overline{\langle \alpha y_1 + \beta y_2, x \rangle}
			\\
			&= \overline{\langle \alpha y_1, x \rangle + \langle \beta y_2, x \rangle}
			\\
			&= \bar{\alpha} \overline{\langle y_1, x \rangle} + \bar{\beta} \overline{\langle y_2, x \rangle}
			\\
			&= \bar{\alpha} \langle x, y_1 \rangle + \bar{\beta} \langle x, y_2 \rangle . \qedhere
		\end{align*}
	\end{proof}
\end{rem}

A vector space $V$ together with an inner product $\langle \cdot, \cdot \rangle$ is termed an
$\emph{inner product space}$.

\begin{exmp}
	Consider the vector space $V=M_{mn}(\F)$. We wish to show that, for some $X,Y \in V$, that
	$tr(X^{*} Y)$ defines an inner product in $V$.
	Hence we have, $\langle X, Y \rangle \doteq tr(X^{*} Y)$ for all $X,Y \in V$.
	\begin{proof}
		First suppose that $\langle X, Y \rangle = tr(X^{*}Y)$ and so;
		\begin{align*}
			\intertext{We check that $tr(X^{*}Y)$ is Hermitian symmetric:}
			\langle X,Y \rangle &= tr(X^{*}Y)
			\\
			&= \overline{tr(\overline{X^{*}Y})}
			\\
			&= \overline{tr(\overline{YX^{*}})}
			\\
			&= \overline{tr( \overline{(X^{*}Y)}^{T} )}
			\\
			&= \overline{tr(Y^{*}X)}
			\\
			&= \overline{\langle Y,X \rangle}.
			\intertext{Check that $tr(X^{*} Y)$
		is linear in the first parameters by considering some $\alpha,\beta \in \F$ so that}
		\langle \alpha X_1 + \beta X_2, Y \rangle &= tr( (\alpha X_1 + \beta X_2)^{*} Y )
		\\
		&= tr( (\alpha X_1)^{*} Y + (\beta X_2)^{*} Y )
		\\
		&= tr( (\alpha X_1)^{*} Y ) + tr( (\beta X_2)^{*} Y )
		\\
		&= \bar{\alpha} tr(X^{*}_1 Y) + \bar{\beta} tr(X^{*}_2 Y)
		\\
		&= \bar{\alpha} \langle X_1, Y \rangle + \bar{\beta} \langle X_2, Y \rangle .
		\intertext{and check that $tr(X^{*} Y)$ is positive definite for $X=Y$,}
		\langle X, X \rangle &= tr( X^{*} X ) \geq 0
		\intertext{and when}
		X = 0 & \implies X^{*} = 0
		\\
		& \implies tr( X^{*} X ) = 0. \qedhere
		\end{align*}
	\end{proof}
\end{exmp}

TODO..

\begin{exmp}
	Let $\mathcal{V}$ be the vector space of continuous real-valued functions $[0,1]$ endowed
	with the standard inner product. If $f(x)=x$ and $g(x)=e^x$ then find $\langle f, g \rangle$
	and $\|f\|$.
	\begin{align*}
		\intertext{For $\langle f,g \rangle$ we have,}
		\langle f, g \rangle &= \int_0^1 f(x) \overline{g(x)} \mathrm{d}x
		\\
		&= \int_0^1 x e^x \mathrm{d}x
		\tag{by parts we have,}
		\\
		&= xe^x \|_0^1 - \int_0^1 e^x \mathrm{d}x
		\\
		&= e - (e - 1) = 1.
		\intertext{Now, for $\|f\|$ we have,}
		\|f\| &= \left( \int_0^1 x^2 \mathrm{d}x \right)^{\frac{1}{2}}
		\\
		&= \sqrt{ \frac{1}{x^3} } |_0^1
		\\
		&= \frac{1}{\sqrt{3}}.
	\end{align*}
\end{exmp}

\begin{thm}
	Let $\mathcal{V}$ be a complex vector space with inner product $\langle \cdot, \cdot \rangle$.
	Then the inner product induces a norm $\| \cdot \|$ in $\mathcal{V}$ by the definition
	\[
		\| x \| = \sqrt{ \langle x , x \rangle } \text{ for } x \in \mathcal{V}.
	\]
	\begin{proof}
		\begin{align*}
			\intertext{Notice that $\| \cdot \|$ is trivially positive definite as follows:}
			\| x \| &= \sqrt{ \langle x, x \rangle } \geq 0 \Leftrightarrow x=0, \, \forall x \in \mathcal{V}.
			\intertext{By linearity we see that,}
			\| \alpha x \| &= \sqrt{ \langle \alpha x, \alpha x \rangle }
			\\
			&= \sqrt{ \alpha \bar{\alpha} \langle x, x \rangle }
			\\
			&= \sqrt{ | \alpha |^2 \langle x, x \rangle } = | \alpha | \| x \|
			\, \forall \alpha \in \C \text{ and } \forall x \in \mathcal{V}.
			\intertext{Finally the triangle inequality follows as shown,}
			\| x + y \|^2 &= \langle x+y , x+y \rangle
			\\
			&= \| x \|^2 + \| y \|^2 + \langle x,y \rangle + \langle y,x \rangle
			\\
			&= \| x \|^2 + \| y \|^2 + \langle x,y \rangle + \overline{\langle x,y \rangle}
			\\
			&= \| x \|^2 + \| y \|^2 + 2 \R [\langle x,y \rangle]
			\\
			& \leq \| x \|^2 + \| y \|^2 + 2 | \langle x,y \rangle |
			\\
			& \leq \| x \|^2 + \| y \|^2 + 2 \| x \| \| y \|
			\\
			&= \left( \| x \| + \| y \| \right)^2 \, \forall x,y \in \mathcal{V}. \qedhere
		\end{align*}
	\end{proof}
\end{thm}

\begin{defn}[Isometric]
	Consider two inner product spaces $\mathcal{U}$ and $\mathcal{V}$ with the map
	$T: \mathcal{U} \to \mathcal{V}$. Then the map $T$ preserves inner products ($\emph{isometric}$)
	if, for any $\mathbf{u},\mathbf{u}' \in \mathcal{U}$, we have
	$\langle T(\mathbf{u}), T(\mathbf{u}') \rangle = \langle \mathbf{u}, \mathbf{u}' \rangle$.
\end{defn}

\begin{rem}
	A map need not be linear for it to be isometric.
\end{rem}

\begin{thm}
	Isometric maps are monomorphic.
\end{thm}

\begin{proof}
	Suppose some inner product preserving maps $T: \mathcal{U} \to \mathcal{V}$. Then it is
	sufficient to show that the dimension of the kernel of $T$ is zero for injectivity.
	\begin{align*}
		\intertext{Pick some $\mathbf{u} \in \ker T$ so that,}
		\langle T(\mathbf{u}), T(\mathbf{u}) \rangle &= \langle 0, 0 \rangle
		\\
		&= 0.
		\intertext{However, since $T$ preserves inner products then}
		\langle \mathbf{u},\mathbf{u} \rangle &= \langle T(\mathbf{u}), T(\mathbf{u}) \rangle
		\intertext{and so,}
		\langle \mathbf{u},\mathbf{u} \rangle &= 0
		\\
		\implies \mathbf{u} &= \mathbf{0}
		\intertext{hence $T$ is monomorphic.}
	\end{align*}
\end{proof}

TODO.. MOVE ME..

\begin{exmp}
	Find a basis for the orthogonal complement of
	$W=span \left\{
		\begin{pmatrix} 1\\1\\1\\0 \end{pmatrix}
		,
		\begin{pmatrix} 2\\2\\1\\1 \end{pmatrix}
	\right\} \in \R^4$.
	Therefore we do the following:
	\begin{align*}
		W^{\perp} &= \ker
		\begin{pmatrix}
			1 & 1 & 1 & 0 \\
			2 & 2 & 1 & 1
		\end{pmatrix}
		\\
		&= \ker
		\begin{pmatrix}
			1 & 1 & 1 & 0 \\
			0 & 0 & -1 & 1
		\end{pmatrix}
		\tag{after a row reduction}
		\\
		&= span \left\{
			\begin{pmatrix} 0\\-1\\1\\1 \end{pmatrix}
			,
			\begin{pmatrix} -1\\0\\1\\1 \end{pmatrix}
		\right\}.
	\end{align*}
\end{exmp}
