% Copyright © 2013 Edward O'Callaghan. All Rights Reserved.

\section{Quadratic Form} % (fold)
\label{sec:quadraticform}

A quadratic or more generally, a \emph{quadric}, form is the
linear combination of quadratic terms with scalar coefficients
with possible cross terms. More precisely, quadratic forms are
in fact homogeneous quadratic polynomials in $n$ variables.

The special case from the quadratic equation,
\[
 a x^2 + b x + c = 0
\]
where $a \in \R : a \neq 0$, $x \in \F$ and fix $b=c=0$ so
to be homogeneous, then it remains that we have the unary
quadratic function,
\[
 q(x) = a x^2
\]

This is a primary example of a simple one dimensional
quadratic form. It however, turns out more generally that a
\emph{quadric} has a more generalised linear algebraic form
known as the \emph{quadratic form} which is defined herein.

\begin{defn}[Quadratic Form]
For an $n \times 1$ vector $\mathbf{x}$ and $n \times n$
symmetric matrix $A$ the \emph{quadratic form} is defined as:
\[
 Q(\mathbf{x}) = \mathbf{x}^{T} A \mathbf{x}.
\]
\end{defn}

Notice that in the special case of the unary quadratic function
the $x$ term is a scalar in the field $\F$. However in the
more generalised form $\mathbf{x}$ is now a vector with
components in some field $\F$ where the vector space is
defined on, typically $\F = \R$. Observe that, if the
quadratic form $Q(\mathbf{x})$ is defined over some vector space
$\mathcal{V}(\F)$ then the quadratic form defines a mapping in
following way, $Q : \mathcal{V} \to \F$.
Such a quadratic mapping motivates the following definition:

\begin{defn}[Quadratic Space]
 A \emph{quadratic space} is the pairing $(\mathcal{V}, q)$
 of some vector space $\mathcal{V}(\F)$ over some field $\F$
 and the quadratic map $q : \mathcal{V} \to \F$ defined by
 the quadratic form $q = Q(\mathbf{v}) = \mathbf{v}^{T} A \mathbf{v}$
 for some symmetric matrix $A$ that is well defined for vectors
 from the vector space $\mathcal{V}$.
\end{defn}

\begin{exmp}
 Consider the vector space $\R^2$ and the quadratic map,
 $q : \R^2 \to \R$ defined by the square of the Euclidean norm.
 That is, $q(\mathbf{v}) = \| \mathbf{v} \|^2 = x^2 + y^2$.
 Hence a Euclidean normed vector space is a \emph{quadratic space}.
\end{exmp}

In any case, digressing back to the quadratic form consider for example,
some of the following motivating cases given,
\begin{exmp}
\begin{align*}
 q(x) &= \alpha x^2 \tag{unary}
 \\
 q(x,y) &= \alpha x^2 + \beta x y + \gamma y^2 \tag{binary}
 \\
 q(x,y,z) &= \alpha x^2 + \beta y^2 + \gamma z^2 + a xy + b xz + c yz \tag{ternary}
\end{align*}
\end{exmp}
By re-representing these forms in terms of the more compact matrix
representations we may make use of various linear algebraic tools
such as eigen-decomposition.

\begin{rem}
 The quadratic form is analogous to the notion of \emph{completing the square}.
 By way of this, we can remember the quadratic form in the geometric context as
 completing the hypersquares.
\end{rem}

Quadratic forms allow us to characterise various geometric curves and surfaces.
In particular,
\begin{align*}
 \left( \frac{x}{a} \right)^2 + \left( \frac{y}{b} \right)^2 &= c^2 \tag{eclipse}
 \\
 \left( \frac{x}{a} \right)^2 - \left( \frac{y}{b} \right)^2 &= c^2 \tag{hyperbola}
\end{align*}

Consider then an examples of a quadric form,
\begin{exmp}
 $Q(x,y) = x^2 + 2 \sqrt{2} xy + 4y^2$.
\end{exmp}
It is not yet apparent as to what this surface looks like geometrically. However,
by a change of bases we may derive a more familiar representation to characterise
the surface. The change of bases we pick will be the normalised orthogonal eigenbasis.
The rational for this choose is that the eigenvectors are \emph{stretched only} by
a scalar amount (the corresponding eigenvalues), and so are invariant to rotation or
reflection under the curve. We are guarantied that the eigenvectors are orthogonal by
the spectral theorem when $A$ is symmetric and that all the eigenvalues are real.
Moreover, by normalising the eigenvectors we remain with a representation in a new
coordinate system that does not `interfere` with the `look`, if you like, of the curve.
This orthonormal basis is called the \emph{principle axes} given in the following theorem.

First recall that,
\begin{lem}
 If $A$ is a real symmetric matrix, then by the spectral theorem
 $A$ has all real eigenvalues and the eigenspaces of $A$ are
 mutually orthogonal and so $A$ is orthonormally diagonalisable.
\end{lem}

\begin{thm}[Principal Axis Theorem]
 Given the quadratic form,
 $Q(\mathbf{x}) = \mathbf{x}^{T} A \mathbf{x}$
 where $A$ is a real symmetric matrix. Then by the above lemma,
 we have that, in particular by Gram-Schmidt, $A$ is orthogonally
 diagonalisable into an orthonormal eigenbasis.

 Suppose then that
 $\{\lambda_1, \lambda_2, \dots, \lambda_n\}$
 are the eigenvalues of $A$ and
 $\{\mathbf{u}_1, \mathbf{u}_2, \dots, \mathbf{u}_n\}$
 the eigenvectors of $A$. Then let
 $\{\hat{\mathbf{u}}_1, \hat{\mathbf{u}}_2, \dots, \hat{\mathbf{u}}_n\}$
 denote the set of normalised eigenvectors of $A$ and let $S$ be the
 matrix of the orthogonal eigenvectors normalised,
 \begin{align*}
  S &=
  \begin{pmatrix}
  \frac{\mathbf{u}_1}{\| \mathbf{u}_1 \|}
  &
  \frac{\mathbf{u}_2}{\| \mathbf{u}_2 \|}
  &
  \dots
  &
  \frac{\mathbf{u}_n}{\| \mathbf{u}_n \|}
  \end{pmatrix}
  \\
  \Rightarrow
  S &=
  \begin{pmatrix}
  \hat{\mathbf{u}}_1
  &
  \hat{\mathbf{u}}_2
  &
  \dots
  &
  \hat{\mathbf{u}}_n
  \end{pmatrix}.
 \end{align*}

 Then with the substitution $\mathbf{x} = S \mathbf{y}$
 we have the \emph{principle axes form}:
 \[
 Q(\mathbf{x})
 = \mathbf{y}^{T} \Lambda \mathbf{y}
 = \sum_{i=1}^{n} \lambda_i y_{i}^{2}.
 \]
 The set of normalised eigenvectors,
 $\{\hat{\mathbf{u}}_1, \hat{\mathbf{u}}_2, \dots, \hat{\mathbf{u}}_n\}$
 is said to be the \emph{principle axes} of the quadratic form $Q(\mathbf{x})$.
\end{thm}

\begin{proof}
  Consider the quadratic form, $Q(\mathbf{x}) = \mathbf{x}^{T} A \mathbf{x}$
  and suppose that the matrix $A$ is real and symmetric, i.e. $A=A^{T}$.
  Suppose then that the set $\{\lambda_1, \lambda_2, \dots, \lambda_n\}$
  are the eigenvalues of $A$ and $\{\mathbf{u}_1, \mathbf{u}_2, \dots, \mathbf{u}_n\}$
  are the eigenvectors of $A$. Then let,
  \begin{align*}
   \Lambda &=
   \begin{pmatrix}
   \lambda_1 & 0 & \dots & 0 \\
   0 & \lambda_2 & & \vdots \\
   \vdots & & \ddots & 0 \\
   0 & \dots & 0 & \lambda_n
   \end{pmatrix}
  \end{align*}
  be the matrix with eigenvalues down the diagonal as so that we may diagonalise
  $A$ by the change of basis $A= \mathcal{B} \Lambda \mathcal{B}^{-1}$
  where $\mathcal{B}$ is the matrix,
  \begin{align*}
   \mathcal{B} &=
   \begin{pmatrix}
    \mathbf{u}_1 & \mathbf{u}_2 & \dots & \mathbf{u}_n
   \end{pmatrix}
  \end{align*}
  of eigenvectors.
  Let us now write a matrix $S$ of the orthogonal eigenvectors normalised,
  \begin{align*}
   S &=
   \begin{pmatrix}
   \frac{\mathbf{u}_1}{\| \mathbf{u}_1 \|}
   &
   \frac{\mathbf{u}_2}{\| \mathbf{u}_2 \|}
   &
   \dots
   &
   \frac{\mathbf{u}_n}{\| \mathbf{u}_n \|}
   \end{pmatrix}.
   \intertext{Hence we have that,}
   Q &= \mathbf{x}^{T} A \mathbf{x}
   \\
   &= \mathbf{x}^{T} (S \Lambda S^{-1}) \mathbf{x}
   \\
   &= \mathbf{x}^{T} (S \Lambda S^{T}) \mathbf{x}
   \\
   &= (S^{T} \mathbf{x})^{T} \Lambda (S^{T} \mathbf{x}).
   \intertext{It remains that,}
   Q &= (S^{T} \mathbf{x})^{T} \Lambda (S^{T} \mathbf{x}).
   \intertext{Consider now the substitution $\mathbf{x} = S \mathbf{y}$}
   \Rightarrow Q &= (S^{T} (S \mathbf{y}))^{T} \Lambda (S^{T} (S \mathbf{y}))
   \\
   &= \mathbf{y}^{T} (S^{T} S) \Lambda (S^{T} S) \mathbf{y}
   \\
   &= \mathbf{y}^{T} \Lambda \mathbf{y}
   \\
   &= \sum_{i=1}^{n} \lambda_i y_{i}^{2}. \qedhere
  \end{align*}
\end{proof}

\begin{lem}
 For the quadratic form $Q(\mathbf{x}) = \mathbf{x}^{T} A \mathbf{x}$, if
 all the eigenvalues of $A$ are strictly positive then the quadric $Q(\mathbf{x})$
 is an \emph{ellipsoid}. If all the eigenvalues are strictly negative then the quadric
 is an \emph{imaginary ellipsoid}. If some eigenvalues are negative and some positive
 then the quadric is said to be an \emph{hyperboloid}. Finally, if there exists at least
 one eigenvalues of scale zero then the quadric is some kind of \emph{paraboloid} to be
 determined.
\end{lem}

\begin{exmp}
 Consider again the quadric form, $Q(x,y) = x^2 + 2 \sqrt{2} xy + 4y^2$ and rewrite this
 in the matrix quadratic form in the following way,
 \begin{align*}
  Q(x,y) = Q(\mathbf{x}) &= \mathbf{x}^{T} A \mathbf{x}
  \\
  &= \mathbf{x}^{T}
  \begin{pmatrix}
   1 & \sqrt{2} \\
   \sqrt{2} & 4
  \end{pmatrix}
  \mathbf{x} \tag{division by two much like completing the square.}
  \end{align*}
  Now we wish to diagonalise the matrix $A$ and so we find the eigenvectors
  and corresponding eigenvalues of $A$ in the usual way. By direct computation
  we see that $A$ has eigenvalues $\lambda_1 = 5$ and $\lambda_2 = 2$. Hence,
  by the principal axes theorem we have that,
  \begin{align*}
   Q(\mathbf{y}) &= 5 y_1^{2} + 2 y_2^{2}.
  \end{align*}

  We may wish to sketch the level curve when $Q=1$.
  Consider then the curve $5 y_1^{2} + 2 y_2^{2} = 1$ and observe that this
  is an \emph{ellipse} that intersects with the $y_1$-axis at $-1/\sqrt{5}$
  and $1/\sqrt{5}$ and intersects with the $y_2$ axis at $1/\sqrt{2}$ and
  $-1/\sqrt{2}$.

  By direct computation we find the eigenvectors are,
  \begin{align*}
   \mathbf{v}_1 &= \begin{pmatrix} \sqrt{2} \\ -1 \end{pmatrix},
   \\
   \mathbf{v}_2 &= \begin{pmatrix} 1 \\ \sqrt{2} \end{pmatrix}.
  \end{align*}
  Recall now the substitution $\mathbf{x} = S \mathbf{y}$ and set the matrix
  $S$ as the orthogonal eigenvectors normalised in the following way,
  \begin{align*}
   S &= \frac{1}{\sqrt{3}}
   \begin{pmatrix}
    \sqrt{2} & 1 \\
    -1 & \sqrt{2}
   \end{pmatrix}.
  \end{align*}
  Where $\mathbf{y} = \begin{pmatrix} y_1 \\ y_2 \end{pmatrix}$ and
  $\mathbf{x} = \begin{pmatrix} x \\ y \end{pmatrix}$. Hence, if we then
  plot the normalised eigenbasis on the $XY$-plane and then plot the ellipse
  on the normalised eigenbasis we may observe that the original quadratic form of
  $Q(x,y) = x^2 + 2 \sqrt{2} xy + 4y^2$ is merely a rotated ellipse in the usual XY-plane.
  In particular, the matrix $S$ is a rotation matrix of the ellipse on the principal axes
  (normalised eigenbasis) into the $XY$-plane.

  Finally we can find the closes points to the origin by observing that in the eigen
  coordinate system, that is the $Y_1 Y_2$-plane, we have the closes points as
  $\pm \frac{1}{\sqrt{5}}$.
\end{exmp}

\begin{exmp}
Given $3x^2 - 10xy + 3y^2 = 24$ find the shortest distance
to the origin.

\begin{align*}
\intertext{Consider the quadratic form, $Q=\mathbf{x}^{T} A \mathbf{x}$, and let}
A &=
\begin{pmatrix}
3 & -5 \\
-5 & 3
\end{pmatrix}
\intertext{which has eigenvalues $\lambda_1 = 8$ and $\lambda_2 = -2$ with eigenvectors,}
\mathbf{v}_1 &= \begin{pmatrix} 1 \\ -1 \end{pmatrix},
\\
\mathbf{v}_2 &= \begin{pmatrix} 1 \\ 1 \end{pmatrix}.
\intertext{Let us know write a matrix $S$ of the othogonal eigenvectors normalised,}
S &= \frac{1}{\sqrt{2}} \begin{pmatrix} \mathbf{v}_1 & \mathbf{v}_2 \end{pmatrix}
\intertext{and matrix $\Lambda$ with eigenvalues down the diagonal as so,}
\Lambda &= \begin{pmatrix} \lambda_1 & 0 \\ 0 & \lambda_2 \end{pmatrix}.
\intertext{Then by the principal axes theorem we have that,}
Q = \mathbf{y}^{T} \Lambda \mathbf{y} &= \sum_{i=1}^{n} \lambda_i y_{i}^{2}
\\
&= 8 y_1^2 -2 y_2^2.
\intertext{Consider again the substitution $\mathbf{x}=S \mathbf{y}$ then it remains that,}
Q &= (S^{T} \mathbf{x})^{T} \Lambda (S^{T} \mathbf{x})
\\
&= \frac{1}{2} \begin{pmatrix} (x-y) & (x+y) \end{pmatrix}
\begin{pmatrix} 8 & 0 \\ 0 & -2 \end{pmatrix}
\begin{pmatrix} (x-y) \\ (x+y) \end{pmatrix}
\\
&= 4 (x-y)^2 - 1 (x+y)^2.
\intertext{Thus we have the hyperboloid $4(x-y)^2 - 1(x+y)^2 = 24$ with the shortest distance to the orgin $\sqrt{3}$.}
\end{align*}
\end{exmp}
