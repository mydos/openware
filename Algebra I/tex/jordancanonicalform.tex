% Copyright © 2012 Edward O'Callaghan. All Rights Reserved.

\section{Jordan Canonical Form} % (fold)
\label{sec:JordanForm}

The Jordan (normal) canonical form is a set of "almost diagonal"
matries, called the $\emph{Jordan blocks}$. The Jordan blocks
contain the diagonal components composed of eigenvalues. We call
this form, the $\emph{block diagonal form}$. The set of Jordan
blocks form a $\emph{equivalence class}$ under similarity invariance
of the eigen-invariant subspaces.

\begin{defn}[Jordan block]
	A Jordan block $\mathcal{J}_k(\lambda)$ is a $k \times k$
	$\emph{upper triangular}$ square matrix of the form:
	\[
		\mathcal{J}_k(\lambda) =
		\begin{pmatrix}
			\lambda & 1 & & & \\
			0 & \lambda & 1 & & \text{\huge{0}} \\
			& & \ddots & \ddots & \\
			\text{\huge{0}} & & & \ddots & 1 \\
			& & & & \lambda
		\end{pmatrix}
	\]
	where there are $k-1$ (+1)'s in the $\emph{superdiagonal}$ and
	$k$ of a particular scalar eigenvalue $\lambda$ in the main
	diagonal while all other entries are zero. The trivial $1 \times 1$
	Jordan block of $\lambda$ is $\mathcal{J}_1(\lambda) = ( \lambda )$.
\end{defn}

\begin{defn}[Jordan matrix]
	A Jordan matrix $\mathcal{J} \in \mathcal{M}_{n,n}(\F)$ is a
	direct sum (\ref{sec:directsum}) of Jordan blocks. That is,
	\[
		\mathcal{J} =
		\bigoplus_{i=1}^{N} \mathcal{J}_{\lambda_i , m_k} : 1 \leq i \leq N.
	\]
	where each $\lambda_i \in \sigma(A)$ for some matrix $A \in \mathcal{M}(\F)$
	need not be distinct.
\end{defn}

\begin{exmp}
	\begin{align*}
		\intertext{Suppose,}
		A &=
		\begin{pmatrix}
			3 & 1 & 1 \\
			-4 & -1 & 0 \\
			0 & 0 & 1
		\end{pmatrix}.
		\intertext{We wish to find the Jordan canonical form so that we have the
			usual change of basis given by, $C \mathcal{J} = A C$. First we find
		the characteristic polynomial,}
		\chi_{A}(\lambda) &= (3 - \lambda) (\lambda + 1)^2 - 4 (\lambda + 1)
		\\
		&= -(\lambda + 1) \left\{ (\lambda + 1) (\lambda - 3) + 4 \right\}
		\\
		&= -(\lambda + 1) (\lambda^2 - 2 \lambda + 1)
		\\
		\Rightarrow \chi_{A}(\lambda) &= (\lambda + 1) (\lambda - 1)^2
		\intertext{and so, $\lambda_1 = -1$ and $\lambda_2 = 1$ with algebraic
		multiplicity two. Next recall that,}
		\mathcal{E}_{\lambda}(n) &= \ker (A - \lambda I)^n
		\intertext{and, in particular, that}
		\text{geometric multiplicity} &= \dim \mathcal{E}_{\lambda}(1).
		\intertext{For $\lambda_1$ we recall that the geometric multiplicity is
			always less than or equal to the algebraic multiplicity and since
			$\lambda_1$ has algebraic multiplicity of one it has geometric
		multiplicity of one, hence has the trivial Jordan block:}
		\mathcal{J}_1(\lambda_1) &= \mathcal{J}_1(-1).
		\intertext{To find the corresponding eigenvector to $\lambda_1$ we look
		in the span of the kernel as follows}
		\mathbf{v}_1 \in \mathcal{E}_{\lambda_1}(1) &= \mathcal{E}_{-1}(1)
		\\
		&= \ker (A + I)^1
		\\
		&= \ker
		\begin{pmatrix}
			4 & 1 & 1 \\
			-4 & 0 & 0 \\
			0 & 0 & 0
		\end{pmatrix}
		\\
		&= \ker
		\begin{pmatrix}
			0 & 1 & 1 \\
			1 & 0 & 0 \\
			0 & 0 & 0
		\end{pmatrix} \tag{after a row reduction}
		\\
		\Rightarrow \mathbf{v}_1 & \in span \left\{ \begin{pmatrix} 0 \\ 1 \\ -1 \end{pmatrix} \right\}.
		\intertext{Note that it is important we find $\mathbf{v}_1$ first as we shall see. Now, see that}
		\mathcal{E}_{\lambda_2}(1) = \mathcal{E}_{1}(1) &= \ker (A - I)^1
		\\
		&= \ker
		\begin{pmatrix}
			2 & 1 & 1 \\
			-4 & -2 & 0 \\
			0 & 0 & -2
		\end{pmatrix}
		\\
		&= \ker
		\begin{pmatrix}
			2 & 1 & 1 \\
			-2 & -1 & 0 \\
			0 & 0 & -1
		\end{pmatrix}
		\\
		&= \ker
		\begin{pmatrix}
			0 & 0 & 0 \\
			2 & 1 & 0 \\
			0 & 0 & 1
		\end{pmatrix} \tag{after row reductions.}
		\intertext{Hence, $\dim \mathcal{E}_{\lambda_2}(1) = 2$ and so we have the corresponding Jordan block,}
		\mathcal{J}_2 (\lambda_2) &= \mathcal{J}_2(1).
		\intertext{Now, to find the corresponding eigenvectors to $\lambda_2$ we need only pick a vector
		$\textbf{not}$ in the span of $\mathcal{E}_{\lambda_2}(1)$ and is linearly independent from $\mathbf{v}_1$.
		Consider the vector $\mathbf{v}_2 \notin \mathcal{E}_{\lambda_2}(1)$ with,}
		\mathbf{v}_2 &= \begin{pmatrix} 0 \\ 1 \\ 0 \end{pmatrix}.
		\intertext{Then by the Jordan chain we see that,}
		\mathbf{v}_3 &= \mathcal{E}_{\lambda_2}(1) \mathbf{v}_2
		\\
		&= \begin{pmatrix} 1 \\ -2 \\ 0 \end{pmatrix} \in span \mathcal{E}_{\lambda_2}(1)
		\intertext{thus, $\mathcal{E}_{\lambda_2}(1) \mathbf{v}_3 = \mathbf{0}$ and since the geometric multiplicity
		of $\lambda_2$ was two we now have the spanning set of corresponding eigenvectors $\mathbf{v}_2,\mathbf{v}_3$. So,}
		C \mathcal{J} &= A C
		\\
		\Rightarrow A &= C \mathcal{J} C^{-1}
		\\
		\Rightarrow A &= C \mathcal{J}_{1}(\lambda_1) \oplus \mathcal{J}_{2}(\lambda_2) C^{-1}
		\\
		A &=
		\begin{pmatrix}
			0 & 0 & 1 \\
			1 & 1 & -2 \\
			-1 & 0 & 0
		\end{pmatrix}
		\begin{pmatrix}
			-1 & 0 & 0 \\
			0 & 1 & 1 \\
			0 & 0 & 1
		\end{pmatrix}
		\begin{pmatrix}
			0 & 0 & 1 \\
			1 & 1 & -2 \\
			-1 & 0 & 0
		\end{pmatrix}^{-1}.
	\end{align*}
\end{exmp}

