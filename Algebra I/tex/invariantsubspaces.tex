% Copyright © 2012 Edward O'Callaghan. All Rights Reserved.

\section{Invariant Subspaces} % (fold)
\label{sec:invariantsubspace}

\begin{defn}[Characteristic polynomial]
	Consider some matrix $A \in \mathcal{M}_{n,n}(\F)$ with
	the eigenvector $\mathbf{x}$ and eigenvalue $\lambda \in \F$ relation,
	\begin{align*}
		A \mathbf{x} &= \lambda \mathbf{x} : \mathbf{x} \neq \mathbf{0}
		\intertext{then we have,}
		A \mathbf{x} - \lambda \mathbf{x} &= \mathbf{0}
		\\
		\Rightarrow
		( A - \lambda I_n ) \mathbf{x} &= \mathbf{0}.
		\intertext{Hence, the characteristic polynomial is given by}
		\chi_{A} (\lambda) &= \left| A - \lambda I_n \right|.
	\end{align*}
\end{defn}

\begin{thm}[Clayley-Hamilton]
	For any matrix $A \in \mathcal{M}_{n,n}(\F)$ with
	characteristic polynomial $\chi_{A} (\lambda) : \lambda \in \F$.
	Then $A$ satisfies $\chi_{A} (\lambda)$, that is, $\chi_{A} (A) = 0$.
\end{thm}

\begin{exmp}
	Suppose some matrix $A \in \mathcal{M}_{n,n}(\F)$ is $\emph{idempotent}$,
	that is $A^2 = A$. Then show that each eigenvalue of $A$ is either zero or one.
	\begin{proof}
		\begin{align*}
			\intertext{Suppose that,}
			A \mathbf{x} &= \lambda \mathbf{x} : \mathbf{x} \neq \mathbf{0}
			\intertext{by induction we see that,}
			A^2 \mathbf{x} &= \lambda^2 \mathbf{x}
			\\
			\Rightarrow A \mathbf{x} &= \lambda^2 \mathbf{x}
			\\
			&= \lambda \mathbf{x}
			\\
			\Rightarrow \lambda &= \lambda^2 \mbox{ iff } \lambda \in \{0,1\}. \qedhere
		\end{align*}
	\end{proof}
\end{exmp}

\begin{exmp}
	Suppose some matrix $A \in \mathcal{M}_{n,n}(\F)$ is $\emph{nilpotent}$,
	that is $A^k = \mathbf{0}$ for some positive integer $k$. Then show that all
	the eigenvalues of $A$ are zero.
	\begin{proof}
		\begin{align*}
			\intertext{Suppose that,}
			A \mathbf{x} &= \lambda \mathbf{x} : \mathbf{x} \neq \mathbf{0}
			\intertext{by induction we see that,}
			A^k \mathbf{x} &= \lambda^k \mathbf{x}
			\\
			\Rightarrow \mathbf{0} &= \lambda^k \mathbf{x}
			\\
			\Rightarrow \lambda^k &= 0
			\\
			\text{and so, } \lambda &= 0 \text{ for every } \lambda \in \F. \qedhere
		\end{align*}
	\end{proof}
\end{exmp}

\begin{defn}[Eigen Subspace]
	For some matrix $A \in \mathcal{M}_{n,n}$.
	Define a eigen-subspace $\mathcal{E}_{\lambda}$ for every eigenvalue $\lambda$ as
	the space that contains all the eigenvectors for that particular eigenvalue. That is,
	\begin{align*}
		\mathcal{E}_{\lambda}(k) &= \ker (A - \lambda I_n)^k.
	\end{align*}
\end{defn}

\begin{defn}[Spectrum]
	The $\emph{spectrum}$ of a linear operator $A \in \mathcal{M}_{n,n}(\C)$
	is the set of all its eigenvalues, denoted by $\sigma(A)$. That is,
	\begin{align*}
		\sigma(A) &=
		\{ \lambda \in \C : A \mathbf{x} = \lambda \mathbf{x} : \mathbf{x} \neq \mathbf{0} \}.
	\end{align*}
\end{defn}

\begin{defn}[Spectral radius]
	The $\emph{spectral radius}$ pf a linear operator $A \in \mathcal{M}_{n,n}(\C)$
	is the radius of the smallest disc centered at the origin in the complex plane
	that includes all the eigenvalues of $A$, denoted by $\rho(A)$. That is,
	\begin{align*}
		\rho(A) &= \max \{ | \lambda | : \lambda \in \sigma(A) \}.
	\end{align*}
\end{defn}

\begin{defn}[T-invariant subspace]
	For a vector space $\mathcal{V}$ consider the linear endomorphism
	$T: \mathcal{V} \to \mathcal{V}$. Then, for some subspace $\mathcal{W} \leq \mathcal{V}$,
	$\mathcal{W}$ is said to be $T$-invariant if $T(\mathcal{W}) \subseteq \mathcal{W}$.
\end{defn}

\begin{exmp}
	Show that,
	\begin{align*}
		W=span \left\{
			\begin{pmatrix}
				1\\1\\0\\1
			\end{pmatrix}
			,
			\begin{pmatrix}
				1\\0\\2\\0
			\end{pmatrix}
		\right\}
		& \text{ and }
		W'=span \left\{
			\begin{pmatrix}
				1\\1\\0\\0
			\end{pmatrix}
			,
			\begin{pmatrix}
				1\\1\\-1\\0
			\end{pmatrix}
		\right\}
	\end{align*}
	are $T$-invariant subspaces of the linear map:
	\[
		T=
		\begin{pmatrix}
			4 & -2 & -1 & -1 \\
			3 & -1 & -1 & -1 \\
			-2 & 2 & 2 & 0 \\
			1 & -1 & 0 & 1
		\end{pmatrix}.
	\]
	Observe that,
	\begin{align*}
		T(W) &= span \left\{
			\begin{pmatrix}
				1\\1\\0\\1
			\end{pmatrix}
			,
			\begin{pmatrix}
				2\\1\\2\\1
			\end{pmatrix}
		\right\}
		\subseteq W
		\intertext{and that}
		T(W') &= span \left\{
			\begin{pmatrix}
				2\\2\\0\\0
			\end{pmatrix}
			,
			\begin{pmatrix}
				3\\3\\-2\\0
			\end{pmatrix}
		\right\}
		\subseteq W'
		\intertext{and so, both $W$ and $W'$ are $T$-invariant subspaces.}
	\end{align*}
\end{exmp}
