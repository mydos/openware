% Copyright © 2013 Edward O'Callaghan. All Rights Reserved.
% !Tex root = Algebra I.tex

\section{Unitary Operators} % (fold)
\label{sec:unitary}

A linear operator whose inverse is its adjoint is said to be
\emph{unitary}. Unitary operators can thus be thought of as
generalizations of complex numbers whose absolute value is
\emph{unity} and hence the name. We should this later by
considering the eigenvalues of such operators. However,
more precisely,

\begin{defn}[Unitary]
 A linear map $U : \mathcal{V} \to \mathcal{V}$ is said to be
 \emph{unitary} if $U^{*} U = I = U U^{*}$.
\end{defn}

We thus have the subgroup of the general linear group $GL$
called the \emph{unitary group} defined as,
\[
 U_n(\C) = \{ U \in GL_n(\C) : U^{*}U = I = UU^{*} \}.
\]

\begin{cor}
 If $U$ is unitary then we have that adjoint of $U$
 acts as its inverse $U^{*}=U^{-1}$.
\end{cor}

\begin{lem}
 Suppose that $U$ is unitary then $| \det(U) | = 1$.
\end{lem}

\begin{proof}
 Observe that,
 \begin{align*}
  1 &= \det(I)
  \\
  &= \det(U^{*} U)
  \\
  &= \det(\bar{U}^{T}) \det(U)
  \\
  &= \overline{\det(U^{T})} \det(U)
  \\
  &= \left| \det(U) \right|^{2}. \qedhere
 \end{align*}
\end{proof}

\begin{lem}
 Suppose that $U$ is unitary then modulus of the eigenvalues
 $| \lambda | = 1$ is unity.
\end{lem}

\begin{proof}
 By definition $U \mathbf{x} = \lambda \mathbf{x} : \mathbf{x} \neq \mathbf{0}$.
 Hence,
 \begin{align*}
  U \mathbf{x} &= \lambda \mathbf{x}
  \\
  U^{*} U \mathbf{x} &= \bar{\lambda} \lambda \mathbf{x}
  \\
  \Rightarrow I \mathbf{x} &= \bar{\lambda} \lambda \mathbf{x}
  \\
  \Rightarrow \bar{\lambda} \lambda &= 1. \qedhere
 \end{align*}
\end{proof}

\begin{prop}
 Let $\mathcal{H},\mathcal{K}$ be two Hilbert spaces and
 $U : \mathcal{H} \to \mathcal{K}$ be a unitary operator
 mapping between them. Then $U$ is a \emph{normal isometry}.
\end{prop}

\begin{proof}
 We are required to show that, for every $g,h \in \mathcal{H}$,
 \[
  \langle g, h \rangle_{\mathcal{H}}
  = \langle U(g), U(h) \rangle_{\mathcal{K}}.
 \]
 Hence,
 \begin{align*}
  \langle U(g), U(h) \rangle_{\mathcal{K}}
  &= (U h)^{*} (U g)
  \\
  &= h^{*} U^{*} U g
  \\
  &= h^{*} g = \langle g, h \rangle_{\mathcal{H}}. \qedhere
 \end{align*}
\end{proof}
