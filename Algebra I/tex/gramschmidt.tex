% Copyright © 2012 Edward O'Callaghan. All Rights Reserved.
\section{Gram-Schmidt} % (fold)
\label{sec:gram-schmidt}

The Gram-Schmidt algorithm is a method of orthonormalising
a finite set of linearly independent vectors that span an
inner product space into a orthonormal basis.

\begin{defn}[Projection operator]
	Consider two $n \times 1$ vectors $\mathbf{u},\mathbf{v} \in \F^n$.
	Then the \textbf{orthogonal projection}
	$\emph{of } \mathbf{v} \emph{ onto}$ the line spanned by $\mathbf{u}$
	is given by,
	\[
		proj_{\mathbf{u}} (\mathbf{v})
		= \frac{\langle \mathbf{v},\mathbf{u} \rangle}
		{\langle \mathbf{u}, \mathbf{u} \rangle} \mathbf{u}.
	\]
\end{defn}

Consider a finite sequence of vectors given by the spanning set
$S = \{ \mathbf{v}_1 , \dots , \mathbf{v}_k \}$. Then the
sequence of Gram-Schmidt orthogonalised vectors
$\{\mathbf{u}_1, \dots , \mathbf{u}_k \}$ is given by,
$\mathbf{u}_k = \mathbf{v}_k - \sum_{j=1}^{k-1} proj_{\mathbf{u}_k} (\mathbf{v}_k)$.

Hence, we may normalise the sequence by, $\mathbf{e}_k = \frac{\mathbf{u}_k}{\| \mathbf{u}_k \|}$.
