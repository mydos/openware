% Copyright © 2012 Edward O'Callaghan. All Rights Reserved.

\section{Determinants} % (fold)
\label{sec:determinants}

\begin{defn}[Determinant]
	\[
		\det(A) = \sum_{\sigma \in S_n} sgn(\sigma) \prod_{i=1}^{n} a_{i,\sigma(i)}.
	\]
\end{defn}

\begin{exmp}
	Find the determinant of some matrix $A \in \mathcal{M}_{2,2}(\F)$.
	\begin{align*}
		\intertext{Suppose}
		A &=
		\begin{pmatrix}
			a_{11} & a_{12} \\
			a_{21} & a_{22}
		\end{pmatrix}
		\intertext{and take the permutation group}
		S_2 &= \{ \sigma_1 , \sigma_2 \}
		\intertext{with $\sigma_1 = id$ and $\sigma_2 = [1 \, 2]$. Then, by definition, we have:}
		\det(A) &= \sum_{\sigma \in S_2} sgn(\sigma) \prod_{i=1}^{2} a_{i,\sigma(i)}
		\\
		&= sgn(\sigma_1)( a_{1,\sigma_1(1)} a_{2,\sigma_1(2)} )
		+ sgn(\sigma_2)( a_{1,\sigma_2(1)} a_{2,\sigma_2(2)} )
		\\
		&= +( a_{11} a_{22} ) - ( a_{12} a_{21} )
		\\
		&= a_{11} a_{22} - a_{12} a_{21}.
	\end{align*}
\end{exmp}

\begin{exmp}
	Find the determinant of some matrix $A \in \mathcal{M}_{3,3}(\F)$.
	\begin{align*}
		\intertext{Suppose}
		A &=
		\begin{pmatrix}
			a_{11} & a_{12} & a_{13} \\
			a_{21} & a_{22} & a_{23} \\
			a_{31} & a_{32} & a_{33}
		\end{pmatrix}
		\intertext{and take the permutation group}
		S_6 &= \{ \sigma_1 , \sigma_2 , \sigma_3 , \sigma_4 , \sigma_5 , \sigma_6 \}
		\intertext{with}
		\sigma_1 = [1 \, 2 \, 3] & \, \sigma_2 = [1 \, 3 \, 2]
		\\
		\sigma_3 = [2 \, 1 \, 3] & \, \sigma_4 = [2 \, 3 \, 1]
		\\
		\sigma_5 = [3 \, 2 \, 1] & \, \sigma_6 = [3 \, 1 \, 2]
		\intertext{Then, by definition, we have:}
		\det(A) &= \sum_{\sigma \in S_6} sgn(\sigma) \prod_{i=1}^{6} a_{i,\sigma(i)}
		\\
		&= sgn(\sigma_1)( a_{1,\sigma_1(1)} a_{2,\sigma_1(2)} a_{3,\sigma_1(3)}
		a_{4,\sigma_1(4)} a_{5,\sigma_1(5)} a_{6,\sigma_1(6)} )
		\\
		&+ sgn(\sigma_2)( a_{1,\sigma_2(1)} a_{2,\sigma_2(2)} a_{3,\sigma_2(3)}
		a_{4,\sigma_2(4)} a_{5,\sigma_2(5)} a_{6,\sigma_2(6)} )
		\\
		&+ sgn(\sigma_3)( a_{1,\sigma_3(1)} a_{2,\sigma_3(2)} a_{3,\sigma_3(3)}
		a_{4,\sigma_3(4)} a_{5,\sigma_3(5)} a_{6,\sigma_3(6)} )
		\\
		&+ sgn(\sigma_4)( a_{1,\sigma_4(1)} a_{2,\sigma_4(2)} a_{3,\sigma_4(3)}
		a_{4,\sigma_4(4)} a_{5,\sigma_4(5)} a_{6,\sigma_4(6)} )
		\\
		&+ sgn(\sigma_5)( a_{1,\sigma_5(1)} a_{2,\sigma_5(2)} a_{3,\sigma_5(3)}
		a_{4,\sigma_5(4)} a_{5,\sigma_5(5)} a_{6,\sigma_5(6)} )
		\\
		&+ sgn(\sigma_6)( a_{1,\sigma_6(1)} a_{2,\sigma_6(2)} a_{3,\sigma_6(3)}
		a_{4,\sigma_6(4)} a_{5,\sigma_6(5)} a_{6,\sigma_6(6)} )
	\end{align*}
\end{exmp}

\begin{lem}
	\[
		\det(A^{T}) = \det(A).
	\]
\end{lem}

\begin{proof}
	Consider some $\sigma$ such that $\sigma_1 , \dots , \sigma_n$
	is in order so that $\tau = \sigma^{-1}$ and take the matrix $A^i_j$
	so that $(A^i_j)^{T} = A^j_i$.
	\begin{align*}
		\intertext{Then we see,}
		A^{\sigma(i)}_j &= A^j_{\sigma^{-1}(j)} = A^j_{\tau(j)}
		\intertext{with $sgn(\tau) = sgn(\sigma)$. So,}
		\det(A) &= \sum_{\sigma \in S_n} sgn(\sigma) \prod_{i=1}^n a_{i,\sigma(i)}
		\\
		&= \sum_{\tau \in S_n} sgn(\tau) \prod_{i=1}^n a_{\tau(i), \tau \cdot \sigma(i)}
		\\
		&= \sum_{\tau \in S_n} sgn(\tau) \prod_{i=1}^n a_{\tau(i),i}
		\\
		&= \det(A^{T}). \qedhere
	\end{align*}
\end{proof}

\begin{lem}
	\[
		\det(AB) = \det(A) \det(B).
	\]
\end{lem}

\begin{proof}
	Let $A=[a^i_j]$ and $B=[b^i_j]$ with $A,B \in \mathcal{M}_{nn}(\F)$ so that
	$AB=\left[ \sum_{k=1}^n a^i_k b^k_j \right]$.
	\begin{align*}
		\det(AB)
		&= \sum_{\sigma \in S_n} sgn(\sigma) \prod_{i=1}^{n} \left[ \sum_{k=1}^{n} a^i_k b^k_{\sigma(i)} \right]
		\\
		&= ..
	\end{align*}
\end{proof}
