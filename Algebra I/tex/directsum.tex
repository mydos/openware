% Copyright © 2012 Edward O'Callaghan. All Rights Reserved.

\section{Direct Sum} % (fold)
\label{sec:directsum}

Consider two subspaces $\mathcal{U},\mathcal{V} \leq \mathcal{W}$
of $\mathcal{W}$. Then $\mathcal{U}$ and $\mathcal{V}$ are said to be
$\textbf{complementry}$ if every vector $w \in \mathcal{W}$ has a
$\emph{unique}$ decomposition $w=u+v : u \in \mathcal{U}$
and $v \in \mathcal{V}$. The vector space $\mathcal{W}$ is then said to
be the $\textbf{internal direct sum}$ of subspaces $\mathcal{U}$
and $\mathcal{V}$, written $\mathcal{W} = \mathcal{U} \oplus \mathcal{V}$.

\begin{defn}[Direct sum]
	Consider two vector spaces $\mathcal{U}$ and $\mathcal{V}$.
	Then the sum $\mathcal{U} + \mathcal{V}$ is called $\emph{direct}$
	if and only if $\mathcal{U} \cap \mathcal{V} = \{ \mathbf{0} \}$,
	written $\mathcal{U} \oplus \mathcal{V}$.
\end{defn}

\begin{defn}[Complementary subspaces]
	Both $\mathcal{U}$ and $\mathcal{V}$ are complementary subspaces
	of $\mathcal{W}$ if and only if,
	\begin{itemize}
		\item $\mathcal{W} = \mathcal{U} + \mathcal{V}$ and,
		\item $\mathcal{U} \cap \mathcal{V} = \{ \mathbf{0} \}$.
	\end{itemize}
	That is, if $\mathcal{U},\mathcal{V} \leq \mathcal{W}$ we have that
	$\mathcal{W} = \mathcal{U} \oplus \mathcal{V}$.
\end{defn}

\begin{defn}[External direct sum]
	Consider two arbitrary vector spaces $\mathcal{U}$ and $\mathcal{V}$
	over some field $\F$. We may define their direct sum constructively
	by setting $\mathcal{U} \oplus \mathcal{V}
	= \mathcal{U} \times \mathcal{V}
	= \{ (u,v) : u \in \mathcal{U}, v \in \mathcal{V} \}$ with vector
	addition and scalar multiplication defined by:
	\begin{itemize}
		\item $(u,v)+(u',v')=(u+u',v+v')$ and
		\item $\lambda (u,v) = ( \lambda u, \lambda v) : \lambda \in \F$.
	\end{itemize}
\end{defn}

Consider two vector spaces $\mathcal{U}$ and $\mathcal{V}$ with the linear
map $\varphi: \mathcal{U} \to \mathcal{V}$ and the short exact sequence
\[
	\mathcal{U} \stackrel{\alpha}{\to}
	\mathcal{U} / \ker(\varphi) \stackrel{\gamma}{\to}
	\im(\varphi) \stackrel{\beta}{\to} \mathcal{V}.
\]

In particular, observe that the map
$\alpha: u \mapsto (u,0)$ and
$\beta: v \mapsto (0,v)$ are injective. While the maps
$\alpha^{'}: (u,0) \mapsto u$ and
$\beta^{'}: (0,v) \mapsto v$ are subjective. That is
$\gamma$ is isomorphic.
Hence we have the following commutative diagram:
%\begin{center}
%	\begin{tikzcd}
%		\mathcal{U} \ar{dd}{\iota_{\mathcal{U}}} \drar{\alpha}
%		& & \mathcal{V} \ar[swap]{dd}{\iota_{\mathcal{V}}} \dlar[swap]{\beta} & \\
%		& \dlar{\alpha^{'}}
%		\mathcal{U} \oplus \mathcal{V}
%		\drar[swap]{\beta^{'}} & \\
%		\mathcal{U} & & \mathcal{V}
%	\end{tikzcd}
%\end{center}

Observe that the internal and external direct sums are isomorphic in this case.

We can now write higher dimensional vector spaces are direct sums of one
dimensional subspaces by the natural isomorphism outlined above.

\begin{exmp}
	Let $\R_1 = \{ (\mathbf{x}_1,\mathbf{0}) : \mathbf{x}_1 \in \R \}$
	and $\R_2 = \{ (\mathbf{0},\mathbf{x}_2) : \mathbf{x}_2 \in \R \}$.
	Then we see that $\R^2 = \R_1 \oplus \R_2 \cong \R \oplus \R$.
\end{exmp}

More generally, consider the subspace $\R_i$ with vectors of the form
\[
	\R_i = \{ ( \mathbf{0}_1, \dots , \mathbf{0}_{i-1}, \mathbf{x}_i,
	\mathbf{0}_{i+1}, \dots , \mathbf{0}_n) : \mathbf{x}_i \in \R \}.
\]
Then we have the following:
\begin{align*}
	\R^n &= \bigoplus_{i=1}^{n} \R_i
	\\
	&= \R_1 \oplus \R_2 \oplus \dots \oplus \R_n
	\\
	& \cong \underbrace{\R \oplus \dots \oplus \R}_{\text{n times}}
	\\
	&= \bigoplus_{i=1}^{n} \R.
\end{align*}
A subspace of $\R^n$ that is isomorphic to $\R_i$ is called the
$\emph{canonical copy}$ of $\R_i$ in $\R$.

\begin{exmp}
	\begin{align*}
		\intertext{Let,}
		A &=
		\begin{pmatrix}
			1 & 0 \\
			0 & 1
		\end{pmatrix}
		\\
		B &=
		\begin{pmatrix}
			2 & 1 & 0 \\
			0 & 2 & 1 \\
			0 & 0 & 2
		\end{pmatrix}
		\text{ and } C &= (1).
		\intertext{Then,}
		A \oplus B \oplus C &=
		\begin{pmatrix}
			A & 0 & 0 \\
			0 & B & 0 \\
			0 & 0 & C
		\end{pmatrix}.
	\end{align*}
\end{exmp}

\begin{rem}
	Note that direct sums are $\emph{associative}$, that is,
	$(A \oplus B) \oplus C = A \oplus (B \oplus C)$.
\end{rem}
