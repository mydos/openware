% Copyright © 2012 Edward O'Callaghan. All Rights Reserved.

\section{Groups} % (fold)
\label{sec:groups}

\begin{defn}[Binary operation]
 A $\textbf{binary operation}$ on a set $\mathcal{X}$ is a map
 $\circ : \mathcal{X} \times \mathcal{X} \to \mathcal{X}$.
 $\textbf{N.B.}$ that the binary operation is $\emph{closed}$.
\end{defn}

\begin{defn}[Magma]
 A $\textbf{magma}$ is a set $\mathcal{M}$ equipped with a binary operation $\circ$.
 We denote the magma as the tuple pair $(\mathcal{M}, \circ)$.
\end{defn}


\begin{defn}[Semi-group]
 A $\textbf{semi-group}$ is a set $\mathcal{G}$ equipped with binary operation that is $\emph{associative}$.
 Hence, a semi-group is a magma where the operation is $\emph{associative}$;
 That is, given any $x,y,z \in \mathcal{G}$ then $x \circ (y \circ z) = (x \circ y) \circ z \in \mathcal{G}$.
 We denote the semi-group as the tuple pair $(\mathcal{G}, \circ)$, not to be confused with a magma from context.
\end{defn}

\begin{defn}[Monoid]
 A $\textbf{semi-group with idenitity}$ or, $\textbf{monoid}$ for short, is a semi-group $(\mathcal{G}, \circ)$
 with a unique identity element $e \in \mathcal{G}$ such that $x \circ e = x = e \circ x \, \forall x \in \mathcal{G}$
\end{defn}


\begin{proof}[Proof: unquieness of idenitity]
 Assume some other identity $e^{'}$ exists in $\mathcal{G}$ then, $e^{'} = e^{'} \circ e = e \circ e^{'} = e. \qedhere$
\end{proof}


\begin{exmp}
 Given $\mathcal{G} = \mathbb{N}$ with the binary law of composition $\circ$ to be defined as arithmetic addition $+$.
 Then, $(\mathbb{N}, +)$ forms a semi-group with identity $0$. Verify the axioms.
\end{exmp}


\begin{defn}[Group]
 A $\textbf{group}$ is a monoid where every element has an inverse. An abelian group is a group that is commutative.
\end{defn}

\begin{exmp}
 Given $\mathcal{G} = \mathbb{Z}$ with the binary law of composition $\circ$ to be defined as arithmetic addition $+$.
 Then, $(\mathbb{Z}, +)$ forms a semi-group with identity $0$. Verify the axioms.
\end{exmp}

\begin{question}
 Why does the set of naturals $\mathbb{N}$ not form a group under multiplication, however does form a monoid?
\end{question}

\begin{defn}[Subgroup]
	A group $\mathcal{H}$ is a $\textbf{subgroup}$ of a group $\mathcal{G}$ if the restriction of the binary operation
	$\circ : \mathcal{H} \times \mathcal{H} \to \mathcal{H}$ is a group operation on $\mathcal{H}$.
	In particular, A non-empty subset $\mathcal{H}$ of a group $\mathcal{G}$ is a subgroup of
	$\mathcal{G}$ if and only if $h_1 \circ h_2 \in \mathcal{H}$ whenever $h_1, h_2 \in \mathcal{H}$,
	and $h^{-1} \in \mathcal{H}$ whenever $h \in \mathcal{H}$. We denote the subgroup by $\mathcal{H} \leq \mathcal{G}$.
\end{defn}

\begin{thm}[Smallest subgroup]
	If $\mathcal{A}$ is a subset of a group $\mathcal{G}$, there is a $\emph{smallest}$ subgroup
	$\mbox{Gp}(\mathcal{A})$ of $\mathcal{G}$ which contains $\mathcal{A}$, the subgroup $\emph{generated}$
	by $\mathcal{A}$.
\end{thm}

\begin{exmp}
	Suppose $\mathcal{A}=\{g\}$ then $\mbox{Gp}(\mathcal{A})=\mbox{Gp}(g)$ and so
	$\mbox{Gp}(g) = \{ g^n : n \in \Z \}$, where $g^0 = e$, $g^n$ is the product of $n$ copies of $g$
	where $n>0$, and $g^n$ is the product of $|n|$ copies of $g^{-1}$ when $n<0$.
\end{exmp}

\begin{defn}[Cyclic group]
	A group $\mathcal{G}$ is $\emph{cyclic}$ if $\mathcal{G}=\mbox{Gp}(g)$ for some $g \in \mathcal{G}$.
	Such a element is called a $\emph{generator}$ of the group.
\end{defn}

\begin{defn}[Group order]
	If a group $\mathcal{G}$ has finitely many elements, then the $\emph{order}$
	$o(\mathcal{G})$ is the number of elements of $\mathcal{G}$.
\end{defn}

\begin{defn}[Normal subgroup]
	A subgroup $\mathcal{H}$ of a group $\mathcal{G}$ is a $\textbf{normal}$, or $\emph{self-conjugate}$, if
	$g^{-1} h g \in \mathcal{H} \, \forall g \in \mathcal{G}$ and $h \in \mathcal{H}$. We denote the normal
	$\mathcal{H} \unlhd \mathcal{G}$.
\end{defn}

\begin{defn}[Simple group]
	A group $\mathcal{G}$ is $\textbf{simple}$ if it has no normal subgroups other than $\{e\}$ and $\mathcal{G}$.
\end{defn}

\subsection{Group Homomorphisms} % (fold)
\label{subsec:homomorphisms}
Homomorphisms are structure preserving mappings. In group homomorphisms we preserve the structure of
the binary operation $\circ$ as follows;
\begin{defn}[Homomorphism]
	Let $\mathcal{G}$ and $\mathcal{H}$ be two groups. Then a mapping
	\[
		\varphi : \mathcal{G} \to \mathcal{H}
	\]
	is called a $\emph{homomorphism}$ if
	\[
		\varphi(x \circ y) = \varphi(x) \circ \varphi(y) : x,y \in \mathcal{G}
	\]
\end{defn}
It follows that, for some $g \in \mathcal{G}$ we have,
\begin{align*}
	\varphi ( e_g ) &= \varphi (g \circ g^{-1})
	\\
	&= \varphi (g) \circ \varphi (g^{-1})
	\\
	&= \varphi (g) \circ (\varphi (g) )^{-1}
	\\
	&= e_h \in \mathcal{H}.
\end{align*}
That is the identity $e$ has been preserved.
Hence, it does not matter if we compose in $\mathcal{G}$ and map to $\mathcal{H}$ or take two elements in $\mathcal{G}$
then compose the mapped elements in $\mathcal{H}$, since the group structure has been preserved.

How much information about the elements inside the structure is, however, another quality to consider. Hence we fix some
terminology here.
\begin{itemize}
	\item A homomorphism that is injective is called monomorphic.
	\item A homomorphism that is surjective is called epimorphic.
	\item A homomorphism that is bijective is called isomorphic.
\end{itemize}
Thus we have the following definitions by considering a group homomorphism $\varphi : \mathcal{G} \to \mathcal{H}$.

\begin{defn}[Monomorphic]
	$\varphi$ is $\textbf{monomorphic}$ if for $\varphi(x) = \varphi(y) \implies x = y \, \forall x,y \in \mathcal{G}$.
\end{defn}

\begin{defn}[Epimorphic]
	$\varphi$ is $\textbf{epimorphic}$ if $\forall h \in \mathcal{H} \exists g \in \mathcal{G}$ so that $\varphi(g) = h$.
\end{defn}

\begin{defn}[Isomorphic]
	$\varphi$ is $\textbf{isomorphic}$ if $\varphi$ is $\textbf{both}$ mono- and epic- morphic.
\end{defn}

Some special cases are sometimes of particular interest and we shall outline them now.

\begin{defn}[Endomorphic]
	A monomorphism $\mathcal{G} \to \mathcal{G}$ for a group $\mathcal{G}$
	is called an $\emph{endomorphism}$ of $\mathcal{G}$.
\end{defn}

\begin{defn}[Automorphic]
	A isomorphism $\mathcal{G} \to \mathcal{G}$ for a group $\mathcal{G}$
	is called an $\emph{automorphism}$ of $\mathcal{G}$.
\end{defn}

\begin{rem}
	The set $Aut(\mathcal{G})$ of automorphisms of $\mathcal{G}$ forms a group, when composition of
	mappings is taken as the group law of composition.
\end{rem}

\subsection{Properties of homomorphisms} % (fold)
\label{subsec:propertiesofhomomorphisms}

\begin{defn}[kernel]
	If $\varphi : \mathcal{G} \to \mathcal{H}$ is a group homomorphism,
	then the $\emph{kernel}$ is the set
	$\ker(\varphi) = \{ g \in \mathcal{G} : \varphi (g) = e_h \in \mathcal{H} \}$.
\end{defn}

If $\varphi : \mathcal{G} \to \mathcal{H}$ is a group homomorphism, then
observe that $\ker(\varphi)$ is a normal subgroup of of $\mathcal{G}$.

\subsection{Cosets} % (fold)
\label{subsec:cosets}

Let $\mathcal{G}$ be a group and $\mathcal{H}$ be a subgroup of $\mathcal{G}$ with $g \in \mathcal{G} : g \notin H$, then
\begin{defn}[Left Coset]
	$gH = \{gh : h \in H\}$ is a $\textbf{left coset of } \mathcal{H}$ in $\mathcal{G}$.
\end{defn}

\begin{defn}[Right Coset]
	$Hg = \{hg : h \in H\}$ is a $\textbf{right coset of } \mathcal{H}$ in $\mathcal{G}$.
\end{defn}

\begin{defn}[Normal Subgroup]
	If $gH = Hg$ then $\mathcal{H}$ is a $\textbf{normal}$ subgroup of $\mathcal{G}$, denoted by $\mathcal{H} \unlhd \mathcal{G}$.
\end{defn}

\subsection{Factor (or Quotient) groups} % (fold)
\label{subsec:factorgroups}
Let $\mathcal{G}$ be a commutative group and consider a subgroup $\mathcal{H}$.
Then $\mathcal{H}$ determines an equivalence relation in $\mathcal{G}$ given by
\[
	x \sim x' \mbox{ iff } x - x' \in \mathcal{H}.
\]
..

\subsection{Non-commutative Groups} % (fold)
\label{sec:noncommutative-groups}
A common class of non-commutative groups are transformation groups.
Note:
\begin{defn}[Transformation]
 A bijective map $\varphi: X \to X$ is called a $\textbf{transformation}$ of X.
 \begin{note}
  The most trivial case is the $\emph{idenitity map} \, \Id{X}$ by $\Id{X}(x) = x, \, \forall x \in X$.
 \end{note}
\end{defn}
Hence, there exists a inverse $\varphi^{-1}$ of $\varphi$ such that $\varphi^{-1} \circ \varphi = \Id{X} = \varphi \circ \varphi^{-1}$.
Now, take two transformations of X, $\varphi$ and $\psi$, and let the product $\varphi \circ \psi$ be well defined.
Then the set of all transformations of X form the group $\textbf{Transf(X)}$.
Since, given $\varphi , \psi , \phi \in Transf(X)$ then we have associativity, $\varphi \circ (\psi \circ \phi) = (\varphi \circ \psi) \circ \phi$.
We have identity $e = \Id{X} \in Transf(X)$ and so, inverses $\forall \varphi \in Transf(X) \exists ! \varphi^{-1} : \varphi \circ \varphi^{-1} = e$.
Closure follows from the composition of two transformations $\varphi$ and $\psi$, since $(\varphi \circ \psi)^{-1} = \psi^{-1} \circ \varphi^{-1}$.

A transformation group is a type of group action which describes symmetries of objects. More abstractly,
since a group $\mathcal{G}$ is a category with a single object in which every morphism is bijective.
Then, a group action is a $\emph{forgetful functor}$ $\mathcal{F}$ from the group $\mathcal{G}$ in the category $\textbf{Grp}$
to the set category $\textbf{Set}$ that is, $\mathcal{F} : \mathcal{G} \to \textbf{Set}$.

\subsection{Group actions} % (fold)
\label{subsec:groupaction}

For any mathematical object (e.g. sets, groups, vector spaces) $X$ an isomorphism
of $X$ is a symmetry of $X$. The set of all isomorphisms of $X$, or symmetries
of $X$, form a group called the symmetry group of $X$, denoted $Sym(X)$.
More formally;

\begin{defn}[Group action]
	An $\emph{action}$ of a group $\mathcal{G}$ on a mathematical object $X$ is
	a mapping $\mathcal{G} \times X \to X$, defined by $(g,x) \mapsto g . x$
	satisfying:
	\begin{itemize}
			\item $e . x = x \, \forall x \in X$ and
			\item $(g h) . x = g . (h . x) \, \forall g,h \in \mathcal{G}, x \in X$.
	\end{itemize}
	That is, we have the ($\emph{left}$) $\mathcal{G}$-action on $X$ and denote this
	by $\mathcal{G} \acts X$.
\end{defn}

Notice that we may study properties of the symmetries of some mathematical object $X$
without reference to the structure of $X$ in particular.

\subsection{Permutations} % (fold)
\label{sec:permutations}
Take a finite set X with $|X|=n$, then the transformations of X are called $\textbf{permutations}$ of the
elements of X. In particular, the group of permutations of $X=\{ 1, 2, \cdots, n \}$ is a $\textbf{symmetric group}$,
denoted $S_n$, with $\textbf{order}$ $|S_n|=n!$. Thus, by taking any subgroup of $S_n$ we have a
$\textbf{permutation group}$. Also note that, for finite sets, $\emph{permutation}$ and $\emph{bijective maps}$
refer to the same operation, namely rearrangement of elements of X.
Another way is to consider, a group $\mathcal{G}$ and set X.
Then a group action is defined as a group homomorphism $\varphi$ from $\mathcal{G}$ to the symmetric group of X.
That is, the action $\varphi: \mathcal{G} \to S_n(X)$, assigns a permutation of X to each element of the group $\mathcal{G}$ in the following way:
\begin{itemize}
 \item From the identity element $e \in \mathcal{G}$ to the identity transformation $\Id{X}$ of X, that is, $\varphi : e \to \Id{X}$;
 \item A product of group homomorphisms $\varphi \circ \psi \in \mathcal{G}$ is then the composite of permutations given by $\varphi$ and $\psi$ in X.
\end{itemize}
Given that each element of $\mathcal{G}$ is represented as a permutation. Then a group action can also be consider as a permutation representation.

A permutation $\sigma \in S_n$ can be written,
\[
 \sigma =
 \begin{pmatrix}
  1 & 2 & \cdots & n \\
  a_1 & a_2 & \cdots & a_n
 \end{pmatrix}
 \, \, \,
 \text{where } a_1 = \sigma(1), a_2 = \sigma(2), \cdots .
\]

The idenitity permutation $\Id{n} \in S_n$ is simply,
\[
 \Id{n} =
 \begin{pmatrix}
  1 & 2 & \cdots & n \\
  1 & 2 & \cdots & n
 \end{pmatrix}
\].

Since $|S_n|=n!$ then the total number of ways n elements maybe permuted is $n!$.

Take any two permutations $\sigma,\pi \in S_n$ then composition is well defined as $\textbf{functional composition}$
as follows.

Given,
\[
 \sigma =
 \begin{pmatrix}
  1 & 2 & \cdots & n \\
  a_1 & a_2 & \cdots & a_n
 \end{pmatrix}
 \, \, and \, \,
 \pi =
 \begin{pmatrix}
  a_1 & a_2 & \cdots & a_n \\
  b_1 & b_2 & \cdots & b_n
 \end{pmatrix}
\]
then,

\begin{align*}
 \pi \circ \sigma &=
 \begin{pmatrix}
  1 & 2 & \cdots & n \\
  \pi(a_1) & \pi(a_2) & \cdots & \pi(a_n)
 \end{pmatrix}
 \\
 &=
 \begin{pmatrix}
  1 & 2 & \cdots & n \\
  b_1 & b_2 & \cdots & b_n
 \end{pmatrix}
\end{align*}
.

A inverse of any permutation $\sigma \in S_n$ is given by,
\[
 \sigma^{-1} =
 \begin{pmatrix}
  1 & 2 & \cdots & n \\
  a_1 & a_2 & \cdots & a_n
 \end{pmatrix}^{-1}
 =
 \begin{pmatrix}
  a_1 & a_2 & \cdots & a_n \\
  1 & 2 & \cdots & n
 \end{pmatrix}
\]

\subsection{Permutation parity}
Consider the algebraic structure:
\[
	\triangle_n (x_1, \dots , x_n) = \prod_{i<j} (x_i - x_j)
\]
TODO..

\subsection{Fields} % (fold)
\label{subsec:fields}
We now may build higher order algebraic structures using the notion of a group.

\begin{defn}[Field]
	A $\textbf{field}$ $\mathbb{F}$ is a set together with two binary operations, addition and multiplication, such that:
	\begin{itemize}
		\item addition forms an abelian group,
		\item multiplication forms a abelian quasi-group, i.e. a commutative multiplicative group on the set $\mathbb{F} - \{0\}$, 
	\end{itemize}
	coupled together with a law of distribution between the two binary operations.
\end{defn}
