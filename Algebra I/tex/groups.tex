\section{Groups} % (fold)
\label{sec:groups}

\begin{defn}[Binary operation]
 A $\textbf{binary operation}$ on a set $\mathcal{G}$ is a map
 $\circ : \mathcal{G} \times \mathcal{G} \to \mathcal{G}$.
 $\textbf{N.B.}$ that the binary operation is $\emph{closed}$.
\end{defn}

\begin{defn}[Magma]
 A $\textbf{magma}$ is a set $\mathcal{M}$ equipped with a binary operation $\circ$.
 We denote the magma as the tuple pair $(\mathcal{M}, \circ)$.
\end{defn}


\begin{defn}[Semi-group]
 A $\textbf{semi-group}$ is a set $\mathcal{G}$ equipped with binary operation that is $\emph{associative}$.
 Hence, a semi-group is a magma where the operation is $\emph{associative}$;
 That is, given any $x,y,z \in \mathcal{G}$ then $x \circ (y \circ z) = (x \circ y) \circ z \in \mathcal{G}$.
 We denote the semi-group as the tuple pair $(\mathcal{G}, \circ)$, not to be confused with a magma from context.
\end{defn}

\begin{defn}[Monoid]
 A $\textbf{semi-group with idenitity}$ or, $\textbf{monoid}$ for short, is a semi-group $(\mathcal{G}, \circ)$
 with a unquie idenitity elememt $e \in \mathcal{G}$ such that $x \circ e = x = e \circ x \, \forall x \in \mathcal{G}$
\end{defn}


% TODO: Add proof of unquieness of idenitity.


\begin{exmp}
 Given $\mathcal{G} = \mathbb{Z}$ with the binary law of composition $\circ$ to be defined as arithmetic addition $+$.
 Then, $(\mathbb{Z}, +)$ forms a semi-group with idenitity $0$. Verify the axioms.
\end{exmp}


\begin{defn}[Group]
 A $\textbf{group}$ is a monoid where every element has an inverse. A abelian group is a group that is commutative.
\end{defn}


% TODO: Give some examples of groups..


\subsection{Non-commutative groups} % (fold)
\label{sec:noncommutative-groups}
A common class of non-commutative groups are transformation groups.
Note:
\begin{defn}[Transformation]
 A bijective map $\varphi: X \to X$ is called a $\textbf{transformation}$ of X.
 \begin{note}
  The most trivial case is the $\emph{idenitity map} \, \Id{X}$ by $\Id{X}(x) = x, \, \forall x \in X$.
 \end{note}
\end{defn}
Hence, there exists a inverse $\varphi^{-1}$ of $\varphi$ such that $\varphi^{-1} \circ \varphi = \Id{X} = \varphi \circ \varphi^{-1}$.
Now, take two transformations of X, $\varphi$ and $\psi$, and let the product $\varphi \circ \psi$ be well defined.
Then the set of all transformations of X form the group $\textbf{Transf(X)}$.
Since, given $\varphi , \psi , \phi \in Transf(X)$ then we have associatity, $\varphi \circ (\psi \circ \phi) = (\varphi \circ \psi) \circ \phi$.
We have idenitity $e = \Id{X} \in Transf(X)$ and so, inverses $\forall \varphi \in Transf(X) \exists ! \varphi^{-1} : \varphi \circ \varphi^{-1} = e$.
Closure follows from the composition of two transformations $\varphi$ and $\psi$, since $(\varphi \circ \psi)^{-1} = \psi^{-1} \circ \varphi^{-1}$.

A transformation group is a type of group action which describes symmetries of objects. More abstractly,
since a group $\mathcal{G}$ is a category with a single object in which every morphism is bijective.
Then, a group action is a $\emph{forgetful functor}$ $\mathcal{F}$ from the group $\mathcal{G}$ in the category $\textbf{Grp}$
to the set category $\textbf{Set}$ that is, $\mathcal{F} : \mathcal{G} \to \textbf{Set}$.

That is, for a group $\mathcal{G}$ and set X, a group action is defined as a group homomorphism $\varphi$
from $\mathcal{G}$ to the symmetric group of X.
The action assigns a permutation of X to each element of the group in such a way that the permutation of X assigned to:
\begin{itemize}
 \item The identity element $e \in \mathcal{G}$ is the identity transformation of X, that is, $\Id{X}$;
 \item A product $\varphi \circ \psi \in \mathcal{G}$ is the composite of the permutations assigned to $\varphi$ and $\psi$.
\end{itemize}
Given that each element of $\mathcal{G}$ is represented as a permutation.
Then a group action can also be consider as a permutation representation.

\subsection{Permutations} % (fold)
\label{sec:permutations}
Now take a finite set X with $|X|=n$, then the transformations of X are called $\textbf{permutations}$ of the
elements of X. In particular, the group of permutatons of $X=\{ 1, 2, \cdots, n \}$ is a $\textbf{symmetric group}$ of
$\emph{order}$ n, denoted $S_n$ with $\textbf{order}$ $|S_n|=n!$. Thus, by taking any subgroup of $S_n$ we have a
$\textbf{permutation group}$. Also note that, for finite sets, $\emph{permutations}$ and $\emph{bijective maps}$
refer to the same operation, namely rearrangement of elements of X.

A permutaton $\sigma \in S_n$ can be notated by,
\[
 \sigma =
 \begin{pmatrix}
  1 & 2 & \cdots & n \\
  a_1 & a_2 & \cdots & a_n
 \end{pmatrix}
 \, \, \,
 \text{where } a_1 = \sigma(1), a_2 = \sigma(2), \cdots .
\]

The idenitity permutation $\Id{n} \in S_n$ is simply,
\[
 \Id{n} =
 \begin{pmatrix}
  1 & 2 & \cdots & n \\
  1 & 2 & \cdots & n
 \end{pmatrix}
\].

Since $|S_n|=n!$ then the total number of ways n elements maybe permuted is $n!$.

Take any two permutations $\sigma,\pi \in S_n$ then composition is well defined as $\textbf{functional composition}$
as follows.

Given,
\[
 \sigma =
 \begin{pmatrix}
  1 & 2 & \cdots & n \\
  a_1 & a_2 & \cdots & a_n
 \end{pmatrix}
 \, \, and \, \,
 \pi =
 \begin{pmatrix}
  a_1 & a_2 & \cdots & a_n \\
  b_1 & b_2 & \cdots & b_n
 \end{pmatrix}
\]
then,

\begin{align*}
 \pi \circ \sigma &=
 \begin{pmatrix}
  1 & 2 & \cdots & n \\
  \pi(a_1) & \pi(a_2) & \cdots & \pi(a_n)
 \end{pmatrix}
 \\
 &=
 \begin{pmatrix}
  1 & 2 & \cdots & n \\
  b_1 & b_2 & \cdots & b_n
 \end{pmatrix}
\end{align*}
.

A inverse of any permutation $\sigma \in S_n$ is given by,
\[
 \sigma^{-1} =
 \begin{pmatrix}
  1 & 2 & \cdots & n \\
  a_1 & a_2 & \cdots & a_n
 \end{pmatrix}^{-1}
 =
 \begin{pmatrix}
  a_1 & a_2 & \cdots & a_n \\
  1 & 2 & \cdots & n
 \end{pmatrix}
\]
