% Copyright © 2012 Edward O'Callaghan. All Rights Reserved.

\section{First Isomorphism Theorem} % (fold)
\label{sec:firstisomorphismtheorem}

\begin{thm}
Let $\mathcal{G}$ and $\mathcal{H}$, and let
$\varphi : \mathcal{G} \to \mathcal{H}$ be a group homomorphism.
Then:
\begin{itemize}
	\item The kernel of $\varphi$ is a normal subgroup of $\mathcal{G}$;
		$\ker(\varphi) \unlhd \mathcal{G}$,
	\item The image of $\varphi$ is a subgroup of $\mathcal{H}$;
		$\im(\varphi) \leq \mathcal{H}$, and
	\item The image of $\varphi$ is also isomorphic to the factor group
		$\mathcal{G} / \ker(\varphi)$;
		$\im(\varphi) \cong \mathcal{G} / \ker(\varphi)$.
\end{itemize}
In particular, if $\varphi$ is epimorphic then
$\mathcal{H} \cong \mathcal{G} / \ker(\varphi)$.
\end{thm}

We may represent these fundamental relations in the following commutative diagram.

% FIXME: For some reason this broke in later versions of LaTeX, investigate why..
%\begin{tikzcd}
%    & & 0 \ar{dr} &  & 0\\
%	& & & \stack{\mathcal{G} / \ker(\varphi)}{\cong\;\im(\varphi)} \ar{ur} \ar{dr}{\iota} & \\
%	& & \mathcal{G} \ar{rr}{\varphi} \ar{ur}{\pi} &  & \mathcal{H} \\
%    & \ker(\varphi) \ar{ur}{\kappa} &  &  & \\
%    0 \ar{ur} & 
%\end{tikzcd}

Notice the $\emph{exact sequence}$ that runs from the lower left to the upper right of the commutative diagram.
