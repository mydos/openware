\section{Adjugate (Classical Adjoint)} % (fold)
\label{sec:adjugate}

\begin{defn}[Cofactor]
	For some square matrix $A \in \mathcal{M}_{n,n}(\F)$
	then the minor of the $a_{ij}$ entry, denoted by
	$M_{i,j}$, is defined to be the determinant of the
	$\emph{submatrix}$ obtained by removing the $(i,j)^{th}$
	row and column from $A$. That is,
	\[
		C_{i,j} = (-1)^{i+j} M_{i,j}
	\]
	where $C_{i,j}$ is called the $\emph{cofactor}$ of $a_{i,j}$.
\end{defn}

\begin{exmp}
	Consider the matrix $A \in \mathcal{M}_{3,3}$ and suppose
	\begin{align*}
		A &=
		\begin{pmatrix}
			a_{11} & a_{12} & a_{13} \\
			a_{21} & a_{22} & a_{23} \\
			a_{31} & a_{32} & a_{33}
		\end{pmatrix}.
		\intertext{We may find the $C_{2,3}$ cofactor in the following way;}
		\intertext{Observe that the minor $M_{2,3}$ may be found as follows,}
		M_{2,3} &=
		\begin{vmatrix}
			a_{11} & a_{12} \\
			a_{31} & a_{32}
		\end{vmatrix}
		\\
		&= a_{11} a_{32} - a_{12} a_{31}
		\intertext{and so the cofactor $C_{2,3}$ is by definition,}
		C_{2,3} &= (-1)^{2+3} M_{2,3}
		\\
		&= (-1)^{5} ( a_{11} a_{32} - a_{12} a_{31} )
		\\
		&= a_{12} a_{31} - a_{11} a_{31}.
	\end{align*}
\end{exmp}

Notice that we may now find the determinant of some $n \times n$ square matrix
in terms of its cofactors. The process of this cofactor expansion is called the
Laplace expansion.

\begin{thm}
	Suppose $A \in \mathcal{M}_{n,n}$ is some $n \times n$ square matrix then
	\begin{align*}
		\intertext{in terms of the $i^{th}$ row we have,}
		\det(A) &= \sum_{i=1}^n a_{ij} C_{ij}
		\intertext{or in terms of the $j^{th}$ column}
		&= \sum_{j=1}^{n} a_{ij} C_{ij}.
	\end{align*}
\end{thm}

\begin{proof}
	TODO: Prove Laplace expansion.
\end{proof}

\begin{exmp}
	Consider the matrix
	\begin{align*}
		A &=
		\begin{pmatrix}
			1 & 3 & 5 \\
			2 & 1 & 3 \\
			5 & 2 & 1
		\end{pmatrix}.
		\intertext{Then,}
		\det(A) &=
		1 \cdot
		\begin{vmatrix}
			1 & 3 \\
			2 & 1
		\end{vmatrix}
		-3 \cdot
		\begin{vmatrix}
			2 & 3 \\
			5 & 1
		\end{vmatrix}
		+5 \cdot
		\begin{vmatrix}
			2 & 1 \\
			5 & 2
		\end{vmatrix}
		\\
		&= 1 \cdot (-5) -3 \cdot (-13) +5 \cdot (-1) = 29.
	\end{align*}
\end{exmp}

Recall that the determinant is a measure of overall scaling of a matrix. Observe
that each cofactor are essentially sub determinants or sub scalings. By this we
are motivated to form a matrix of these cofactors in the following way:

\begin{defn}[Matrix of cofactors]
	Suppose $A \in \mathcal{M}_{n,n}(\R)$ is some $n \times n$ real square matrix
	given by,
	\begin{align*}
		A &=
		\begin{pmatrix}
			a_{11} & a_{12} & \cdots & a_{1n} \\
			a_{21} & a_{22} & \cdots & a_{2n} \\
			\vdots & \vdots & \ddots & \vdots \\
			a_{n1} & a_{n2} & \cdots & a_{nn}
		\end{pmatrix}
		\intertext{with cofactors $C_{ij}$ of $a_{ij}$.
		Then the $\emph{cofactor matrix}$ $C$ is defined as,}
		C &=
		\begin{pmatrix}
			C_{11} & C_{12} & \cdots & C_{1n} \\
			C_{21} & C_{22} & \cdots & C_{2n} \\
			\vdots & \vdots & \ddots & \vdots \\
			C_{n1} & C_{n2} & \cdots & C_{nn}
		\end{pmatrix}.
	\end{align*}
\end{defn}

Now since a matrix of a linear transformation is really just a linear dilation
(\nameref{eigenvalues}) of each dimension of the coordinate system (\nameref{eigenvectors}).
Then for a real matrix we may find the inverse by means of dilating each respective
dimension back. This is essentially what the adjugate matrix is. In fact, even when
the matrix is $\emph{singular}$ (non-invertible) the adjugate is still well defined.
That is, the adjugate is in actual fact, the pre-image of a real linear transformation.

More formally, we have the following definition;

\begin{defn}[Adjugate (Classical adjoint)]
	Suppose $A \in \mathcal{M}_{n,n}(\R)$ is some real $n \times n$ square matrix with
	cofactor matrix $C$. Then the $\emph{adjugate}$ or classical adjoint of $A$ is
	defined as,
	\[
		adj(A) = C^T.
	\]
\end{defn}

Thus we have the following result:
\begin{lem}
	For some non-singular $n \times n$ real square matrix $A \in \mathcal{M}_{n,n}(\R)$
	we have,
	\[
		adj(A) = A^{-1} \det(A).
	\]
\end{lem}

\begin{proof}
	TODO: prove this lemma..
\end{proof}

\begin{exmp}
	\begin{align*}
		\intertext{Suppose $A \in \mathcal{M}_{2,2}(\R)$ is given by,}
		A &=
		\begin{pmatrix}
			a_{11} & a_{12} \\
			a_{21} & a_{22}
		\end{pmatrix}
		\intertext{then the adjugate of $A$ is given by,}
		adj(A) &=
		\begin{pmatrix}
			a_{22} & -a_{21} \\
			-a_{12} & a_{11}
		\end{pmatrix}^T
		\\
		&=
		\begin{pmatrix}
			a_{22} & -a_{12} \\
			-a_{21} & a_{11}
		\end{pmatrix}.
	\end{align*}
\end{exmp}

Some useful properties follow from the adjugate:
\begin{itemize}
	\item $adj(I) = I$,
	\item $adj(AB) = adj(B) adj(A)$,
	\item $adj(A^T) = adj(A)^T$.
\end{itemize}

\begin{proof}
	TODO prove the above properties.
\end{proof}
