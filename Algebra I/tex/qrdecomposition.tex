% Copyright © 2012 Edward O'Callaghan. All Rights Reserved.

\section{QR decomposition} % (fold)
\label{sec:QRdecomposition}
The QR decomposition of a matrix $A \in \mathcal{M}_{n,m}(\F)$
into a product of an orthogonal matrix $Q$ and an upper
triangular matrix $R$, written $A=QR$. However, first we
must look at the Gram-Schmidt algorithm in order to obtain
an $\emph{orthonomalised babsis}$ of $A$.

\begin{defn}[Projection operator]
	Consider two $n \times 1$ vectors $\mathbf{u},\mathbf{v} \in \F^n$.
	Then the \textbf{orthogonal projection}
	$\emph{of } \mathbf{v} \emph{ onto}$ the line spanned by $\mathbf{u}$
	is given by,
	\[
		proj_{\mathbf{u}} (\mathbf{v})
		= \frac{\langle \mathbf{v},\mathbf{u} \rangle}
		{\langle \mathbf{u}, \mathbf{u} \rangle} \mathbf{u}.
	\]
\end{defn}

Recall that when the field $\F$ is taken to be real for two $n \times 1$
vectors $\mathbf{u}, \mathbf{v} \in \R^n$ then the inner product is then
$\langle \mathbf{u} , \mathbf{v} \rangle = \mathbf{u}^T \mathbf{v}$.
When the field $\F$ is taken to be complex for two $n \times 1$ vectors
$\mathbf{u}, \mathbf{v} \in \C^n$ then the inner product is then
$\langle \mathbf{u} , \mathbf{v} \rangle = \mathbf{u}^{*} \mathbf{v}$.
In fact, since the conjugate of a real number is itself the usual
complex version can simply be used without issue.

\subsection{Gram-Schmidt} % (fold)
\label{subsec:gram-schmidt}

The Gram-Schmidt algorithm is a method of orthonormalising
a finite set of linearly independent vectors that span an
inner product space into a orthonormal basis.

\begin{align*}
	\intertext{Consider a finite sequence of vectors given by the spanning set,}
	S &= \{ \mathbf{v}_1 , \dots , \mathbf{v}_k \}.
	\intertext{Then the sequence of Gram-Schmidt orthogonalised vectors,
	$\{ \mathbf{u}_1, \dots , \mathbf{u}_k \}$ is given by,}
	\mathbf{u}_k &= \mathbf{v}_k - \sum_{j=1}^{k-1} proj_{\mathbf{u}_k} (\mathbf{v}_k).
	\intertext{Hence, we may normalise the sequence by,}
	\mathbf{e}_k &= \frac{\mathbf{u}_k}{\| \mathbf{u}_k \|}.
\end{align*}

\begin{exmp}
	\begin{align*}
		\intertext{Consider the spanning set,}
		S &= \left\{
			\mathbf{v}_1 = \begin{pmatrix} 1 \\ 2 \\ 3 \end{pmatrix},
			\mathbf{v}_2 = \begin{pmatrix} 5 \\ 2 \\ 1 \end{pmatrix},
			\mathbf{v}_3 = \begin{pmatrix} 3 \\ 2 \\ 4 \end{pmatrix}
		\right\}.
		\intertext{Let $\mathbf{u}_1 = \mathbf{v}_1$ and so,}
	\end{align*}
\end{exmp}
