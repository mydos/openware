% Copyright © 2013 Edward O'Callaghan. All Rights Reserved.

\section{Characters} % (fold)
\label{sec:characters}
A \emph{group character} is a group homomorphism,
$\chi : \mathcal{G} \to \C^{\times}$, from a finite abelian group
to the multiplicative group of nonzero complex numbers.
In particular;

\begin{defn}[Character]
	Let $\mathcal{G}$ be a finite abelian group of order $n$, written additively.
	A \emph{character} of $\mathcal{G}$ is a group homomorphism,
	$\chi : \mathcal{G} \to \C^{\times}$, of $\mathcal{G}$ such that:

	\[
		\chi (g_1 + g_2) = \chi(g_1) \chi(g_2) : g_1,g_2 \in \mathcal{G}.
	\]
\end{defn}

\begin{lem}
	\begin{align*}
		\chi(g)^n &= \chi(ng)
		\\
		&= \chi(0) = 1 : g \in \mathcal{G}.
	\end{align*}
	Hence the values of $\chi$ are the $n^{th}$ roots of unity.
\end{lem}

\begin{lem}
	\begin{align*}
		\chi(-g) &= \chi(g)^{-1}
		\\
		&= \overline{\chi(g)}
	\end{align*}
	where the bar denotes the complex conjugation.
\end{lem}

\begin{defn}[Principle Character]
	The \emph{principle character}, denoted by $\chi_{0}$, is defined by
	\[
		\chi_{0}(g) \doteq 1 : g \in \mathcal{G}.
	\]
\end{defn}

\begin{prop}
	For any non-principle character $\chi$ of $\mathcal{G}$,
	\[
		\sum_{g \in \mathcal{G}} \chi(g) = 0.
	\]
\end{prop}

\begin{proof}
	Let $h \in \mathcal{G} : \chi(h) \neq 1$ and let
	$S = \sum_{g \in \mathcal{G}} \chi(g)$. Then,
	\begin{align*}
		\chi(h) \cdot S &= \chi(h) \sum_{g \in \mathcal{G}} \chi(g)
		\\
		&= \sum_{g \in \mathcal{G}} \chi(h) \chi(g)
		\\
		&= \sum_{g \in \mathcal{G}} \chi(g + h)
		\\
		&= S.
		\intertext{Hence it follows that,}
		\chi(h) \cdot S &= S
		\\
		\left( \chi(h) - 1 \right) \cdot S &= 0
		\intertext{and since $\chi(h) \neq 1$ then,}
		\Rightarrow S &= 0. \qedhere
	\end{align*}
\end{proof}

\begin{cor}[First orthogonality relation for characters]
	Let $\chi$ and $\psi$ be two characters of $\mathcal{G}$. Then
	\[
		\sum_{g \in \mathcal{G}} \overline{\chi(g)} \psi(g) =
		\begin{cases}
			n & \text{if } \chi = \psi , \\
			0 & \text{otherwise}.
		\end{cases}
	\]
\end{cor}

\begin{proof}
	Consider the two cases.
	\begin{enumerate}[i.)]
		\item
			For when $\chi = \psi$ it is trivially so, by that,
			\begin{align*}
				\overline{\chi(g)} &=  \chi(g)^{-1}
				\\
				\Rightarrow \overline{\chi(g)} \chi(g) &= 1 \tag{for each $g \in \mathcal{G}$}
				\intertext{and that $\left| \mathcal{G} \right| = n$.}
			\end{align*}
		\item
			If $\chi \neq \psi$ then $\overline{\chi} \psi$ is a
			non-principle character and so $\overline{\chi(g)} \psi(g) = 0$
			for each $g \in \mathcal{G}$. \qedhere
	\end{enumerate}
\end{proof}

\begin{rem}
	As observed in the last proof, the point wise product of the
	characters $\chi$ and $\psi$ is again a character:
	\[
		(\chi \psi)(g) \doteq \chi(g) \psi(g).
	\]
\end{rem}

\begin{prob}
	Let $\hat{\mathcal{G}}$ denote the set of characters.
	Check that $\hat{\mathcal{G}}$ forms an abelian group
	under the operation defined by
	$(\chi \psi)(g) \doteq \chi(g) \psi(g)$ for every $g \in \mathcal{G}$.
	We call $\hat{\mathcal{G}}$ the \emph{dual group} of $\mathcal{G}$.
\end{prob}

\begin{prop}
	Let $\omega$ be a \emph{primitive} $n^{th}$ root of unity. Then the
	map $\chi_j : \Z_n \to \C^{\times}$ defined by $\chi_j(k) = \omega^{k j} : k \in \Z_n$
	is a character of $\Z_n$ for every $j \in \Z$. Moreover,
	\begin{enumerate}
		\item $\chi_i = \chi_j \Leftrightarrow i \equiv j (\mod n)$;
		\item $\chi_j = \chi_{1}^{j}$;
		\item $\hat{\Z_n} = \{ \chi_0, \dots , \chi_{n-1} \}$;
		\item Consequently, $\hat{\Z_n} = \Z_n$. % FIXME: should be isomorphic symbole?
	\end{enumerate}
\end{prop}

\begin{proof}
	TODO..
\end{proof}

\begin{prop}
	If $\mathcal{G}$ is a direct sum, $\mathcal{G} = H_1 \oplus H_2$, and
	$\psi_i : H_i \to \C^{\times}$ is a character of $H_i$, with $i \in \{1,2\}$,
	then $\chi = \psi_1 \oplus \psi_2$, defined by
	\[
		\chi(h_1,h_2) \doteq \psi_1(h_1) \cdot \psi_2(h_2),
	\]
	is a character of $\mathcal{G}$. Moreover, all characters of $\mathcal{G}$ are
	of this form. Consequently,
	\[
		\hat{\mathcal{G}} = \hat{H_1} \oplus \hat{H_2}.
	\]
\end{prop}

\begin{proof}
	TODO..
\end{proof}

\begin{cor}
	\[
		\hat{\mathcal{G}} = \mathcal{G}. % FIXME: Should be isomorphic symbole!
	\]
\end{cor}

\begin{proof}
	TODO..
\end{proof}
