% Copyright © 2012 Edward O'Callaghan. All Rights Reserved.
% !Tex root = Algebra II.tex

\subsection{Exact sequence} % (fold)
\label{sec:exactsequence}
An $\textbf{exact sequence}$ may  either be a finite or infinite sequence of objects and morphisms between them.
Such a sequence is constructed so that the image of one morphism equals the kernel of the next.

In particular;

\begin{defn}[Exact Sequence]
	Consider the sequence of $n$ group homomorphism between $n+1$ groups
	in the following way:

	\begin{center}
	\begin{tikzcd}
		\mathcal{G}_{0} \rar{\varphi_{1}} & \mathcal{G}_{1}
		\rar{\varphi_{2}} & \mathcal{G}_{2} \rar{\varphi_{3}}
		& \dots \rar{\varphi_{n}} & \mathcal{G}_{n}
	\end{tikzcd}
\end{center}

Then the sequence is said to be $\emph{exact}$ if,
\[
	\ker(\varphi_{k+1}) = \im(\varphi_{k})
\]
for every $k \in \{1 \dots n\}$. For $n=3$ the sequence is said to be a
$\textbf{short exact sequence}$.
\end{defn}

\begin{exmp}
	Suppose we have $\mathcal{K} \unlhd \mathcal{G}$ and that
	$q: \mathcal{G} \to \mathcal{G} / \mathcal{K}$ is the quotient mapping.
	Then,
	\begin{center}
	\begin{tikzcd}
		1 \rar & \mathcal{K} \rar{\subseteq} & \mathcal{G}
		\rar{q} & \mathcal{G} / \mathcal{K} \rar & 1
	\end{tikzcd}
\end{center}
is a short exact sequence.
\end{exmp}
