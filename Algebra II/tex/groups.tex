% Copyright © 2013 Edward O'Callaghan. All Rights Reserved.
% !Tex root = Algebra II.tex

\section{Groups} % (fold)
\label{sec:groups}

\begin{defn}[Binary operation]
 A $\textbf{binary operation}$ on a set $\mathcal{X}$ is a map
 $\circ : \mathcal{X} \times \mathcal{X} \to \mathcal{X}'$.
 $\textbf{N.B.}$ that the binary operation need not be closed.
\end{defn}

\begin{defn}[Magma]
 A $\textbf{magma}$ is a set $\mathcal{M}$ equipped with a binary operation $\circ$ that is closed
 under the operation on $\mathcal{M}$. We denote the magma as the tuple pair $(\mathcal{M}, \circ)$.
\end{defn}


\begin{defn}[Semi-group]
 A $\textbf{semi-group}$ is a set $\mathcal{G}$ equipped with binary operation that is $\emph{associative}$.
 Hence, a semi-group is a magma where the operation is $\emph{associative}$;
 That is, given any $x,y,z \in \mathcal{G}$ then $x \circ (y \circ z) = (x \circ y) \circ z \in \mathcal{G}$.
 We denote the semi-group as the tuple pair $(\mathcal{G}, \circ)$, not to be confused with a magma from context.
\end{defn}

\begin{defn}[Monoid]
 A $\textbf{semi-group with idenitity}$ or, $\textbf{monoid}$ for short, is a semi-group $(\mathcal{G}, \circ)$
 with a unique identity element $e \in \mathcal{G}$ such that $x \circ e = x = e \circ x \, \forall x \in \mathcal{G}$
\end{defn}


\begin{proof}[Proof: unquieness of idenitity]
 Assume some other identity $e^{'}$ exists in $\mathcal{G}$ then, $e^{'} = e^{'} \circ e = e \circ e^{'} = e. \qedhere$
\end{proof}


\begin{exmp}
 Given $\mathcal{G} = \mathbb{N}$ with the binary law of composition $\circ$ to be defined as arithmetic addition $+$.
 Then, $(\mathbb{N}, +)$ forms a semi-group with identity $0$. Verify the axioms.
\end{exmp}


\begin{defn}[Group]
 A $\textbf{group}$ is a monoid where every element has an inverse. An abelian group is a group that is commutative.
\end{defn}

\begin{exmp}
 Given $\mathcal{G} = \mathbb{Z}$ with the binary law of composition $\circ$ to be defined as arithmetic addition $+$.
 Then, $(\mathbb{Z}, +)$ forms a semi-group with identity $0$. Verify the axioms.
\end{exmp}

\begin{question}
 Why does the set of naturals $\mathbb{N}$ not form a group under multiplication, however does form a monoid?
\end{question}

\begin{defn}[Group order]
	If a group $\mathcal{G}$ has $n$ finitely many elements the $\emph{order}$, denoted
	$\left| \mathcal{G} \right| = n$, is the number of elements of $\mathcal{G}$.
\end{defn}

\begin{defn}[Group element order]
	For a element $g$ in some group $\mathcal{G}$ the order of $g$ is defined to be
	the least positive integer $k$ such that $g^k = e$, where $e$ denotes the group identity,
	with respect to the groups law of composition. In symbols, $o(g)=k$. If no such $k$ exists
	then $g$ is said to have infinite order.
\end{defn}

\begin{rem}
	A non-trivial element, $g \neq e$, of finite order, $o(g)=k < \infty$, is called a
	\emph{torsion element} and for when $k=2$ it is called an \emph{involution}.
\end{rem}

\begin{thm}
	Every finite group of even order has a non-trivial involution. That is, for some group
	$\mathcal{G}$ where $|\mathcal{G}| = 2n < \infty$ we have that, there exists some
	non-trivial element $g \neq e$ in $\mathcal{G}$ such that $g^2 = e$.
\end{thm}

\begin{exmp}[Matrix Groups]
	Linear maps of vector spaces form groups that have characteristic properties. For some
	vector space $\mathcal{V}$ over some field $\F$ we may define the following groups taking
	matrix multiplication as the binary law of composition.
	\begin{enumerate}[i.)]
		\item The \emph{General Linear} group defined by,
			\[
				GL(\F) \doteq \{ M \in \mathcal{M} : det(M) \neq 0 \}.
			\]
		\item The \emph{Special Linear} group defined by,
			\[
				SL(\F) \doteq \{ M \in GL(\F) : \det(M) = 1 \}.
			\]
		\item The \emph{Orthogonal} group defined by,
			\[
				O(\R) \doteq \{ M \in GL(\R) : M^{T} M = I \}.
			\]
		\item The \emph{Special Orthogonal} group defined by,
			\[
				SO(\R) \doteq \{ M \in O(\R) : \det(M) = 1 \}.
			\]
		\item The \emph{Unitary} group defined by,
			\[
				U(\C) \doteq \{ M \in GL(\C) : M^{*} M = I \}.
			\]
		\item The \emph{Special Unitary} group defined by,
			\[
				SU(\C) \doteq \{ M \in U(\C) : \det(M) = 1 \}.
			\]
		\item The \emph{Symplectic} group is defined by,
			\begin{align*}
				Sp(2n,\F) & \doteq \{ M \in \mathcal{M}_{2n} : M^{T} \Omega M = \Omega \}
				\intertext{where $\Omega$ is skew-symmetric and defined as the block matrix,}
				\Omega & \doteq
				\begin{pmatrix}
					0 & I_n \\
					-I_n & 0
				\end{pmatrix}.
			\end{align*}
	\end{enumerate}
\end{exmp}

\begin{exmp}[Lorentz Group]
	The Lorentz group is defined as,
	\[
		\mathcal{L}(\R) \doteq
		\{ M \in \mathcal{M}_{2}(\R) : M^{T} C M = C \}
	\]
	where $C$ describes the Lorentz inner product with respect to the
	standard basis, i.e.
	\[
		C =
		\begin{pmatrix}
			1 & 0 \\
			0 & -1
		\end{pmatrix}.
	\]
\end{exmp}

\begin{defn}[Automorphism Group]
	Suppose $\mathcal{S}=(S,*)$ is some algebraic structure and $\mathbb{S}$ is the set
	of automorphism of $\mathcal{S}$. Then we may define the structure $(\mathbb{S}, \circ)$,
	where $\circ$ is defined as functional composition, as the \emph{group of automorphisms}
	of $\mathcal{S}$, denoted $Aut(\mathcal{S})$ or $\mathscr{A}(\mathcal{S})$. That is,
	\[
		\mathscr{A}(\mathcal{S}) \doteq (\mathbb{S}, \circ) \text{ where }
		\mathbb{S} \doteq
		\{ \phi : \mathcal{S} \to \mathcal{S}, \text{ where } \phi \text{ is a isomorphic}. \}
	\]
	For some algebraic structure $\mathcal{S}$ on set $S$.
\end{defn}


% Cyclic Group Section.
% Copyright © 2013 Edward O'Callaghan. All Rights Reserved.
% !Tex root = Algebra II.tex

\subsection{Cyclic Groups}
\label{sec:cyclicgroups}

\subsubsection{Generating Sets}

\begin{defn}[Generating Set]
	For some $S \subseteq \mathcal{G}$ define $S^{-1} = \{ s^{-1} : s \in S \}$
	and let $\langle S \rangle$ denote the set of all elements of
	$\mathcal{G}$ that can be written as finite products of elements of
	$S \cup S^{-1}$. That is,
	\[
		\langle S \rangle
		\doteq \{ g \in \mathcal{G} : g = s_0 \dots s_n \text{ where }
		s_i \in S \cup S^{-1} \}.
	\]
\end{defn}

\begin{lem}
	The generating set $\langle S \rangle$ is a subgroup of $\mathcal{G}$,
	called the \emph{subgroup generated by} $S$.
\end{lem}

\begin{defn}[Finitely Generated]
	Let $\mathcal{G}$ be a group. Then $\mathcal{G}$ is said to be
	\emph{finitely generated} if there is a finite set $S \subseteq \mathcal{G}$
	such that $\mathcal{G} = \langle S \rangle$.
\end{defn}

\begin{exmp}
	Consider the group $\mathcal{G} = \Z_{5}^{\times}$ and notice that
	$\mathcal{G} = \langle 2 \rangle$. Since,
	\begin{align*}
		2^1 &= 2, \\
		2^2 &= 4, \\
		2^3 &= 8 \equiv 3 \pmod 5 , \\
		2^4 &= 16 \equiv 1 \pmod 5
	\end{align*}
	and so the element $2$ is a generator of the multiplicative group
	$\Z_{5} - \{0\}$.
\end{exmp}

\subsubsection{Cyclic Groups}

\begin{defn}[Cyclic group]
	A group $\mathcal{G}$ is $\emph{cyclic}$ if $\mathcal{G}=\mbox{Gp}(g)$ for some $g \in \mathcal{G}$.
	Such a element is called a $\emph{generator}$ of the group.
\end{defn}



\subsection{Permutations} % (fold)
\label{sec:permutations}
Take a finite set X with $|X|=n$, then the transformations of X are called $\textbf{permutations}$ of the
elements of X. In particular, the group of permutations of $X=\{ 1, 2, \cdots, n \}$ is a $\textbf{symmetric group}$,
denoted $S_n$, with $\textbf{order}$ $|S_n|=n!$. Thus, by taking any subgroup of $S_n$ we have a
$\textbf{permutation group}$. Also note that, for finite sets, $\emph{permutation}$ and $\emph{bijective maps}$
refer to the same operation, namely rearrangement of elements of X.
Another way is to consider, a group $\mathcal{G}$ and set X.
Then a group action is defined as a group homomorphism $\varphi$ from $\mathcal{G}$ to the symmetric group of X.
That is, the action $\varphi: \mathcal{G} \to S_n(X)$, assigns a permutation of X to each element of the group $\mathcal{G}$ in the following way:
\begin{itemize}
 \item From the identity element $e \in \mathcal{G}$ to the identity transformation $\Id{X}$ of X, that is, $\varphi : e \to \Id{X}$;
 \item A product of group homomorphisms $\varphi \circ \psi \in \mathcal{G}$ is then the composite of permutations given by $\varphi$ and $\psi$ in X.
\end{itemize}
Given that each element of $\mathcal{G}$ is represented as a permutation. Then a group action can also be consider as a permutation representation.

A permutation $\sigma \in S_n$ can be written,
\[
 \sigma =
 \begin{pmatrix}
  1 & 2 & \cdots & n \\
  a_1 & a_2 & \cdots & a_n
 \end{pmatrix}
 \, \, \,
 \text{where } a_1 = \sigma(1), a_2 = \sigma(2), \cdots .
\]

The identity permutation $\Id{n} \in S_n$ is simply,
\[
 \Id{n} =
 \begin{pmatrix}
  1 & 2 & \cdots & n \\
  1 & 2 & \cdots & n
 \end{pmatrix}
\].

Since $|S_n|=n!$ then the total number of ways n elements maybe permuted is $n!$.

Take any two permutations $\sigma,\pi \in S_n$ then composition is well defined as $\textbf{functional composition}$
as follows.

Given,
\[
 \sigma =
 \begin{pmatrix}
  1 & 2 & \cdots & n \\
  a_1 & a_2 & \cdots & a_n
 \end{pmatrix}
 \, \, and \, \,
 \pi =
 \begin{pmatrix}
  a_1 & a_2 & \cdots & a_n \\
  b_1 & b_2 & \cdots & b_n
 \end{pmatrix}
\]
then,

\begin{align*}
 \pi \circ \sigma &=
 \begin{pmatrix}
  1 & 2 & \cdots & n \\
  \pi(a_1) & \pi(a_2) & \cdots & \pi(a_n)
 \end{pmatrix}
 \\
 &=
 \begin{pmatrix}
  1 & 2 & \cdots & n \\
  b_1 & b_2 & \cdots & b_n
 \end{pmatrix}
\end{align*}
.

A inverse of any permutation $\sigma \in S_n$ is given by,
\[
 \sigma^{-1} =
 \begin{pmatrix}
  1 & 2 & \cdots & n \\
  a_1 & a_2 & \cdots & a_n
 \end{pmatrix}^{-1}
 =
 \begin{pmatrix}
  a_1 & a_2 & \cdots & a_n \\
  1 & 2 & \cdots & n
 \end{pmatrix}
\]

\subsection{Permutation parity}
Consider the algebraic structure:
\[
	\triangle_n (x_1, \dots , x_n) = \prod_{i<j} (x_i - x_j)
\]
TODO..


\subsection{Symmetric Group}
TODO FIX sections??

\begin{defn}[Dihedral group]
	The \emph{dihedral group} $\mathcal{D}_n$ is defined as the symmetries of a regular n-gon.
	The order $| \mathcal{D}_n | = 2n$ as there are $n$ rotations and $n$ reflections.
\end{defn}

\subsection{Group actions} % (fold)
\label{subsec:groupaction}

For any mathematical object (e.g. sets, groups, vector spaces) $X$ an isomorphism
of $X$ is a symmetry of $X$. The set of all isomorphisms of $X$, or symmetries
of $X$, form a group called the symmetry group of $X$, denoted $Sym(X)$.
More formally;

\begin{defn}[Group action]
	An $\emph{action}$ of a group $\mathcal{G}$ on a mathematical object $X$ is
	a mapping $\mathcal{G} \times X \to X$, defined by $(g,x) \mapsto g . x$
	satisfying:
	\begin{itemize}
			\item $e . x = x \, \forall x \in X$ and
			\item $(g h) . x = g . (h . x) \, \forall g,h \in \mathcal{G}, x \in X$.
	\end{itemize}
	That is, we have the ($\emph{left}$) $\mathcal{G}$-action on $X$ and denote this
	by $\mathcal{G} \acts X$.
\end{defn}

Notice that we may study properties of the symmetries of some mathematical object $X$
without reference to the structure of $X$ in particular.


% Subgroup Section.
% Copyright © 2013 Edward O'Callaghan. All Rights Reserved.

\subsection{Subgroups} % (fold)
\label{sec:subgroups}

\begin{defn}[Subgroup]
	A group $\mathcal{H}$ is a $\textbf{subgroup}$ of a group $\mathcal{G}$ if the restriction of the binary operation
	$\circ : \mathcal{H} \times \mathcal{H} \to \mathcal{H}$ is a group operation on $\mathcal{H}$.
	In particular, A non-empty subset $\mathcal{H}$ of a group $\mathcal{G}$ is a subgroup of
	$\mathcal{G}$ if and only if $h_1 \circ h_2 \in \mathcal{H}$ whenever $h_1, h_2 \in \mathcal{H}$,
	and $h^{-1} \in \mathcal{H}$ whenever $h \in \mathcal{H}$. We denote the subgroup by $\mathcal{H} \leq \mathcal{G}$.
\end{defn}

\begin{thm}[Smallest subgroup]
	If $\mathcal{A}$ is a subset of a group $\mathcal{G}$, there is a $\emph{smallest}$ subgroup
	$\mbox{Gp}(\mathcal{A})$ of $\mathcal{G}$ which contains $\mathcal{A}$, the subgroup $\emph{generated}$
	by $\mathcal{A}$.
\end{thm}

\begin{exmp}
	Suppose $\mathcal{A}=\{g\}$ then $\mbox{Gp}(\mathcal{A})=\mbox{Gp}(g)$ and so
	$\mbox{Gp}(g) = \{ g^n : n \in \Z \}$, where $g^0 = e$, $g^n$ is the product of $n$ copies of $g$
	where $n>0$, and $g^n$ is the product of $|n|$ copies of $g^{-1}$ when $n<0$.
\end{exmp}

\begin{defn}[Normal subgroup]
	A subgroup $\mathcal{H}$ of a group $\mathcal{G}$ is a $\textbf{normal}$, or $\emph{self-conjugate}$, if
	$g h g^{-1} = h$ for all $g \in \mathcal{G}$ and for all $h \in \mathcal{H}$. We denote the normal
	$\mathcal{H} \unlhd \mathcal{G}$.
\end{defn}

\begin{defn}[Simple group]
	A group $\mathcal{G}$ is $\textbf{simple}$ if it has no normal subgroups other than the trivial normal
	subgroups $\{e\}$ and $\mathcal{G}$.
\end{defn}


\subsubsection{Sylow's Theorems}

The Norwegian mathematician \emph{Ludwig Sylow} established
some important results while investigating subgroups of prime
order.

\begin{defn}[p-subgroup]
	TODO.
\end{defn}

\begin{defn}[Sylow p-subgroup]
	TODO.
\end{defn}

\begin{thm}[First Sylow Theorem]
	Let $p$ be prime and $\mathcal{G}$ be a group such that
	$\left| \mathcal{G} \right| = k p^n$ where $p\not| k$.
	Then $\mathcal{G}$ has \emph{at least} one Sylow
	p-subgroup.
\end{thm}

\begin{thm}[Second Sylow Theorem]
	Let $P$ be a Sylow p-subgroup of some finite group $\mathcal{G}$.
	Let $Q$ be any p-subgroup of $\mathcal{G}$. Then $Q$ is contained
	in a conjugate of $P$.
\end{thm}

\begin{thm}[Third Sylow Theorem]
	All the Sylow p-subgroups of a finite group are conjugate.
\end{thm}

\begin{thm}[Fourth Sylow Theorem]
	The number of Sylow p-subgroups of a finite group is congruent to
	$1(\mod p)$.
\end{thm}

\begin{thm}[Fifth Sylow Theorem]
	The number of Sylow p-subgroups of a finite group is a divisor of
	their common subgroup index.
\end{thm}

%%%%%
% TODO: Improve me and perhaps MOVE ME..?
We now look at a representation theorem for groups known as Cayley's Theorem.
This theorem informs us that; In order to study finite groups it is only
necessary to study subgroups of the symmetric group. In particular,

\begin{thm}[Cayley's Theorem]
	Let $S_n$ denote the symmetric group on $n$ letters. Every finite group
	is isomorphic to a subgroup of $S_n$ for some $n \in \Z$.
\end{thm}

\begin{proof}
	Let $\mathcal{H} = \{e\}$. By applying permutation of Cosets to $\mathcal{H}$
	so that $\mathbb{S} = \mathcal{G}$ and $\ker (\theta) = \{e\}$. The result
	follows by the First Isomorphism Theorem. \qedhere
\end{proof}
%%%%%

\begin{defn}[Characteristic Subgroup]
	Let $\mathcal{G}$ be a group and $\mathcal{H}$ be a subgroup $\mathcal{H} \leq \mathcal{G}$
	such that for every $\phi \in Aut(\mathcal{G})$ we have $\phi (\mathcal{H})=\mathcal{H}$,
	where $Aut(\mathcal{G})$ denotes the group of automorphisms of $\mathcal{G}$. Then $\mathcal{H}$
	is \emph{characteristic in} $\mathcal{G}$, or \emph{a characteristic subgroup of} $\mathcal{G}$.
\end{defn}

\begin{thm}[Characteristic Subgroup Transivity]
	Suppose $\mathcal{G}$ is a group and let $\mathcal{H}$ be a characteristic subgroup of $\mathcal{G}$
	and $\mathcal{K}$ a characteristic subgroup of $\mathcal{H}$. Then $\mathcal{K}$ is a characteristic
	subgroup of $\mathcal{G}$.
\end{thm}

\begin{proof}
	Let $\phi : \mathcal{G} \to \mathcal{G}$ be a group automorphism. Since $\mathcal{H}$ is a characteristic
	subgroup of $\mathcal{G}$, by definition, we have that
	\[
		\phi(\mathcal{H})=\mathcal{H}.
	\]
	That is, the restriction of $\phi$ to $\mathcal{H}$, written $\phi |_{\mathcal{H}}$, is a automorphism
	of $\mathcal{H}$. Now, since $\mathcal{K}$ is a characteristic subgroup of $\mathcal{H}$, we have that
	\begin{align*}
		\phi |_{\mathcal{H}}(\mathcal{K}) &= \mathcal{K}
		\\
		\Rightarrow \phi(\mathcal{K}) &= \mathcal{K}
	\end{align*}
	and so $\mathcal{K}$ is a characteristic subgroup pf $\mathcal{G}$.
\end{proof}


% Group Homomorphism Section.
% Copyright © 2013 Edward O'Callaghan. All Rights Reserved.

\subsection{Group Homomorphisms} % (fold)
\label{subsec:homomorphisms}

Homomorphisms are structure preserving mappings. In group homomorphisms we preserve the
group structure, defined by the binary law of composition. In particular,

\begin{defn}[Group Homomorphism]
	Let $(\mathcal{G}, \circ)$ and $(\mathcal{H},\dagger)$ be two groups. Then a mapping
	$\varphi : \mathcal{G} \to \mathcal{H}$ is called a $\emph{group homomorphism}$ if
	\[
		\varphi(g_1 \circ g_2) = \varphi(g_1) \dagger \varphi(g_2) : g_1,g_2 \in \mathcal{G}.
	\]
\end{defn}

It follows that, for some $g \in \mathcal{G}$ we have,
\begin{align*}
	\varphi ( e_g ) &= \varphi (g \circ g^{-1})
	\\
	&= \varphi (g) \dagger \varphi (g^{-1})
	\\
	&= \varphi (g) \dagger (\varphi (g) )^{-1}
	\\
	&= e_h \in \mathcal{H}.
\end{align*}
That is the identity $e$ has been preserved.

In this way, it does not matter if we compose in $\mathcal{G}$ and map to $\mathcal{H}$ or take two elements
in $\mathcal{G}$ then compose the mapped elements in $\mathcal{H}$, since the group structure has been preserved.

How much information about the elements inside the structure is, however, another quality to consider. Hence we fix some
terminology here.
\begin{itemize}
	\item A homomorphism that is injective is called monomorphic.
	\item A homomorphism that is surjective is called epimorphic.
	\item A homomorphism that is bijective is called isomorphic.
\end{itemize}
Thus we have the following definitions by considering a group homomorphism $\varphi : \mathcal{G} \to \mathcal{H}$.

\begin{defn}[Monomorphic]
	$\varphi$ is $\textbf{monomorphic}$ if for $\varphi(x) = \varphi(y) \implies x = y \, \forall x,y \in \mathcal{G}$.
\end{defn}

\begin{defn}[Epimorphic]
	$\varphi$ is $\textbf{epimorphic}$ if $\forall h \in \mathcal{H} \exists g \in \mathcal{G}$ so that $\varphi(g) = h$.
\end{defn}

\begin{defn}[Isomorphic]
	$\varphi$ is $\textbf{isomorphic}$ if $\varphi$ is $\textbf{both}$ mono- and epic- morphic.
\end{defn}

Some special cases are sometimes of particular interest and we shall outline them now.

\begin{defn}[Endomorphic]
	A monomorphism $\mathcal{G} \to \mathcal{G}$ for a group $\mathcal{G}$
	is called an $\emph{endomorphism}$ of $\mathcal{G}$.
\end{defn}

\begin{defn}[Automorphic]
	A isomorphism $\mathcal{G} \to \mathcal{G}$ for a group $\mathcal{G}$
	is called an $\emph{automorphism}$ of $\mathcal{G}$.
\end{defn}

\begin{rem}
	The set $Aut(\mathcal{G})$ of automorphisms of $\mathcal{G}$ forms a group, when composition of
	mappings is taken as the group law of composition.
\end{rem}

\begin{exmp}[Trivial Homomorphism]
	The trivial group homomorphism $id_{\mathcal{G}} : \mathcal{G} \to \mathcal{G}$, given
	by the mapping $g \mapsto g$ for every $g \in \mathcal{G}$, is in fact a group automorphism.
\end{exmp}

\begin{exmp}
	Consider $\psi : GL_n(\R) \to \R^{\times}$ defined by the mapping
	$A \mapsto \det(A)$ and recall that $\det(AB) = \det(A) \det(B)$.
	That is, the determinant is a group homomorphism.
\end{exmp}

\begin{exmp}
	Consider $\psi : \mathcal{G} \to S_n / A_n$ where $\mathcal{G}=\{-1,1\}$, defined by
	$1 \mapsto A_n$ and $-1 \mapsto (1 \, 2) A_n$, and observe that $\phi$ is a group homomorphism.
\end{exmp}

\begin{prob}
	Consider the map $\phi : \R \to SL_2(\R)$ defined by,
	\[
		x \mapsto
		\begin{pmatrix}
			1 & x \\
			0 & 1
		\end{pmatrix}.
	\]
	Show that $\phi(x + y) = \phi(x) \cdot \phi(y)$.
	Also, prove that $\phi$ is injective.
\end{prob}

\begin{exmp}
	Consider the map $\exp : \R^{+} \to \R^{\times}$ from the additive to the
	multiplicative group, defined by $x \mapsto e^x$, is a group homomorphism.
	Since, $\exp{(x + y)} = \exp{(x)} \cdot \exp{(y)}$.
\end{exmp}

\begin{exmp}
	Consider the linear transformation $T : \mathcal{V} \to \mathcal{W}$.
	By definition of linearity, $T(\vec{v}_1 + \vec{v}_2) = T(\vec{v}_1) + T(\vec{v}_2)$,
	the mapping $T$ is a group homomorphism from the additive group of vector space $\mathcal{V}$
	to the additive group of vector space $\mathcal{W}$.
\end{exmp}

\begin{prob}
	Suppose $N \unlhd \mathcal{G}$ and $\pi : \mathcal{G} \to \mathcal{G} / N$,
	given by the mapping $g \mapsto g N$ for every $g \in \mathcal{G}$. Show that $\pi$
	is a group homomorphism and then show that it is surjective.
\end{prob}

\begin{prob}
	Suppose $\phi : \C^{\times} \to \R^{\times}$ given by the mapping $z \mapsto | z |$. Show
	that $\phi$ is a group homomorphism. Is $\phi$ bijective?
\end{prob}

\begin{prop}
	Let $\varphi : \mathcal{G} \to \mathcal{H}$ be a group homomorphism.
	\begin{enumerate}[i.)]
		\item $\varphi(1_{\mathcal{G}}) = 1_{\mathcal{H}}$,
		\item $\varphi(g^{-1}) = \varphi(g)^{-1}$ for all $g \in \mathcal{G}$,
		\item If $\mathcal{G}' \leq \mathcal{G}$ then $\varphi(\mathcal{G}') \leq \mathcal{H}$ when the restriction
			$\mathcal{H} = \varphi |_{\mathcal{G}'} (\mathcal{G})$ holds,
		\item If $\varphi$ is an isomorphism, then so is its inverse $\varphi^{-1} : \mathcal{H} \to \mathcal{G}$,
		\item If $\psi : \mathcal{G} \to \mathcal{H}$ and $\phi : \mathcal{H} \to \mathcal{K}$ are group
			homomorphisms then so is $\phi \circ \psi$.
	\end{enumerate}
\end{prop}

\begin{proof}
	For $i.)$ we see that,
	\begin{align*}
		1_{\mathcal{H}} \cdot \varphi(1_{\mathcal{G}}) &= \varphi(1_{\mathcal{G}}) \tag{and that} \\
		\varphi(1_{\mathcal{G}}) &= \varphi(1_{\mathcal{G}} \circ 1_{\mathcal{G}}) \\
		&= \varphi(1_{\mathcal{G}}) \cdot \varphi(1_{\mathcal{G}})
		\intertext{so we have that,}
		1_{\mathcal{H}} \cdot \varphi(1_{\mathcal{G}}) &= \varphi(1_{\mathcal{G}}) \\
		\Rightarrow 1_{\mathcal{H}} \cdot \varphi(1_{\mathcal{G}}) \cdot \varphi(1_{\mathcal{G}})^{-1}
		&= \varphi(1_{\mathcal{G}}) \cdot \varphi(1_{\mathcal{G}})^{-1} \\
		\Rightarrow 1_{\mathcal{H}} &= \varphi(1_{\mathcal{G}}). \qedhere
	\end{align*}
\end{proof}

\begin{proof}
	For $ii.)$ we see that,
	\begin{align*}
		g g^{-1} = 1_{\mathcal{G}} &= g^{-1} g \\
		\Rightarrow 1_{\mathcal{H}} &= \varphi(g) \varphi(g^{-1}) \\
		&= \varphi(g^{-1}) \varphi(g).
		\intertext{Hence,}
		\varphi(g^{-1}) &= \varphi(g)^{-1}. \qedhere
	\end{align*}
\end{proof}

\begin{proof}
	For $iii.)$ we check for closure and inverses as follows.
	Define:
	\begin{align*}
		\varphi(\mathcal{G}') & \doteq \{ \varphi(g) : g \in \mathcal{G}' \} \\
		\varphi(g) \varphi(g') &= \varphi(g g') \in \varphi(\mathcal{G}') : g,g' \in \mathcal{G}'
		\intertext{and so $\varphi(\mathcal{G}')$ is closed. Now given,}
		\varphi(g) & \in \varphi(\mathcal{G}') : g \in \mathcal{G}'
		\intertext{then we have that,}
		\varphi(g)^{-1} = \varphi(g^{-1}) & \in \varphi(\mathcal{G}') : g^{-1} \in \mathcal{G}'
		\intertext{and so $\varphi(\mathcal{G}')$ has inverses. Hence $\varphi(\mathcal{G}')$ is a subgroup.}
	\end{align*}
\end{proof}

\begin{proof}
	For $iv.)$ we consider the mapping $\varphi^{-1} : \mathcal{H} \to \mathcal{G}$ defined by
	$h \mapsto g$ if $\varphi(g)=h$. First we check that $\varphi^{-1}$ is a group homomorphism, that is:
	\begin{align*}
		\varphi(h h')^{-1} &= \varphi(h)^{-1} \varphi(h')^{-1}.
		\intertext{Suppose that $g=\varphi(h)^{-1}$ and $g'=\varphi(h')^{-1}$ and since,}
		\varphi(g g') &= \varphi(g) \varphi(g') \\
		&= h h' \\
		\Rightarrow \varphi(h h')^{-1} &= g g' \\
		&= \varphi(h)^{-1} \varphi(h')^{-1}
	\end{align*}
	and so $\varphi^{-1}$ is a group homomorphism. It now trivially follows that $\varphi^{-1}$ is
	isomorphic if $\varphi$ is isomorphic. \qedhere
\end{proof}

\begin{proof}
	For $v.)$ we see directly that, given $\psi(g g') = \psi(g) \cdot \psi(g')$ and
	$\phi(h h') = \phi(h) \dagger \phi(h')$, we have that:
	\begin{align*}
		\phi \circ \psi (g g') &= \phi ( \psi(g) \cdot \psi(g') ) \\
		&= \phi \circ \psi(g) \dagger \phi \circ \psi(g') \\
		&= \phi(h) \dagger \phi(h') = \phi(h h'). \qedhere
	\end{align*}
\end{proof}

\begin{defn}[kernel]
	If $\varphi : \mathcal{G} \to \mathcal{H}$ is a group homomorphism,
	then the $\emph{kernel}$ is the set
	$\ker(\varphi) = \{ g \in \mathcal{G} : \varphi (g) = e_h \in \mathcal{H} \}$.
\end{defn}

If $\varphi : \mathcal{G} \to \mathcal{H}$ is a group homomorphism, then
observe that $\ker(\varphi)$ is a normal subgroup of of $\mathcal{G}$.

% Copyright © 2013 Edward O'Callaghan. All Rights Reserved.

\section{Characters} % (fold)
\label{sec:characters}
A \emph{group character} is a group homomorphism,
$\chi : \mathcal{G} \to \C^{\times}$, from a finite abelian group
to the multiplicative group of nonzero complex numbers.
In particular;

\begin{defn}[Character]
	Let $\mathcal{G}$ be a finite abelian group of order $n$, written additively.
	A \emph{character} of $\mathcal{G}$ is a group homomorphism,
	$\chi : \mathcal{G} \to \C^{\times}$, of $\mathcal{G}$ such that:

	\[
		\chi (g_1 + g_2) = \chi(g_1) \chi(g_2) : g_1,g_2 \in \mathcal{G}.
	\]
\end{defn}

\begin{lem}
	\begin{align*}
		\chi(g)^n &= \chi(ng)
		\\
		&= \chi(0) = 1 : g \in \mathcal{G}.
	\end{align*}
	Hence the values of $\chi$ are the $n^{th}$ roots of unity.
\end{lem}

\begin{lem}
	\begin{align*}
		\chi(-g) &= \chi(g)^{-1}
		\\
		&= \overline{\chi(g)}
	\end{align*}
	where the bar denotes the complex conjugation.
\end{lem}

\begin{defn}[Principle Character]
	The \emph{principle character}, denoted by $\chi_{0}$, is defined by
	\[
		\chi_{0}(g) \doteq 1 : g \in \mathcal{G}.
	\]
\end{defn}

\begin{prop}
	For any non-principle character $\chi$ of $\mathcal{G}$,
	\[
		\sum_{g \in \mathcal{G}} \chi(g) = 0.
	\]
\end{prop}

\begin{proof}
	Let $h \in \mathcal{G} : \chi(h) \neq 1$ and let
	$S = \sum_{g \in \mathcal{G}} \chi(g)$. Then,
	\begin{align*}
		\chi(h) \cdot S &= \chi(h) \sum_{g \in \mathcal{G}} \chi(g)
		\\
		&= \sum_{g \in \mathcal{G}} \chi(h) \chi(g)
		\\
		&= \sum_{g \in \mathcal{G}} \chi(g + h)
		\\
		&= S.
		\intertext{Hence it follows that,}
		\chi(h) \cdot S &= S
		\\
		\left( \chi(h) - 1 \right) \cdot S &= 0
		\intertext{and since $\chi(h) \neq 1$ then,}
		\Rightarrow S &= 0. \qedhere
	\end{align*}
\end{proof}

\begin{cor}[First orthogonality relation for characters]
	Let $\chi$ and $\psi$ be two characters of $\mathcal{G}$. Then
	\[
		\sum_{g \in \mathcal{G}} \overline{\chi(g)} \psi(g) =
		\begin{cases}
			n & \text{if } \chi = \psi , \\
			0 & \text{otherwise}.
		\end{cases}
	\]
\end{cor}

\begin{proof}
	Consider the two cases.
	\begin{enumerate}[i.)]
		\item
			For when $\chi = \psi$ it is trivially so, by that,
			\begin{align*}
				\overline{\chi(g)} &=  \chi(g)^{-1}
				\\
				\Rightarrow \overline{\chi(g)} \chi(g) &= 1 \tag{for each $g \in \mathcal{G}$}
				\intertext{and that $\left| \mathcal{G} \right| = n$.}
			\end{align*}
		\item
			If $\chi \neq \psi$ then $\overline{\chi} \psi$ is a
			non-principle character and so $\overline{\chi(g)} \psi(g) = 0$
			for each $g \in \mathcal{G}$. \qedhere
	\end{enumerate}
\end{proof}

\begin{rem}
	As observed in the last proof, the point wise product of the
	characters $\chi$ and $\psi$ is again a character:
	\[
		(\chi \psi)(g) \doteq \chi(g) \psi(g).
	\]
\end{rem}

\begin{prob}
	Let $\hat{\mathcal{G}}$ denote the set of characters.
	Check that $\hat{\mathcal{G}}$ forms an abelian group
	under the operation defined by
	$(\chi \psi)(g) \doteq \chi(g) \psi(g)$ for every $g \in \mathcal{G}$.
	We call $\hat{\mathcal{G}}$ the \emph{dual group} of $\mathcal{G}$.
\end{prob}

\begin{prop}
	Let $\omega$ be a \emph{primitive} $n^{th}$ root of unity. Then the
	map $\chi_j : \Z_n \to \C^{\times}$ defined by $\chi_j(k) = \omega^{k j} : k \in \Z_n$
	is a character of $\Z_n$ for every $j \in \Z$. Moreover,
	\begin{enumerate}
		\item $\chi_i = \chi_j \Leftrightarrow i \equiv j (\mod n)$;
		\item $\chi_j = \chi_{1}^{j}$;
		\item $\hat{\Z_n} = \{ \chi_0, \dots , \chi_{n-1} \}$;
		\item Consequently, $\hat{\Z_n} = \Z_n$. % FIXME: should be isomorphic symbole?
	\end{enumerate}
\end{prop}

\begin{proof}
	TODO..
\end{proof}

\begin{prop}
	If $\mathcal{G}$ is a direct sum, $\mathcal{G} = H_1 \oplus H_2$, and
	$\psi_i : H_i \to \C^{\times}$ is a character of $H_i$, with $i \in \{1,2\}$,
	then $\chi = \psi_1 \oplus \psi_2$, defined by
	\[
		\chi(h_1,h_2) \doteq \psi_1(h_1) \cdot \psi_2(h_2),
	\]
	is a character of $\mathcal{G}$. Moreover, all characters of $\mathcal{G}$ are
	of this form. Consequently,
	\[
		\hat{\mathcal{G}} = \hat{H_1} \oplus \hat{H_2}.
	\]
\end{prop}

\begin{proof}
	TODO..
\end{proof}

\begin{cor}
	\[
		\hat{\mathcal{G}} = \mathcal{G}. % FIXME: Should be isomorphic symbole!
	\]
\end{cor}

\begin{proof}
	TODO..
\end{proof}


% Cosets Section.
\subsection{Cosets} % (fold)
\label{subsec:cosets}

% TODO: Explain Equivalence relations and partitions

% TODO: Explain Cosets and Index

Let $\mathcal{G}$ be a group and $\mathcal{H}$ be a subgroup of $\mathcal{G}$ with $g \in \mathcal{G} : g \notin H$, then
\begin{defn}[Left Coset]
	$gH = \{gh : h \in H\}$ is a $\textbf{left coset of } \mathcal{H}$ in $\mathcal{G}$.
\end{defn}

\begin{defn}[Right Coset]
	$Hg = \{hg : h \in H\}$ is a $\textbf{right coset of } \mathcal{H}$ in $\mathcal{G}$.
\end{defn}

%% TODO: Should normal subgroup be defined here???
\begin{defn}[Normal Subgroup]
	If $gH = Hg$ then $\mathcal{H}$ is a $\textbf{normal}$ subgroup of $\mathcal{G}$, denoted by $\mathcal{H} \unlhd \mathcal{G}$.
\end{defn}
%%

\begin{thm}[Lagrange's Theorem]
	TODO.
\end{thm}

\begin{proof}
	TODO.
\end{proof}


\subsection{Factor (or Quotient) groups} % (fold)
\label{subsec:factorgroups}
Let $\mathcal{G}$ be a commutative group and consider a subgroup $\mathcal{H}$.
Then $\mathcal{H}$ determines an equivalence relation in $\mathcal{G}$ given by
\[
	x \sim x' \mbox{ iff } x - x' \in \mathcal{H}.
\]
..

\subsection{Non-commutative Groups} % (fold)
\label{sec:noncommutative-groups}
A common class of non-commutative groups are transformation groups.
Note:
\begin{defn}[Transformation]
 A bijective map $\varphi: X \to X$ is called a $\textbf{transformation}$ of X.
 \begin{note}
  The most trivial case is the $\emph{idenitity map} \, \Id{X}$ by $\Id{X}(x) = x, \, \forall x \in X$.
 \end{note}
\end{defn}
Hence, there exists a inverse $\varphi^{-1}$ of $\varphi$ such that $\varphi^{-1} \circ \varphi = \Id{X} = \varphi \circ \varphi^{-1}$.
Now, take two transformations of X, $\varphi$ and $\psi$, and let the product $\varphi \circ \psi$ be well defined.
Then the set of all transformations of X form the group $\textbf{Transf(X)}$.
Since, given $\varphi , \psi , \phi \in Transf(X)$ then we have associativity, $\varphi \circ (\psi \circ \phi) = (\varphi \circ \psi) \circ \phi$.
We have identity $e = \Id{X} \in Transf(X)$ and so, inverses $\forall \varphi \in Transf(X) \exists ! \varphi^{-1} : \varphi \circ \varphi^{-1} = e$.
Closure follows from the composition of two transformations $\varphi$ and $\psi$, since $(\varphi \circ \psi)^{-1} = \psi^{-1} \circ \varphi^{-1}$.

A transformation group is a type of group action which describes symmetries of objects. More abstractly,
since a group $\mathcal{G}$ is a category with a single object in which every morphism is bijective.
Then, a group action is a $\emph{forgetful functor}$ $\mathcal{F}$ from the group $\mathcal{G}$ in the category $\textbf{Grp}$
to the set category $\textbf{Set}$ that is, $\mathcal{F} : \mathcal{G} \to \textbf{Set}$.
